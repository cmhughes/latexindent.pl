% arara: pdflatex: { files: [latexindent]}
\appendix
 \section{Required Perl modules}\label{sec:requiredmodules}
  If you intend to use \texttt{latexindent.pl} and \emph{not} one of the supplied
  standalone executable files (\texttt{latexindent.exe} is available for Windows users
  without Perl, see \cref{subsubsec:latexindent:exe}), then you will need a few standard
  Perl modules.

  If you can run the minimum code in \cref{lst:helloworld} as in

  \begin{commandshell}
perl helloworld.pl
  \end{commandshell}

  then you will be able to run \texttt{latexindent.pl}, otherwise you may need to install
  the missing modules; see \cref{sec:module-installer,sec:manual-module-instal}.

  \begin{cmhlistings}[style=tcblatex,language=Perl]{\texttt{helloworld.pl}}{lst:helloworld}
#!/usr/bin/perl

use strict;                         #     |
use warnings;                       #     |
use Encode;                         #     |
use Getopt::Long;                   #     |
use Data::Dumper;                   #  these modules are
use List::Util qw(max);             #  generally part
use PerlIO::encoding;               #  of a default perl distribution
use open ':std', ':encoding(UTF-8)';#     |
use Text::Wrap;                     #     |
use Text::Tabs;                     #     |
use FindBin;                        #     |
use File::Copy;                     #     |
use File::Basename;                 #     |
use File::Path;                     #     |
use File::HomeDir;                  # <--- typically requires install via cpanm
use YAML::Tiny;                     # <--- typically requires install via cpanm

print "hello world";
exit;
\end{cmhlistings}

 \subsection{Module installer script}\label{sec:module-installer}
  \announce{2018-01-13}{perl module helper script} \texttt{latexindent.pl} ships with a
  helper script that will install any missing \texttt{perl} modules on your system; if you
  run

  \begin{commandshell}
perl latexindent-module-installer.pl
\end{commandshell}

  or
  \begin{dosprompt}
perl latexindent-module-installer.pl
 \end{dosprompt}
  then, once you have answered \texttt{Y}, the appropriate modules will be installed onto
  your distribution.

 \subsection{Manually installing modules}\label{sec:manual-module-instal}
  Manually installing the modules given in \cref{lst:helloworld} will vary depending on
  your operating system and \texttt{Perl} distribution.

 \subsubsection{Linux}
  \paragraph{perlbrew}
   Linux users may be interested in exploring Perlbrew \cite{perlbrew}; an example
   installation would be:

   \begin{commandshell}
sudo apt-get install perlbrew
perlbrew init
perlbrew install perl-5.34.0
perlbrew switch perl-5.34.0
sudo apt-get install curl
curl -L http://cpanmin.us | perl - App::cpanminus
cpanm YAML::Tiny
cpanm File::HomeDir
\end{commandshell}

   \index{cpan}

  \paragraph{Ubuntu/Debian}
   For other distributions, the Ubuntu/Debian approach may work as follows

   \begin{commandshell}
sudo apt install perl
sudo cpan -i App::cpanminus
sudo cpanm YAML::Tiny
sudo cpanm File::HomeDir
\end{commandshell}

   or else by running, for example,

   \begin{commandshell}
sudo perl -MCPAN -e'install "File::HomeDir"'
\end{commandshell}

  \paragraph{Ubuntu: using the texlive from apt-get}
   Ubuntu users that install texlive using \texttt{apt-get} as in the following

   \begin{commandshell}
sudo apt install texlive
sudo apt install texlive-latex-recommended
\end{commandshell}

   may need the following additional command to work with \texttt{latexindent.pl}

   \begin{commandshell}
sudo apt install texlive-extra-utils
\end{commandshell}

  \paragraph{Ubuntu: users without perl}

   \texttt{latexindent-linux} is a standalone executable \announce{2022-10-30}{Linux Ubuntu standalone executable} for Ubuntu Linux (and therefore does not require a Perl distribution)
   and caches copies of the Perl modules onto your system. It is available from \cite{latexindent-home}.
   \index{latexindent-linux} \index{linux} \index{TeXLive}

  \paragraph{Arch-based distributions}
   First install the dependencies

   \begin{commandshell}
sudo pacman -S perl cpanminus
\end{commandshell}

   In addition, install \texttt{perl-file-homedir} from AUR, using your AUR helper of
   choice,

   \begin{commandshell}
sudo paru -S perl-file-homedir
\end{commandshell}

   then run the latexindent-module-installer.pl file located at helper-scripts/

  \paragraph{Alpine}
   If you are using Alpine, some \texttt{Perl} modules are not build-compatible with
   Alpine, but replacements are available through \texttt{apk}. For example, you might
   use the commands given in \cref{lst:alpine-install}; thanks to \cite{jun-sheaf} for
   providing these details.

   \begin{cmhlistings}[style=tcblatex,language=Bash]{\texttt{alpine-install.sh}}{lst:alpine-install}
# Installing perl
apk --no-cache add miniperl perl-utils

# Installing incompatible latexindent perl dependencies via apk
apk --no-cache add \
    perl-log-dispatch \
    perl-namespace-autoclean \
    perl-specio \
    perl-unicode-linebreak

# Installing remaining latexindent perl dependencies via cpan
apk --no-cache add curl wget make
ls /usr/share/texmf-dist/scripts/latexindent
cd /usr/local/bin && \
    curl -L https://cpanmin.us/ -o cpanm && \
    chmod +x cpanm
cpanm -n App::cpanminus
cpanm -n File::HomeDir
cpanm -n Params::ValidationCompiler
cpanm -n YAML::Tiny
\end{cmhlistings}

   Users of NixOS might like to see
   \href{https://github.com/cmhughes/latexindent.pl/issues/222}{https://github.com/cmhughes/latexindent.pl/issues/222}
   for tips.
 \subsubsection{Mac}
  Users of the Macintosh operating system might like to explore the following commands,
  for example:

  \begin{commandshell}
brew install perl
brew install cpanm

cpanm YAML::Tiny
cpanm File::HomeDir
\end{commandshell}

  Alternatively,

  \begin{commandshell}
brew install latexindent
\end{commandshell}

  \texttt{latexindent-macos} is a standalone executable \announce{2022-10-30}{macOS standalone executable} for macOS (and therefore does not require a Perl distribution)
  and caches copies of the Perl modules onto your system. It is available from \cite{latexindent-home}.
  \index{latexindent-macos} \index{macOS} \index{TeXLive}

 \subsubsection{Windows}
  Strawberry Perl users on Windows might use \texttt{CPAN client}. All of the modules are
  readily available on CPAN \cite{cpan}. \texttt{indent.log} will contain details of the
  location of the Perl modules on your system.

  \texttt{latexindent.exe} is a standalone executable
  for Windows (and therefore does not require a Perl distribution) and caches copies of the
  Perl modules onto your system; if you wish to see where they are cached, use the
  \texttt{trace} option, e.g
  \begin{dosprompt}
latexindent.exe -t myfile.tex
 \end{dosprompt}

 \subsection{The GCString switch}\label{subsec:the-GCString}
  If you find that the \texttt{lookForAlignDelims} (as in \cref{subsec:align-at-delimiters}) does not work correctly
  for your language, then you may wish to use the \texttt{Unicode::GCString} module \announce{2022-03-25}{Unicode::GCString}.
  \index{perl!Unicode GCString module}
  \index{switches!--GCString demonstration}

  This can be loaded by calling \texttt{latexindent.pl} with the \texttt{GCString} switch
  as in

  \begin{commandshell}
latexindent.pl --GCString myfile.tex
\end{commandshell}

  In this case, you will need to have the \texttt{Unicode::GCString} installed in your
  \texttt{perl} distribution by using, for example,

  \begin{commandshell}
cpanm Unicode::GCString
\end{commandshell}

  Note: this switch does \emph{nothing} for \texttt{latexindent.exe} which loads the
  module by default. Users of \texttt{latexindent.exe} should not see any difference in
  behaviour whether they use this switch or not, as \texttt{latexindent.exe} loads the
  \texttt{Unicode::GCString} module.

 \section{Updating the path variable}\label{sec:updating-path}
  \texttt{latexindent.pl} has a few scripts (available at \cite{latexindent-home}) that can
  update the \texttt{path} variables. Thank you to \cite{jasjuang} for this feature. If
  you're on a Linux or Mac machine, then you'll want \texttt{CMakeLists.txt} from
  \cite{latexindent-home}.
 \subsection{Add to path for Linux}
  To add \texttt{latexindent.pl} to the path for Linux, follow these steps:
  \begin{enumerate}
   \item download \texttt{latexindent.pl} and its associated modules,
         \texttt{defaultSettings.yaml}, to your chosen directory from
         \cite{latexindent-home} ;
   \item within your directory, create a directory called \texttt{path-helper-files} and
         download \texttt{CMakeLists.txt} and \lstinline!cmake_uninstall.cmake.in! from
         \cite{latexindent-home}/path-helper-files to this directory;
   \item run

         \begin{commandshell}
ls /usr/local/bin
\end{commandshell}

         to see what is \emph{currently} in there;
   \item run the following commands

         \begin{commandshell}
sudo apt-get update
sudo apt-get install --no-install-recommends cmake make # or any other generator
mkdir build && cd build
cmake ../path-helper-files
sudo make install
\end{commandshell}

   \item run

         \begin{commandshell}
ls /usr/local/bin
\end{commandshell}

         again to check that \texttt{latexindent.pl}, its modules and
         \texttt{defaultSettings.yaml} have been added.
  \end{enumerate}
  To \emph{remove} the files, run

  \begin{commandshell}
sudo make uninstall
\end{commandshell}

 \subsection{Add to path for Windows}
  To add \texttt{latexindent.exe} to the path for Windows, follow these steps:
  \begin{enumerate}
   \item download \texttt{latexindent.exe}, \texttt{defaultSettings.yaml},
         \texttt{add-to-path.bat} from \cite{latexindent-home} to your chosen directory;
   \item open a command prompt and run the following command to see what is
         \emph{currently} in your \lstinline!%path%! variable;
         \begin{dosprompt}
echo %path%
          \end{dosprompt}
   \item right click on \texttt{add-to-path.bat} and \emph{Run as administrator};
   \item log out, and log back in;
   \item open a command prompt and run
         \begin{dosprompt}
echo %path%
          \end{dosprompt}
         to check that the appropriate directory has been added to your
         \lstinline!%path%!.
  \end{enumerate}
  To \emph{remove} the directory from your \lstinline!%path%!, run
  \texttt{remove-from-path.bat} as administrator.

 \section{Batches of files}\label{sec:batches}

  You can \announce{2022-03-25}{batches of files details} instruct
  \texttt{latexindent.pl} to operate on multiple files. For example, the following calls
  are all valid

  \begin{commandshell}
latexindent.pl myfile1.tex
latexindent.pl myfile1.tex myfile2.tex
latexindent.pl myfile*.tex
        \end{commandshell}

  We note the following features of the script in relation to the switches detailed in
  \cref{sec:how:to:use}.

 \subsection{location of indent.log}
  If the \texttt{-c} switch is \emph{not} active, then \texttt{indent.log} goes to the directory of the final file called.

  If the \texttt{-c} switch \emph{is} active, then \texttt{indent.log} goes to the
  specified directory.

 \subsection{interaction with -w switch}
  If the \texttt{-w} switch is active, as in

  \begin{commandshell}
latexindent.pl -w myfile*.tex
\end{commandshell}

  then files will be overwritten individually. Back-up files can be re-directed via the
  \texttt{-c} switch.

 \subsection{interaction with -o switch}
  If \texttt{latexindent.pl} is called using the \texttt{-o} switch as in

  \begin{commandshell}
latexindent.pl myfile*.tex -o=my-output-file.tex
        \end{commandshell}

  and there are multiple files to operate upon, then the \texttt{-o} switch is ignored
  because there is only \emph{one} output file specified.

  More generally, if the \texttt{-o} switch does \emph{not} have a \texttt{+} symbol at
  the beginning, then the \texttt{-o} switch will be ignored, and is turned it off.

  For example

  \begin{commandshell}
latexindent.pl myfile*.tex -o=+myfile
\end{commandshell}

  \emph{will} work fine because \emph{each} \texttt{.tex} file will output to
  \texttt{<basename>myfile.tex}

  Similarly,

  \begin{commandshell}
latexindent.pl myfile*.tex -o=++
\end{commandshell}

  \emph{will} work because the `existence check/incrementation' routine will be applied.

 \subsection{interaction with lines switch}
  This behaves as expected by attempting to operate on the line numbers specified for each file. See
  the examples in \cref{sec:line-switch}.

 \subsection{interaction with check switches}
  The exit codes for \texttt{latexindent.pl} are given in \vref{tab:exit-codes}.

  When operating on multiple files with the check switch active, as in

  \begin{commandshell}
latexindent.pl myfile*.tex --check
        \end{commandshell}

  then
  \begin{itemize}
   \item exit code 0 means that the text from \emph{none} of the files has been changed;
   \item exit code 1 means that the text from \emph{at least one} of the files been file
         changed.
  \end{itemize}

  The interaction with \texttt{checkv} switch is as in the check switch, but with verbose
  output.

 \subsection{when a file does not exist}
  What happens if one of the files can not be operated upon?
  \begin{itemize}
   \item if at least one of the files does not exist and \texttt{latexindent.pl} has been
         called to act upon multiple files, then the exit code is 3; note that
         \texttt{latexindent.pl} will try to operate on each file that it is called upon,
         and will not exit with a fatal message in this case;
   \item if at least one of the files can not be read and \texttt{latexindent.pl} has
         been called to act upon multiple files, then the exit code is 4; note that
         \texttt{latexindent.pl} will try to operate on each file that it is called upon,
         and will not exit with a fatal message in this case;
   \item if \texttt{latexindent.pl} has been told to operate on multiple files, and some
         do not exist and some cannot be read, then the exit code will be either 3 or 4,
         depending upon which it scenario it encountered most recently.
  \end{itemize}

 \section{latexindent-yaml-schema.json}

  \texttt{latexindent.pl}
  \announce{2022-01-02}{latexindent-yaml-schema.json} ships with
  \texttt{latexindent-yaml-schema.json}
  which might help you when constructing your YAML files.
  \index{json!schema for YAML files}

 \subsection{VSCode demonstration}
  To use \texttt{latexindent-yaml-schema.json} with \texttt{VSCode}, you can use the
  following steps: \index{VSCode} \index{json!VSCode}
  \begin{enumerate}
   \item download \texttt{latexindent-yaml-schema.json} from the \texttt{documentation}
         folder of \cite{latexindent-home}, save it in whichever directory you would
         like, noting it for reference;
   \item following the instructions from \cite{vscode-yaml-demo}, for example, you should
         install the VSCode YAML extension \cite{vscode-yaml-extentions};
   \item set up your \texttt{settings.json} file using the directory you saved the file
         by adapting \cref{lst:settings.json}; on my Ubuntu laptop this file lives at
         \texttt{/home/cmhughes/.config/Code/User/settings.json}.
  \end{enumerate}

  \begin{widepage}
   \cmhlistingsfromfile[style=yaml-LST]{demonstrations/settings.json}[yaml-TCB]{\texttt{settings.json}}{lst:settings.json}
  \end{widepage}

  Alternatively, if you would prefer not to download the json file, you might be able to
  use an adapted version of \cref{lst:settings-alt.json}.

  \begin{widepage}
   \cmhlistingsfromfile[style=yaml-LST]{demonstrations/settings-alt.json}[yaml-TCB]{\texttt{settings-alt.json}}{lst:settings-alt.json}
  \end{widepage}

  Finally, if your TeX distribution is up to date, then
  \texttt{latexindent-yaml-schema.json} \emph{should} be in the documentation folder of
  your installation, so an adapted version of \cref{lst:settings-alt1.json} may work.

  \begin{widepage}
   \cmhlistingsfromfile[style=yaml-LST]{demonstrations/settings-alt1.json}[yaml-TCB]{\texttt{settings-alt1.json}}{lst:settings-alt1.json}
  \end{widepage}

  If you have details of how to implement this schema in other editors, please feel
  encouraged to contribute to this documentation.

 \section{Using conda}\label{sec:app:conda}
  If you use conda you'll only need

  \begin{commandshell}
conda install latexindent.pl -c conda-forge
\end{commandshell}

  This will install the executable and all its dependencies (including perl) in the
  activate environment. You don't even have to worry about \texttt{defaultSettings.yaml}
  as it included too, you can thus skip \cref{sec:requiredmodules,sec:updating-path}.
  \index{conda}

  You can get a conda installation for example from \cite{conda} or from \cite{anacoda}.

 \subsection{Sample conda installation on Ubuntu}
  On Ubuntu I followed the 64-bit installation instructions at \cite{condainstallubuntu}
  and then I ran the following commands:

  \begin{commandshell}
conda create -n latexindent.pl
conda activate latexindent.pl
conda install latexindent.pl -c conda-forge
conda info --envs
conda list
conda run latexindent.pl -vv
\end{commandshell}

  I found the details given at \cite{condainstallhelp} to be helpful.

 \section{Using docker}\label{sec:app:docker}
  If you use docker you'll only need \announce{2022-06-12}{docker support}

  \begin{commandshell}
docker pull ghcr.io/cmhughes/latexindent.pl
\end{commandshell}

  This will download the image packed \texttt{latexindent}'s executable and its all
  dependencies. \index{docker} Thank you to \cite{eggplants} for contributing this
  feature; see also \cite{cmhughesio}. For reference, \emph{ghcr} stands for \emph{GitHub
  Container Repository}.

 \subsection{Sample docker installation on Ubuntu}
  To pull the image and show \texttt{latexindent}'s help on Ubuntu:

  \begin{cmhlistings}[style=tcblatex,language=Bash]{\texttt{docker-install.sh}}{lst:docker-install}
# setup docker if not already installed
if ! command -v docker &> /dev/null; then
  sudo apt install docker.io -y
  sudo groupadd docker
  sudo gpasswd -a "$USER" docker
  sudo systemctl restart docker
  newgrp docker
fi

# download image and execute
docker pull ghcr.io/cmhughes/latexindent.pl
docker run ghcr.io/cmhughes/latexindent.pl -h
\end{cmhlistings}

  Once I have run the above, on subsequent logins I run
  \begin{cmhlistings}[style=tcblatex,language=Bash]{\texttt{docker-install.sh}}{lst:docker-install-subsequent}
newgrp docker
docker run ghcr.io/cmhughes/latexindent.pl -h
\end{cmhlistings}

 \subsection{How to format on Docker}
  When you use \texttt{latexindent} with the docker image, you have to mount target \texttt{tex} file like this:

  \begin{commandshell}
docker run -v /path/to/local/myfile.tex:/myfile.tex ghcr.io/cmhughes/latexindent.pl -s -w myfile.tex
\end{commandshell}

 \section{pre-commit}

  Users of \texttt{.git} may be interested \announce{2022-01-21}{pre-commit for
  latexindent.pl} in exploring the \texttt{pre-commit} tool \cite{pre-commithome}, which
  is supported by \texttt{latexindent.pl}. Thank you to \cite{tdegeusprecommit} for
  contributing this feature, and to \cite{holzhausprecommit} for their contribution to
  it.

  To use the \texttt{pre-commit} tool, you will need to install \texttt{pre-commit};
  sample instructions for Ubuntu are given in \cref{sec:pre-commit-ubuntu}. Once
  installed, there are two ways to use \texttt{pre-commit}: using \texttt{CPAN} or using
  \texttt{conda}, detailed in \cref{sec:pre-commit-cpan} and \cref{sec:pre-commit-conda}
  respectively.

 \subsection{Sample pre-commit installation on Ubuntu}\label{sec:pre-commit-ubuntu}
  On Ubuntu I ran the following command:

  \begin{commandshell}
python3 -m pip install pre-commit
\end{commandshell}

  I then updated my path via .bashrc so that it includes the line in
  \cref{lst:bashrc-update}.
  \begin{cmhlistings}[style=tcblatex,language=Bash]{\texttt{.bashrc} update}{lst:bashrc-update}
...
export PATH=$PATH:/home/cmhughes/.local/bin
\end{cmhlistings}

 \subsection{pre-commit defaults}
  The default values that are employed by \texttt{pre-commit} are shown in \cref{lst:.pre-commit-yaml-default}.
  \index{pre-commit!default}

  \cmhlistingsfromfile[style=yaml-LST]{../.pre-commit-hooks.yaml}[yaml-TCB]{\texttt{.pre-commit-hooks.yaml} (default)}{lst:.pre-commit-yaml-default}

  In particular, the decision has deliberately been made (in collaboration with
  \cite{holzhausprecommit}) to have the default to employ the following switches:
  \texttt{overwriteIfDifferent}, \texttt{silent}, \texttt{local}; this is detailed in the
  lines that specify \texttt{args} in \cref{lst:.pre-commit-yaml-default}.

  \index{pre-commit!warning}
  \index{warning!pre-commit}
  \begin{warning}
   Users of \texttt{pre-commit} will, by default, have the \texttt{overwriteIfDifferent}
   switch employed. It is assumed that such users have version control in place, and are
   intending to overwrite their files.
  \end{warning}

 \subsection{pre-commit using CPAN}\label{sec:pre-commit-cpan}

  To use \texttt{latexindent.pl} with \texttt{pre-commit}, create the file
  \texttt{.pre-commit-config.yaml} given in \cref{lst:.pre-commit-config.yaml-cpan} in
  your git-repository. \index{cpan} \index{git} \index{pre-commit!cpan}

  \cmhlistingsfromfile[style=yaml-LST]{demonstrations/pre-commit-config-cpan.yaml}[yaml-TCB]{\texttt{.pre-commit-config.yaml} (cpan)}{lst:.pre-commit-config.yaml-cpan}
  Once created, you should then be able to run the following command:

  \begin{commandshell}
pre-commit run --all-files
\end{commandshell}

  A few notes about \cref{lst:.pre-commit-config.yaml-cpan}:
  \begin{itemize}
   \item the settings given in \cref{lst:.pre-commit-config.yaml-cpan} instruct
         \texttt{pre-commit} to use \texttt{CPAN} to get dependencies;
   \item this requires \texttt{pre-commit} and \texttt{perl} to be installed on your
         system;
   \item the \texttt{args} lists selected command-line options; the settings in
         \cref{lst:.pre-commit-config.yaml-cpan} are equivalent to calling

         \begin{commandshell}
latexindent.pl -s myfile.tex
\end{commandshell}

         for each \texttt{.tex} file in your repository;
   \item to instruct \texttt{latexindent.pl} to overwrite the files in your repository,
         then you can update \cref{lst:.pre-commit-config.yaml-cpan} so that
         \texttt{args: [-s, -w]}.
  \end{itemize}

  Naturally you can add options, or omit \texttt{-s} and \texttt{-w}, according to your
  preference.

 \subsection{pre-commit using conda}\label{sec:pre-commit-conda}

  You can also rely on \texttt{conda} (detailed in \cref{sec:app:conda}) instead of
  \texttt{CPAN} for all dependencies, including \texttt{latexindent.pl} itself.
  \index{conda} \index{git} \index{pre-commit!conda}

  \cmhlistingsfromfile[style=yaml-LST]{demonstrations/pre-commit-config-conda.yaml}[yaml-TCB]{\texttt{.pre-commit-config.yaml} (conda)}{lst:.pre-commit-config.yaml-conda}
  Once created, you should then be able to run the following command:

  \begin{commandshell}
pre-commit run --all-files
\end{commandshell}

  A few notes about \cref{lst:.pre-commit-config.yaml-cpan}:
  \begin{itemize}
   \item the settings given in \cref{lst:.pre-commit-config.yaml-conda} instruct
         \texttt{pre-commit} to use \texttt{conda} to get dependencies;
   \item this requires \texttt{pre-commit} and \texttt{conda} to be installed on your
         system;
   \item the \texttt{args} lists selected command-line options; the settings in
         \cref{lst:.pre-commit-config.yaml-cpan} are equivalent to calling

         \begin{commandshell}
conda run latexindent.pl -s myfile.tex
\end{commandshell}

         for each \texttt{.tex} file in your repository;
   \item to instruct \texttt{latexindent.pl} to overwrite the files in your repository,
         then you can update \cref{lst:.pre-commit-config.yaml-cpan} so that
         \texttt{args: [-s, -w]}.
  \end{itemize}

 \subsection{pre-commit using docker}\label{sec:pre-commit-docker}

  You can also rely on \texttt{docker} (detailed in \cref{sec:app:docker}) instead of
  \texttt{CPAN} for all dependencies, including \texttt{latexindent.pl} itself.
  \index{docker} \index{git} \index{pre-commit!docker}

  \cmhlistingsfromfile[style=yaml-LST]{demonstrations/pre-commit-config-docker.yaml}[yaml-TCB]{\texttt{.pre-commit-config.yaml} (docker)}{lst:.pre-commit-config.yaml-docker}
  Once created, you should then be able to run the following command:

  \begin{commandshell}
pre-commit run --all-files
\end{commandshell}

  A few notes about \cref{lst:.pre-commit-config.yaml-cpan}:
  \begin{itemize}
   \item the settings given in \cref{lst:.pre-commit-config.yaml-docker} instruct
         \texttt{pre-commit} to use \texttt{docker} to get dependencies;
   \item this requires \texttt{pre-commit} and \texttt{docker} to be installed on your
         system;
   \item the \texttt{args} lists selected command-line options; the settings in
         \cref{lst:.pre-commit-config.yaml-cpan} are equivalent to calling

         \begin{commandshell}
docker run -v /path/to/myfile.tex:/myfile.tex ghcr.io/cmhughes/latexindent.pl -s myfile.tex
\end{commandshell}

         for each \texttt{.tex} file in your repository;
   \item to instruct \texttt{latexindent.pl} to overwrite the files in your repository,
         then you can update \cref{lst:.pre-commit-config.yaml-cpan} so that
         \texttt{args: [-s, -w]}.
  \end{itemize}

 \subsection{pre-commit example using -l, -m switches}
  Let's consider a small example, with local \texttt{latexindent.pl} settings in
  \texttt{.latexindent.yaml}.

  \begin{example}
  We use the local settings given in \cref{lst:.latexindent.yaml}.
  \begin{cmhlistings}[style=tcblatex]{\texttt{.latexindent.yaml}}{lst:.latexindent.yaml}
onlyOneBackUp: 1

modifyLineBreaks:
 oneSentencePerLine:
   manipulateSentences: 1
\end{cmhlistings}

  and \texttt{.pre-commit-config.yaml} as in \cref{lst:.latexindent.yaml-switches}:

  \cmhlistingsfromfile[style=yaml-LST]{demonstrations/pre-commit-config-demo.yaml}[yaml-TCB]{\texttt{.pre-commit-config.yaml} (demo)}{lst:.latexindent.yaml-switches}
  Now running

  \begin{commandshell}
pre-commit run --all-files
\end{commandshell}

  is equivalent to running

  \begin{commandshell}
latexindent.pl -l -m -s -w myfile.tex
\end{commandshell}

  for each \texttt{.tex} file in your repository.

  A few notes about \cref{lst:.latexindent.yaml-switches}:
  \begin{itemize}
   \item the \texttt{-l} option was added to use the local \texttt{.latexindent.yaml}
         (where it was specified to only create one back-up file, as \texttt{git}
         typically takes care of this when you use \texttt{pre-commit});
   \item \texttt{-m} to modify line breaks; in addition to \texttt{-s} to suppress command-line
         output, and \texttt{-w} to format files in place.
  \end{itemize}
  \end{example}

 \section{indentconfig options}\label{app:indentconfig:options}
  This section describes the possible locations for the main configuration file,
  discussed in \cref{sec:indentconfig}. Thank you to \cite{nehctargl} for this
  contribution. \announce{2023-01-01}{indentconfig location options}

  The possible locations of \texttt{indentconfig.yaml} are read one after the other, and
  reading stops when a valid file is found in one of the paths.

  Before stating the list, we give summarise in \cref{tab:environment:summary}.
  \begin{table}[!htb]
   \centering
   \caption{indentconfig environment variable summaries}
   \label{tab:environment:summary}
   \begin{tabular}{lllll}
    \toprule
    environment variable & type              & Linux    & macOS    & Windows  \\
    \midrule
    LATEXINDENT\_CONFIG  & full path to file & \faCheck & \faCheck & \faCheck \\
    XDG\_CONFIG\_HOME    & directory path    & \faCheck & \faClose & \faClose \\
    LOCALAPPDATA         & directory path    & \faClose & \faClose & \faCheck \\
    \bottomrule
   \end{tabular}
  \end{table}

  The following list shows the checked options and is sorted by their respective
  priority. It uses capitalized and with a dollar symbol prefixed names (e.g.
  \texttt{\$LATEXINDENT\_CONFIG}) to symbolize environment variables. In addition to that
  the variable name \texttt{\$homeDir} is used to symbolize your home directory.

  \begin{enumerate}
   \item The value of the environment variable \texttt{\$LATEXINDENT\_CONFIG} is treated
         as highest priority source for the path to the configuration file.
   \item The next options are dependent on your operating system:
         \begin{itemize}
          \item Linux:
                \begin{enumerate}
                 \item The file at
                       \texttt{\$XDG\_CONFIG\_HOME/latexindent/indentconfig.yaml}
                 \item The file at
                       \texttt{\$homeDir/.config/latexindent/indentconfig.yaml}
                \end{enumerate}
          \item Windows:
                \begin{enumerate}
                 \item The file at
                       \texttt{\$LOCALAPPDATA\textbackslash{}latexindent\textbackslash{}indentconfig.yaml}
                 \item The file at
                       \texttt{\$homeDir\textbackslash{}AppData\textbackslash{}Local\textbackslash{}latexindent\textbackslash{}indentconfig.yaml}
                \end{enumerate}
          \item Mac:
                \begin{enumerate}
                 \item The file at
                       \texttt{\$homeDir/Library/Preferences/latexindent/indentconfig.yaml}
                \end{enumerate}
         \end{itemize}
   \item The file at \texttt{\$homeDir/indentconfig.yaml}
   \item The file at \texttt{\$homeDir/.indentconfig.yaml}
  \end{enumerate}
 \subsection{Why to change the configuration location}
  This is mostly a question about what you prefer, some like to put all their configuration files in their home directory (see \texttt{\$homeDir} above),
  whilst some like to sort their configuration. And if you don't care about it, you can just continue using the same defaults.
 \subsection{How to change the configuration location}
  This depends on your preferred location, if, for example, you would like to set a custom location, you would have to change the \texttt{\$LATEXINDENT\_CONFIG} environment variable.

  Although the following example only covers \texttt{\$LATEXINDENT\_CONFIG}, the same
  process can be applied to \texttt{\$XDG\_CONFIG\_HOME} or \texttt{\$LOCALAPPDATA}
  because both are environment variables. You just have to change the path to your chosen
  configuration directory (e.g. \texttt{\$homeDir/.config} or
  \texttt{\$homeDir\textbackslash{}AppData\textbackslash{}Local} on Linux or Windows
  respectively)
 \subsubsection{Linux}
  To change \texttt{\$LATEXINDENT\_CONFIG} on Linux you can run the following command in
  a terminal after changing the path:
  \begin{widepage}

   \begin{commandshell}
echo 'export LATEXINDENT_CONFIG="/home/cmh/latexindent-config.yaml"' >> ~/.profile
    \end{commandshell}

  \end{widepage}
  Context: This command adds the given line to your \texttt{.profile} file (which is
  commonly stored in \texttt{\$HOME/.profile}). All commands in this file a run after
  login, so the environment variable will be set after your next login.

  You can check the value of \texttt{\$LATEXINDENT\_CONFIG} by typing

  \begin{commandshell}
echo $LATEXINDENT_CONFIG 
/home/cmh/latexindent-config.yaml
    \end{commandshell}

  Linux users interested in \texttt{\$XDG\_CONFIG\_HOME} can explore variations of the
  following commands

  \begin{commandshell}
echo $XDG_CONFIG_HOME
echo ${XDG_CONFIG_HOME:=$HOME/.config}
echo $XDG_CONFIG_HOME
mkdir /home/cmh/.config/latexindent
touch /home/cmh/.config/latexindent/indentconfig.yaml
    \end{commandshell}

 \subsubsection{Windows}
  To change \texttt{\$LATEXINDENT\_CONFIG} on Windows you can run the following command
  in \texttt{powershell.exe} after changing the path:
  \begin{widepage}
   \begin{dosprompt}
[Environment]::SetEnvironmentVariable
    ("LATEXINDENT_CONFIG", "\your\config\path", "User")
    \end{dosprompt}
  \end{widepage}
  This sets the environment variable for every user session.
 \subsubsection{Mac}
  To change \texttt{\$LATEXINDENT\_CONFIG} on macOS you can run the following command in
  a terminal after changing the path:

  \begin{commandshell}
echo 'export LATEXINDENT_CONFIG="/your/config/path"' >> ~/.profile
    \end{commandshell}

  Context: This command adds the line to your \texttt{.profile} file (which is commonly
  stored in \texttt{\$HOME/.profile}). All commands in this file a run after login, so
  the environment variable will be set after your next login.

 \section{paths demonstration}\label{sec:appendix:paths}

  As detailed in \vref{subsec:indentconfig} \announce*{2024-04-28}{nested paths
  demonstration}, the \texttt{paths} field can be specified in any of your \texttt{YAML}
  files.

  We will use the file in \cref{lst:paths-demo} for demonstration in what follows.

  \cmhlistingsfromfile*{demonstrations/paths-demo.tex}{\texttt{paths-demo.tex}}{lst:paths-demo}

  \begin{example}
  Consider the settings given in \cref{lst:path1-yaml} and \cref{lst:path2-yaml}.

  \begin{cmhtcbraster}[raster column skip=.1\linewidth]
   \cmhlistingsfromfile*[style=yaml-LST]{demonstrations/path1.yaml}[yaml-TCB]{\texttt{path1.yaml}}{lst:path1-yaml}
   \cmhlistingsfromfile*[style=yaml-LST,showspaces=true]{demonstrations/path2.yaml}[yaml-TCB]{\texttt{path2.yaml}}{lst:path2-yaml}
  \end{cmhtcbraster}

  Upon calling

  \begin{commandshell}
latexindent.pl -l=path1.yaml paths-demo.tex
\end{commandshell}

  then we will receive the output given in \cref{lst:paths-demo-mod1}.

  \cmhlistingsfromfile*[showspaces=true]{demonstrations/paths-demo-mod1.tex}{\texttt{paths-demo-mod1.tex}}{lst:paths-demo-mod1}

  We note that the settings from \cref{lst:path2-yaml} have been called from
  \cref{lst:path1-yaml}.

  On inspection of \texttt{indent.log} from the above call, we see the details of this
  part of the process given in \cref{lst:path-test1:txt}.

  \cmhlistingsfromfile*{demonstrations/path-test1.txt}{\texttt{path-test1.txt}}{lst:path-test1:txt}
  \end{example}

  \begin{example}
  Consider the settings given in \cref{lst:path3-yaml} to \cref{lst:path5-yaml}.

  \begin{cmhtcbraster}[raster columns=3,
    raster left skip=-3.5cm,
    raster right skip=-2cm,
    raster column skip=.03\linewidth]
   \cmhlistingsfromfile*[style=yaml-LST]{demonstrations/path3.yaml}[yaml-TCB]{\texttt{path3.yaml}}{lst:path3-yaml}
   \cmhlistingsfromfile*[style=yaml-LST,showspaces=true]{demonstrations/path4.yaml}[yaml-TCB]{\texttt{path4.yaml}}{lst:path4-yaml}
   \cmhlistingsfromfile*[style=yaml-LST,showspaces=true]{demonstrations/path5.yaml}[yaml-TCB]{\texttt{path5.yaml}}{lst:path5-yaml}
  \end{cmhtcbraster}

  Upon calling

  \begin{commandshell}
latexindent.pl -l=path3.yaml paths-demo.tex
\end{commandshell}

  then we will receive the output given in \cref{lst:paths-demo-mod3}.

  \cmhlistingsfromfile*[showspaces=true]{demonstrations/paths-demo-mod3.tex}{\texttt{paths-demo-mod3.tex}}{lst:paths-demo-mod3}

  We see that \texttt{path3.yaml} calls \texttt{path4.yaml} which in turn calls
  \texttt{path5.yaml}.

  On inspection of \texttt{indent.log} from the above call, we see the details of this
  part of the process given in \cref{lst:path-test3:txt}.

  \cmhlistingsfromfile*{demonstrations/path-test3.txt}{\texttt{path-test3.txt}}{lst:path-test3:txt}
  \end{example}

 \section{logFilePreferences}\label{app:logfile-demo}
  \Vref{lst:logFilePreferences} describes the options for customising the information given
  to the log file, and we provide a few demonstrations here.

  \begin{example}
  Let's say that we start with the code given in \cref{lst:simple}, and the settings
  specified in \cref{lst:logfile-prefs1-yaml}.

  \begin{minipage}{.35\linewidth}
   \cmhlistingsfromfile{demonstrations/simple.tex}{\texttt{simple.tex}}{lst:simple}
  \end{minipage}
  \hfill
  \begin{minipage}{.6\linewidth}
   \cmhlistingsfromfile[style=yaml-LST]{demonstrations/logfile-prefs1.yaml}[yaml-TCB]{\texttt{logfile-prefs1.yaml}}{lst:logfile-prefs1-yaml}
  \end{minipage}

  If we run the following command (noting that \texttt{-t} is active)

  \begin{commandshell}
latexindent.pl -t -l=logfile-prefs1.yaml simple.tex
\end{commandshell}

  then on inspection of \texttt{indent.log} we will find the snippet given in
  \cref{lst:indentlog}.
  \begin{cmhlistings}[style=tcblatex,morekeywords={TRACE}]{\texttt{indent.log}}{lst:indentlog}
       +++++
TRACE: environment found: myenv
       No ancestors found for myenv
       Storing settings for myenvenvironments
       indentRulesGlobal specified (0) for environments, ...
       Using defaultIndent for myenv
       Putting linebreak after replacementText for myenv
       looking for COMMANDS and key = {value}
TRACE: Searching for commands with optional and/or mandatory arguments AND key = {value}
       looking for SPECIAL begin/end
TRACE: Searching myenv for special begin/end (see specialBeginEnd)
TRACE: Searching myenv for optional and mandatory arguments
       ... no arguments found
       -----
     \end{cmhlistings}
  Notice that the information given about \texttt{myenv} is `framed' using \texttt{+++++}
  and \lstinline!-----! respectively.
  \end{example}

 \section{Encoding indentconfig.yaml}\label{app:encoding}
  In relation to \vref{sec:indentconfig}, Windows users that encounter encoding issues
  with \texttt{indentconfig.yaml}, may wish to run the following command in either
  \texttt{cmd.exe} or \texttt{powershell.exe}:
  \begin{dosprompt}
chcp
    \end{dosprompt}
  They may receive the following result
  \begin{dosprompt}
Active code page: 936
    \end{dosprompt}
  and can then use the settings given in \cref{lst:indentconfig-encoding1} within their
  \texttt{indentconfig.yaml}, where 936 is the result of the \texttt{chcp} command.

  \cmhlistingsfromfile[style=yaml-LST]{demonstrations/encoding1.yaml}[yaml-TCB]{\texttt{encoding} demonstration for \texttt{indentconfig.yaml}}{lst:indentconfig-encoding1}

 \section{dos2unix linebreak adjustment}

 \yamltitle{dos2unixlinebreaks}*{integer}
  If you use \texttt{latexindent.pl} on a dos-based Windows file on Linux
  \announce{2021-06-19}{dos2unix linebreaks} then you may find that trailing horizontal
  space is not removed as you hope.

  In such a case, you may wish to try setting \texttt{dos2unixlinebreaks} to 1 and
  employing, for example, the following command.

  \begin{commandshell}
latexindent.pl -y="dos2unixlinebreaks:1" myfile.tex
\end{commandshell}

  See \cite{bersbersbers} for further dertails.

 \section{Differences from Version 2.2 to 3.0}\label{app:differences}
  There are a few (small) changes to the interface when comparing Version 2.2 to Version
  3.0. Explicitly, in previous versions you might have run, for example,
  \index{switches!-o demonstration}

  \begin{commandshell}
latexindent.pl -o myfile.tex outputfile.tex
\end{commandshell}

  whereas in Version 3.0 you would run any of the following, for example,
  \index{switches!-o demonstration}

  \begin{commandshell}
latexindent.pl -o=outputfile.tex myfile.tex
latexindent.pl -o outputfile.tex myfile.tex
latexindent.pl myfile.tex -o outputfile.tex
latexindent.pl myfile.tex -o=outputfile.tex
latexindent.pl myfile.tex -outputfile=outputfile.tex
latexindent.pl myfile.tex -outputfile outputfile.tex
\end{commandshell}

  noting that the \emph{output} file is given \emph{next to} the \texttt{-o} switch.

  The fields given in \cref{lst:obsoleteYaml} are \emph{obsolete} from Version 3.0
  onwards.
  \cmhlistingsfromfile[style=yaml-LST]{demonstrations/obsolete.yaml}[yaml-obsolete]{Obsolete YAML fields from Version 3.0}{lst:obsoleteYaml}

  There is a slight difference when specifying indentation after headings; specifically,
  we now write \texttt{indentAfterThisHeading} instead of \texttt{indent}. See
  \cref{lst:indentAfterThisHeadingOld,lst:indentAfterThisHeadingNew}

  \begin{minipage}{.45\textwidth}
   \cmhlistingsfromfile[style=yaml-LST]{demonstrations/indentAfterThisHeadingOld.yaml}[yaml-TCB]{\texttt{indentAfterThisHeading} in Version 2.2}{lst:indentAfterThisHeadingOld}
  \end{minipage}%
  \hfill
  \begin{minipage}{.45\textwidth}
   \cmhlistingsfromfile[style=yaml-LST]{demonstrations/indentAfterThisHeadingNew.yaml}[yaml-TCB]{\texttt{indentAfterThisHeading} in Version 3.0}{lst:indentAfterThisHeadingNew}
  \end{minipage}%

  To specify \texttt{noAdditionalIndent} for display-math environments in Version 2.2,
  you would write YAML as in \cref{lst:noAdditionalIndentOld}; as of Version 3.0, you
  would write YAML as in \cref{lst:indentAfterThisHeadingNew1} or, if you're using
  \texttt{-m} switch, \cref{lst:indentAfterThisHeadingNew2}.
  \index{specialBeginEnd!update to displaymath V3.0}

  \begin{minipage}{.45\textwidth}
   \cmhlistingsfromfile[style=yaml-LST]{demonstrations/noAddtionalIndentOld.yaml}[yaml-TCB]{\texttt{noAdditionalIndent} in Version 2.2}{lst:noAdditionalIndentOld}
  \end{minipage}%
  \hfill
  \begin{minipage}{.45\textwidth}
   \cmhlistingsfromfile[style=yaml-LST]{demonstrations/noAddtionalIndentNew.yaml}[yaml-TCB]{\texttt{noAdditionalIndent} for \texttt{displayMath} in Version 3.0}{lst:indentAfterThisHeadingNew1}

   \cmhlistingsfromfile[style=yaml-LST]{demonstrations/noAddtionalIndentNew1.yaml}[yaml-TCB]{\texttt{noAdditionalIndent} for \texttt{displayMath} in Version 3.0}{lst:indentAfterThisHeadingNew2}
  \end{minipage}%

  \mbox{}\hfill
  \begin{minipage}{.25\textwidth}
   \hrule

   \hfill\itshape End\\\mbox{}\hfill\mbox{}\rlap{\hfill\includegraphics{logo}}

  \end{minipage}
