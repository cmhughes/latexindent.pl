% arara: pdflatex: { files: [latexindent]}
\subsection{Summary of text wrapping}
	I consider%
	\announce{2021-07-31}{text wrap quick start}
	the most useful starting point for text wrapping to be given in
	\cref{subsec:textwrapping-quick-start} and \cref{subsubsec:text-wrap-remove-para-bfccb}.

	Starting from \cref{lst:textwrap-qs-yaml}, it is likely that you will have to experiment
	with making adjustments (such as that given in \cref{lst:textwrap14-yaml}) depending on
	your preference.

	It is important to note the following:
	\index{verbatim!within summary of text wrapping}
	\begin{itemize}
		\item verbatim code blocks of all types will \emph{not} be affected by the text wrapping
		      routine. See the demonstration in \vref{lst:textwrap2-mod1}, together with environments:
		      \vref{lst:verbatimEnvironments}, commands: \vref{lst:verbatimCommands},
		      \texttt{noIndentBlock}: \cref{lst:noIndentBlock}, \texttt{specialBeginEnd}:
		      \vref{lst:special3-mod1};
		\item comments will \emph{not} be affected by the text wrapping routine (see
		      \vref{lst:textwrap3-mod1});
		\item it is possible to wrap text on a per-code-block and a per-name basis;
		      \announce{2018-08-13}*{updates to textWrapOptions}
		\item indentation is performed \emph{after} the text wrapping routine; as such, indented code
		      will likely exceed any maximum value set in the \texttt{columns} field.
	\end{itemize}
