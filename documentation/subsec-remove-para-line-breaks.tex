% arara: pdflatex: { files: [latexindent]}
\subsection{removeParagraphLineBreaks: modifying line breaks for paragraphs}\label{subsec:removeparagraphlinebreaks} When the \texttt{-m} switch is active
	\texttt{latexindent.pl} has the ability to remove line
	breaks%
	\announce{2017-05-27}{removeParagraphLineBreaks}
	from within paragraphs; the behaviour is controlled by the
	\texttt{removeParagraphLineBreaks} field, detailed in
	\cref{lst:removeParagraphLineBreaks}. Thank you to \cite{jowens} for shaping and
	assisting with the testing of this feature.
\yamltitle{removeParagraphLineBreaks}*{fields}
	This feature is considered complimentary to the \texttt{oneSentencePerLine} feature
	described in \vref{sec:onesentenceperline}.
	\index{specialBeginEnd!removeParagraphLineBreaks}

	\cmhlistingsfromfile[style=removeParagraphLineBreaks]{../defaultSettings.yaml}[MLB-TCB,width=.85\linewidth,before=\centering]{\texttt{removeParagraphLineBreaks}}{lst:removeParagraphLineBreaks}

	This routine can be turned on \emph{globally} for \emph{every} code block type known to
	\texttt{latexindent.pl} (see \vref{tab:code-blocks}) by using the \texttt{all} switch; by
	default, this switch is \emph{off}. Assuming that the \texttt{all} switch is off, then
	the routine can be controlled on a per-code-block-type basis, and within that, on a
	per-name basis. We will consider examples of each of these in turn, but before we do,
	let's specify what \texttt{latexindent.pl} considers as a paragraph:
	\begin{itemize}
		\item it must begin on its own line with either an alphabetic or numeric character, and not
		      with any of the code-block types detailed in \vref{tab:code-blocks};
		\item it can include line breaks, but finishes when it meets either a blank line, a
		      \lstinline!\par! command, or any of the user-specified settings in the
		      \texttt{paragraphsStopAt} field, detailed in \vref{lst:paragraphsStopAt}.
	\end{itemize}

	Let's start with the \texttt{.tex} file in \cref{lst:shortlines}, together with the YAML
	settings in \cref{lst:remove-para1-yaml}.

	\begin{cmhtcbraster}
		\cmhlistingsfromfile[showspaces=true]{demonstrations/shortlines.tex}{\texttt{shortlines.tex}}{lst:shortlines}
		\cmhlistingsfromfile{demonstrations/remove-para1.yaml}[MLB-TCB]{\texttt{remove-para1.yaml}}{lst:remove-para1-yaml}
	\end{cmhtcbraster}

	Upon running the command
	\index{switches!-l demonstration}
	\index{switches!-m demonstration}
	\index{switches!-o demonstration}
	\begin{commandshell}
latexindent.pl -m shortlines.tex -o shortlines1.tex -l remove-para1.yaml
\end{commandshell}
	then we obtain the output given in \cref{lst:shortlines1}.

	\cmhlistingsfromfile[showspaces=true]{demonstrations/shortlines1.tex}{\texttt{shortlines1.tex}}{lst:shortlines1}

	Keen readers may notice that some trailing white space must be present in the file in
	\cref{lst:shortlines} which has crept in to the output in \cref{lst:shortlines1}. This
	can be fixed using the YAML file in \vref{lst:removeTWS-before} and running, for example,
	\index{switches!-l demonstration}
	\index{switches!-m demonstration}
	\index{switches!-o demonstration}
	\begin{commandshell}
latexindent.pl -m shortlines.tex -o shortlines1-tws.tex -l remove-para1.yaml,removeTWS-before.yaml  
\end{commandshell}
	in which case the output is as in \cref{lst:shortlines1-tws}; notice that the double
	spaces present in \cref{lst:shortlines1} have been addressed.

	\cmhlistingsfromfile[showspaces=true]{demonstrations/shortlines1-tws.tex}{\texttt{shortlines1-tws.tex}}{lst:shortlines1-tws}

	Keeping with the settings in \cref{lst:remove-para1-yaml}, we note that the \texttt{all}
	switch applies to \emph{all} code block types. So, for example, let's consider the files
	in \cref{lst:shortlines-mand,lst:shortlines-opt}

	\begin{cmhtcbraster}
		\cmhlistingsfromfile{demonstrations/shortlines-mand.tex}{\texttt{shortlines-mand.tex}}{lst:shortlines-mand}
		\cmhlistingsfromfile{demonstrations/shortlines-opt.tex}{\texttt{shortlines-opt.tex}}{lst:shortlines-opt}
	\end{cmhtcbraster}

	Upon running the commands
	\index{switches!-l demonstration}
	\index{switches!-m demonstration}
	\index{switches!-o demonstration}
	\begin{widepage}
		\begin{commandshell}
latexindent.pl -m shortlines-mand.tex -o shortlines-mand1.tex -l remove-para1.yaml
latexindent.pl -m shortlines-opt.tex -o shortlines-opt1.tex -l remove-para1.yaml
\end{commandshell}
	\end{widepage}

	then we obtain the respective output given in
	\cref{lst:shortlines-mand1,lst:shortlines-opt1}.

	\cmhlistingsfromfile{demonstrations/shortlines-mand1.tex}{\texttt{shortlines-mand1.tex}}{lst:shortlines-mand1}
	\cmhlistingsfromfile{demonstrations/shortlines-opt1.tex}{\texttt{shortlines-opt1.tex}}{lst:shortlines-opt1}

	Assuming that we turn \emph{off} the \texttt{all} switch (by setting it to \texttt{0}),
	then we can control the behaviour of \texttt{removeParagraphLineBreaks} either on a
	per-code-block-type basis, or on a per-name basis.

	For example, let's use the code in \cref{lst:shortlines-envs}, and consider the settings
	in \cref{lst:remove-para2-yaml,lst:remove-para3-yaml}; note that in
	\cref{lst:remove-para2-yaml} we specify that \emph{every} environment should receive
	treatment from the routine, while in \cref{lst:remove-para3-yaml} we specify that
	\emph{only} the \texttt{one} environment should receive the treatment.

	\begin{minipage}{.45\linewidth}
		\cmhlistingsfromfile{demonstrations/shortlines-envs.tex}{\texttt{shortlines-envs.tex}}{lst:shortlines-envs}
	\end{minipage}
	\hfill
	\begin{minipage}{.49\linewidth}
		\cmhlistingsfromfile{demonstrations/remove-para2.yaml}[MLB-TCB]{\texttt{remove-para2.yaml}}{lst:remove-para2-yaml}
		\cmhlistingsfromfile{demonstrations/remove-para3.yaml}[MLB-TCB]{\texttt{remove-para3.yaml}}{lst:remove-para3-yaml}
	\end{minipage}

	Upon running the commands
	\index{switches!-l demonstration}
	\index{switches!-m demonstration}
	\index{switches!-o demonstration}
	\begin{widepage}
		\begin{commandshell}
latexindent.pl -m shortlines-envs.tex -o shortlines-envs2.tex -l remove-para2.yaml
latexindent.pl -m shortlines-envs.tex -o shortlines-envs3.tex -l remove-para3.yaml
\end{commandshell}
	\end{widepage}
	then we obtain the respective output given in
	\cref{lst:shortlines-envs2,lst:shortlines-envs3}.

	\cmhlistingsfromfile{demonstrations/shortlines-envs2.tex}{\texttt{shortlines-envs2.tex}}{lst:shortlines-envs2}
	\cmhlistingsfromfile{demonstrations/shortlines-envs3.tex}{\texttt{shortlines-envs3.tex}}{lst:shortlines-envs3}

	The remaining code-block types can be customised in analogous ways, although note that
	\texttt{commands}, \texttt{keyEqualsValuesBracesBrackets},
	\texttt{namedGroupingBracesBrackets}, \texttt{UnNamedGroupingBracesBrackets} are
	controlled by the \texttt{optionalArguments} and the \texttt{mandatoryArguments}.

	The only special case is the \texttt{mainDocument} field; this is designed for
	`chapter'-type files that may contain paragraphs that are not within any other
	code-blocks. For example, consider the file in \cref{lst:shortlines-md}, with the YAML
	settings in \cref{lst:remove-para4-yaml}.

	Note:
	\announce{2021-09-16}*{removeParagraphLineBreaks: masterDocument now mainDocument}
	\texttt{mainDocument} replaces \texttt{masterDocument} which was used in previous verions
	of \texttt{latexindent.pl}. The field \texttt{masterDocument} is still supported, but it
	is anticipated to be removed in a future version, so I recommend using
	\texttt{mainDocument} instead.

	\begin{cmhtcbraster}
		\cmhlistingsfromfile{demonstrations/shortlines-md.tex}{\texttt{shortlines-md.tex}}{lst:shortlines-md}
		\cmhlistingsfromfile{demonstrations/remove-para4.yaml}[MLB-TCB]{\texttt{remove-para4.yaml}}{lst:remove-para4-yaml}
	\end{cmhtcbraster}

	Upon running the following command
	\index{switches!-l demonstration}
	\index{switches!-m demonstration}
	\index{switches!-o demonstration}
	\begin{widepage}
		\begin{commandshell}
latexindent.pl -m shortlines-md.tex -o shortlines-md4.tex -l remove-para4.yaml
\end{commandshell}
	\end{widepage}
	then we obtain the output in \cref{lst:shortlines-md4}.
	\cmhlistingsfromfile{demonstrations/shortlines-md4.tex}{\texttt{shortlines-md4.tex}}{lst:shortlines-md4}

	Note%
	\announce{2018-08-13}*{updates to all in removeParagraphLineBreaks} that the \texttt{all} field can take the same exceptions
	detailed in \crefrange{lst:textwrap6-yaml}{lst:textwrap8-yaml}.

\yamltitle{paragraphsStopAt}*{fields}
	The paragraph line break routine considers blank lines and the
	\lstinline|\par| command to be the end of a paragraph;
	\announce{2017-05-27}{paragraphsStopAt} you can fine tune the behaviour of the routine
	further by using the \texttt{paragraphsStopAt} fields, shown in
	\cref{lst:paragraphsStopAt}.
	\index{specialBeginEnd!paragraphsStopAt}
	\index{verbatim!in relation to paragraphsStopAt}

	\cmhlistingsfromfile[style=paragraphsStopAt]{../defaultSettings.yaml}[MLB-TCB,width=.85\linewidth,before=\centering]{\texttt{paragraphsStopAt}}{lst:paragraphsStopAt}

	The fields specified in \texttt{paragraphsStopAt} tell \texttt{latexindent.pl} to stop
	the current paragraph when it reaches a line that \emph{begins} with any of the
	code-block types specified as \texttt{1} in \cref{lst:paragraphsStopAt}. By default,
	you'll see that the paragraph line break routine will stop when it reaches an environment
	or verbatim code block at the beginning of a line. It is \emph{not} possible to specify
	these fields on a per-name basis.

	Let's use the \texttt{.tex} file in \cref{lst:sl-stop}; we will, in turn, consider the
	settings in \cref{lst:stop-command-yaml,lst:stop-comment-yaml}.

	\begin{minipage}{.45\linewidth}
		\cmhlistingsfromfile{demonstrations/sl-stop.tex}{\texttt{sl-stop.tex}}{lst:sl-stop}
	\end{minipage}
	\hfill
	\begin{minipage}{.49\linewidth}
		\cmhlistingsfromfile{demonstrations/stop-command.yaml}[MLB-TCB]{\texttt{stop-command.yaml}}{lst:stop-command-yaml}

		\cmhlistingsfromfile{demonstrations/stop-comment.yaml}[MLB-TCB]{\texttt{stop-comment.yaml}}{lst:stop-comment-yaml}
	\end{minipage}

	Upon using the settings from \vref{lst:remove-para4-yaml} and running the commands
	\index{switches!-l demonstration}
	\index{switches!-m demonstration}
	\index{switches!-o demonstration}
	\begin{widepage}
		\begin{commandshell}
latexindent.pl -m sl-stop.tex -o sl-stop4.tex -l remove-para4.yaml
latexindent.pl -m sl-stop.tex -o sl-stop4-command.tex -l=remove-para4.yaml,stop-command.yaml
latexindent.pl -m sl-stop.tex -o sl-stop4-comment.tex -l=remove-para4.yaml,stop-comment.yaml
\end{commandshell}
	\end{widepage}
	we obtain the respective outputs in \crefrange{lst:sl-stop4}{lst:sl-stop4-comment};
	notice in particular that:
	\begin{itemize}
		\item in \cref{lst:sl-stop4} the paragraph line break routine has included commands and
		      comments;
		\item in \cref{lst:sl-stop4-command} the paragraph line break routine has \emph{stopped} at the
		      \texttt{emph} command, because in \cref{lst:stop-command-yaml} we have specified
		      \texttt{commands} to be \texttt{1}, and \texttt{emph} is at the beginning of a line;
		\item in \cref{lst:sl-stop4-comment} the paragraph line break routine has \emph{stopped} at the
		      comments, because in \cref{lst:stop-comment-yaml} we have specified \texttt{comments} to
		      be \texttt{1}, and the comment is at the beginning of a line.
	\end{itemize}
	In all outputs in \crefrange{lst:sl-stop4}{lst:sl-stop4-comment} we notice that the
	paragraph line break routine has stopped at \lstinline!\begin{myenv}! because, by
	default, \texttt{environments} is set to \texttt{1} in \vref{lst:paragraphsStopAt}.

	\cmhlistingsfromfile{demonstrations/sl-stop4.tex}{\texttt{sl-stop4.tex}}{lst:sl-stop4}
	\cmhlistingsfromfile{demonstrations/sl-stop4-command.tex}{\texttt{sl-stop4-command.tex}}{lst:sl-stop4-command}
	\cmhlistingsfromfile{demonstrations/sl-stop4-comment.tex}{\texttt{sl-stop4-comment.tex}}{lst:sl-stop4-comment}

