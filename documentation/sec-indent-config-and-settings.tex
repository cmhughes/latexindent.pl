% arara: pdflatex: { files: [latexindent]}
\section{indentconfig.yaml, local settings and the -y switch }\label{sec:indentconfig}
 The behaviour of \texttt{latexindent.pl} is controlled from the settings specified in
 any of the YAML files that you tell it to load. By default, \texttt{latexindent.pl} will
 only load \texttt{defaultSettings.yaml}, but there are a few ways that you can tell it
 to load your own settings files.

 We focus our discussion on \texttt{indentconfig.yaml}, but there are other options which
 are detailed in \cref{app:indentconfig:options}. \announce*{2023-01-01}{indentconfig
 location options}

\subsection{indentconfig.yaml and .indentconfig.yaml}
 \texttt{latexindent.pl} will always check your home directory for
 \texttt{indentconfig.yaml}
 and \texttt{.indentconfig.yaml} (unless it is called with the \texttt{-d} switch), which
 is a plain text file you can create that contains the \emph{absolute} paths for any
 settings files that you wish \texttt{latexindent.pl} to load. There is no difference
 between \texttt{indentconfig.yaml} and \texttt{.indentconfig.yaml}, other than the fact
 that \texttt{.indentconfig.yaml} is a `hidden' file; thank you to
 \cite{jacobo-diaz-hidden-config} for providing this feature. In what follows, we will use
 \texttt{indentconfig.yaml}, but it is understood that this could equally represent
 \texttt{.indentconfig.yaml}. If you have both files in existence then
 \texttt{indentconfig.yaml} takes priority.

 For Mac and Linux users, their home directory is \texttt{~/username} while Windows
 (Vista onwards) is \lstinline!C:\Users\username!\footnote{If you're not sure where to
 put \texttt{indentconfig.yaml}, don't worry \texttt{latexindent.pl} will tell you in the
 log file exactly where to put it assuming it doesn't exist already.}
 \Cref{lst:indentconfig} shows a sample \texttt{indentconfig.yaml} file.

 \cmhlistingsfromfile[style=yaml-LST]{demonstrations/indent-config.yaml}[yaml-TCB]{\texttt{indentconfig.yaml} (sample)}{lst:indentconfig}

 Note that the \texttt{.yaml} files you specify in \texttt{indentconfig.yaml} will be
 loaded in the order in which you write them. Each file doesn't have to have every switch
 from \texttt{defaultSettings.yaml}; in fact, I recommend that you only keep the switches
 that you want to \emph{change} in these settings files.

 To get started with your own settings file, you might like to save a copy of
 \texttt{defaultSettings.yaml} in another directory and call it, for example,
 \texttt{mysettings.yaml}. Once you have added the path to \texttt{indentconfig.yaml} you
 can change the switches and add more code-block names to it as you see fit -- have a
 look at \cref{lst:mysettings} for an example that uses four tabs for the default indent,
 adds the \texttt{tabbing} environment/command to the list of environments that contains
 alignment delimiters; you might also like to refer to the many YAML files detailed
 throughout the rest of this documentation. \index{indentation!defaultIndent using YAML
 file}

 \cmhlistingsfromfile[style=yaml-LST]{demonstrations/tabbing.yaml}[yaml-TCB]{\texttt{mysettings.yaml} (example)}{lst:mysettings}

 You can make sure that your settings are loaded by checking \texttt{indent.log} for
 details -- if you have specified a path that \texttt{latexindent.pl} doesn't recognise
 then you'll get a warning, otherwise you'll get confirmation that
 \texttt{latexindent.pl} has read your settings file \footnote{Windows users may find
 that they have to end \texttt{.yaml} files with a blank line}. \index{warning!editing
 YAML files}

 \begin{warning}
  When editing \texttt{.yaml} files it is \emph{extremely} important to remember how
  sensitive they are to spaces. I highly recommend copying and pasting from
  \texttt{defaultSettings.yaml} when you create your first
  \texttt{whatevernameyoulike.yaml} file.

  If \texttt{latexindent.pl} can not read your \texttt{.yaml} file it will tell you so in
  \texttt{indent.log}.
 \end{warning}

 If you find that \announce{2021-06-19}{encoding option for indentconfig.yaml}
 \texttt{latexindent.pl} does not read your YAML file, then it might be as a result of
 the default commandline encoding not being UTF-8; normally this will only occur for
 Windows users. In this case, you might like to explore the \texttt{encoding} option for
 \texttt{indentconfig.yaml} as demonstrated in \cref{lst:indentconfig-encoding}.%

 \cmhlistingsfromfile[style=yaml-LST]{demonstrations/encoding.yaml}[yaml-TCB]{The \texttt{encoding} option for \texttt{indentconfig.yaml}}{lst:indentconfig-encoding}

 Thank you to \cite{qiancy98} for this contribution; please see \vref{app:encoding} and
 details within \cite{encoding} for further information.

\subsection{localSettings.yaml and friends}\label{sec:localsettings}
 The \texttt{-l} switch tells \texttt{latexindent.pl} to look for
 \texttt{localSettings.yaml} and/or friends in the \emph{same directory} as
 \texttt{myfile.tex}. For%
 \announce{2021-03-14}*{-l
 switch: localSettings and friends} example, if you use the following command
 \index{switches!-l demonstration}

 \begin{commandshell}
latexindent.pl -l myfile.tex
\end{commandshell}

 then \texttt{latexindent.pl} will search for and then, assuming they exist, load each of
 the following files in the following order:
 \begin{enumerate}
  \item localSettings.yaml
  \item latexindent.yaml
  \item .localSettings.yaml
  \item .latexindent.yaml
 \end{enumerate}
 These files will be assumed to be in the same directory as \texttt{myfile.tex}, or
 otherwise in the current working directory. You do not need to have all of the above
 files, usually just one will be sufficient. In what follows, whenever we refer to
 \texttt{localSettings.yaml} it is assumed that it can mean any of the four named options
 listed above.

 If you'd prefer to name your \texttt{localSettings.yaml} file something different, (say,
 \texttt{mysettings.yaml} as in \cref{lst:mysettings}) then you can call
 \texttt{latexindent.pl} using, for example,

 \begin{commandshell}
latexindent.pl -l=mysettings.yaml myfile.tex
\end{commandshell}

 Any settings file(s) specified using the \texttt{-l} switch will be read \emph{after}
 \texttt{defaultSettings.yaml} and, assuming they exist, any user setting files specified
 in \texttt{indentconfig.yaml}.

 Your settings file can contain any switches that you'd like to change; a sample is shown
 in \cref{lst:localSettings}, and you'll find plenty of further examples throughout this
 manual. \index{verbatim!verbatimEnvironments demonstration (-l switch)}

 \begin{yaml}{\texttt{localSettings.yaml} (example)}{lst:localSettings}
#  verbatim environments - environments specified
#  here will not be changed at all!
verbatimEnvironments:
    cmhenvironment: 0
    myenv: 1
\end{yaml}

 You can make sure that your settings file has been loaded by checking
 \texttt{indent.log} for details; if it can not be read then you receive a warning,
 otherwise you'll get confirmation that \texttt{latexindent.pl} has read your settings
 file.

\subsection{The -y|yaml switch}\label{sec:yamlswitch}
 You%
 \announce{2017-08-21}{demonstration of the -y switch}
 may use the \texttt{-y} switch to load your settings;  for example, if you wished to
 specify the settings from \cref{lst:localSettings} using the \texttt{-y} switch, then you
 could use the following command:
 \index{verbatim!verbatimEnvironments demonstration (-y switch)}

 \begin{commandshell}
latexindent.pl -y="verbatimEnvironments:cmhenvironment:0;myenv:1" myfile.tex
\end{commandshell}

 Note the use of \texttt{;} to specify another field within
 \texttt{verbatimEnvironments}. This is shorthand, and equivalent, to using the following
 command: \index{switches!-y demonstration}

 \begin{commandshell}
latexindent.pl -y="verbatimEnvironments:cmhenvironment:0,verbatimEnvironments:myenv:1" myfile.tex
\end{commandshell}

 You may, of course, specify settings using the \texttt{-y} switch as well as, for
 example, settings loaded using the \texttt{-l} switch; for example, \index{switches!-l
 demonstration} \index{switches!-y demonstration}

 \begin{commandshell}
latexindent.pl -l=mysettings.yaml -y="verbatimEnvironments:cmhenvironment:0;myenv:1" myfile.tex
\end{commandshell}

 Any settings specified using the \texttt{-y} switch will be loaded \emph{after} any
 specified using \texttt{indentconfig.yaml} and the \texttt{-l} switch.

 If you wish to specify any regex-based settings using the \texttt{-y} switch,
 \index{regular expressions!using -y switch} it is important not to use quotes
 surrounding the regex; for example, with reference to the `one sentence per line'
 feature (\vref{sec:onesentenceperline}) and the listings within
 \vref{lst:sentencesEndWith}, the following settings give the option to have sentences
 end with a semicolon \index{switches!-y demonstration}

 \begin{commandshell}
latexindent.pl -m --yaml='modifyLineBreaks:oneSentencePerLine:sentencesEndWith:other:\;'
\end{commandshell}

\subsection{Settings load order}\label{sec:loadorder}
 \texttt{latexindent.pl} loads the settings files in the following order:
 \index{switches!-l in relation to other settings}
 \begin{enumerate}
  \item \texttt{defaultSettings.yaml} is always loaded, and can not be renamed;
  \item \texttt{anyUserSettings.yaml} and any other arbitrarily-named files specified in
        \texttt{indentconfig.yaml};
  \item \texttt{localSettings.yaml} but only if found in the same directory as
        \texttt{myfile.tex}
        and called with \texttt{-l} switch; this file can be renamed, provided that the call to
        \texttt{latexindent.pl} is adjusted accordingly (see \cref{sec:localsettings}). You may
        specify both relative and absolute%
        \announce{2017-08-21}*{-l absolute paths} paths to other YAML files using the \texttt{-l}
        switch, separating multiple files using commas;
  \item any settings \announce{2017-08-21}{-y switch load order} specified in the
        \texttt{-y} switch.%
 \end{enumerate}
 A visual representation of this is given in \cref{fig:loadorder}.

 \begin{figure}[!htb]
  \centering
  \input{figure-schematic}
  \caption{Schematic of the load order described in \cref{sec:loadorder}; solid lines represent
  mandatory files, dotted lines represent optional files. \texttt{indentconfig.yaml} can
  contain as many files as you like. The files will be loaded in order; if you specify
  settings for the same field in more than one file, the most recent takes priority. }
  \label{fig:loadorder}
 \end{figure}
