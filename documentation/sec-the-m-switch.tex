% arara: pdflatex: { files: [latexindent]}
\renewcommand{\imagetouse}{logo}
\section{The -m (modifylinebreaks) switch}\label{sec:modifylinebreaks}
 \fancyhead[R]{\bfseries\thepage%
	 \tikz[remember picture,overlay] {
		 \node at (1,0){\includegraphics{logo}};
	 }}
 All features described in this section will only be relevant if the \texttt{-m} switch is
 used.

 \startcontents[the-m-switch]
 \printcontents[the-m-switch]{}{0}{}

\yamltitle{modifylinebreaks}*{fields}
	\makebox[0pt][r]{%
		\raisebox{-\totalheight}[0pt][0pt]{%
			\tikz\node[opacity=1] at (0,0)
			{\includegraphics[width=4cm]{logo}};}}%	
	As of Version 3.0, \texttt{latexindent.pl} has the \texttt{-m} switch, which permits
	\texttt{latexindent.pl} to modify line breaks, according to the specifications in the
	\texttt{modifyLineBreaks} field. \emph{The settings in this field will only be considered
		if the \texttt{-m} switch has been used}. A snippet of the default settings of this field
	is shown in \cref{lst:modifylinebreaks}.

	\cmhlistingsfromfile[style=modifylinebreaks]{../defaultSettings.yaml}[MLB-TCB,width=.85\linewidth,before=\centering]{\texttt{modifyLineBreaks}}{lst:modifylinebreaks}

	Having read the previous paragraph, it should sound reasonable that, if you call
	\texttt{latexindent.pl} using the \texttt{-m} switch, then you give it permission to
	modify line breaks in your file, but let's be clear:
	\index{warning!the m switch}

	\begin{warning}
		If you call \texttt{latexindent.pl} with the \texttt{-m} switch, then you are giving it
		permission to modify line breaks. By default, the only thing that will happen is that
		multiple blank lines will be condensed into one blank line; many other settings are
		possible, discussed next.
	\end{warning}

\yamltitle{preserveBlankLines}{0|1}
	This field is directly related to \emph{poly-switches}, discussed in
	\cref{sec:poly-switches}. By default, it is set to \texttt{1}, which means that blank
	lines will be \emph{protected} from removal; however, regardless of this setting,
	multiple blank lines can be condensed if \texttt{condenseMultipleBlankLinesInto} is
	greater than \texttt{0}, discussed next.

\yamltitle{condenseMultipleBlankLinesInto}*{positive integer}
	Assuming that this switch takes an integer value greater than \texttt{0},
	\texttt{latexindent.pl} will condense multiple blank lines into the number of blank lines
	illustrated by this switch. As an example, \cref{lst:mlb-bl} shows a sample file with
	blank lines; upon running
	\index{switches!-m demonstration}
	\begin{commandshell}
latexindent.pl myfile.tex -m -o=+-mod1 
\end{commandshell}
	the output is shown in \cref{lst:mlb-bl-out}; note that the multiple blank lines have
	been condensed into one blank line, and note also that we have used the \texttt{-m}
	switch!

	\begin{cmhtcbraster}
		\cmhlistingsfromfile{demonstrations/mlb1.tex}{\texttt{mlb1.tex}}{lst:mlb-bl}
		\cmhlistingsfromfile{demonstrations/mlb1-out.tex}{\texttt{mlb1-mod1.tex}}{lst:mlb-bl-out}
	\end{cmhtcbraster}
