% arara: pdflatex: {shell: yes, files: [latexindent]}
% the -m switch
% the -m switch
% the -m switch
\fancyhead[R]{\bfseries\thepage%
\tikz[remember picture,overlay] {
\node at (1,0){\includegraphics{logo}};
}}
\section{The \texttt{-m} (\texttt{modifylinebreaks}) switch}\label{sec:modifylinebreaks}
All features described in this section will only be relevant if the \texttt{-m} switch
is used.

\yamltitle{modifylinebreaks}*{fields}
\begin{wrapfigure}[7]{r}[0pt]{8cm}
\begin{yaml}[numbers=none]{\texttt{modifyLineBreaks}}[width=.8\linewidth,before=\centering,MLB-TCB]{lst:modifylinebreaks}
modifyLineBreaks:
    preserveBlankLines: 1
    condenseMultipleBlankLinesInto: 1
   ...
	\end{yaml}
  \end{wrapfigure}
\makebox[0pt][r]{%
	\raisebox{-\totalheight}[0pt][0pt]{%
		\tikz\node[opacity=1] at (0,0) {\includegraphics[width=4cm]{logo}};}}%	
One of the most exciting features of Version 3.0 is the \texttt{-m} switch, which 
permits \texttt{latexindent.pl} to modify line breaks, according to the 
specifications in the \texttt{modifyLineBreaks} field. \emph{The settings 
in this field will only be considered if the \texttt{-m} switch has been used}.
A snippet of the default settings of this field is shown in \cref{lst:modifylinebreaks}.

Having read the previous paragraph, it should sound reasonable that, if you call \texttt{latexindent.pl}
using the \texttt{-m} switch, then you give it permission to modify line breaks in your file, 
but let's be clear:

\begin{warning}
  If you call \texttt{latexindent.pl} with the \texttt{-m} switch, then you 
  are giving it permission to modify line breaks. By default, the only 
  thing that will happen is that multiple blank lines will be condensed into 
  one blank line; many other settings are possible, discussed next.
\end{warning}
All YAML-based details in this section only apply if the \texttt{-m} switch is active.

\yamltitle{preserveBlankLines}{0|1}
This field is directly related to \emph{poly-switches}, discussed below. 
By default, it is set to \texttt{1}, which means that blank lines will 
be protected from removal; however, regardless of this setting, multiple 
blank lines can be condensed if \texttt{condenseMultipleBlankLinesInto} is
greater than \texttt{0}, discussed next.

\yamltitle{condenseMultipleBlankLinesInto}*{integer $\geq 0$}
Assuming that this switch takes an integer value greater than \texttt{0}, \texttt{latexindent.pl} will condense multiple blank lines into
the number of blank lines illustrated by this switch. As an example, \cref{lst:mlb1} shows a sample file 
with blank lines; upon running
\begin{commandshell}
latexindent.pl myfile.tex -m  
\end{commandshell}
the output is shown in \cref{lst:mlb1-out}; note that the multiple blank lines have been 
condensed into one blank line, and note also that we have used the \texttt{-m} switch!

\begin{minipage}{.45\textwidth}
\cmhlistingsfromfile{demonstrations/mlb1.tex}{\texttt{mlb1.tex}}{lst:mlb1}
\end{minipage}%
\hfill
\begin{minipage}{.45\textwidth}
\cmhlistingsfromfile{demonstrations/mlb1-out.tex}{\texttt{mlb1.tex} out output}{lst:mlb1-out}
\end{minipage}

\subsection{Poly-switches}
Every other field in the \texttt{modifyLineBreaks} field uses poly-switch, and can take
one of four integer values\footnote{visual learners might like to associate one of the four circles in the logo with one of the four given values}:
\begin{itemize}[font=\bfseries]
  \item[$-1$] \emph{remove mode}: line breaks before or after the \emph{<part of thing>} can be removed (assuming that \texttt{preserveBlankLines} is set to \texttt{0});
  \item[0] \emph{off mode}: line breaks will not be modified for the \emph{<part of thing>} under consideration;
  \item[1] \emph{add mode}: a line break will be added before or after the \emph{<part of thing>} under consideration, assuming that
    there is not already a line break before or after the \emph{<part of thing>};
  \item[2] \emph{comment then add mode}: a comment symbol will be added, followed by a line break before or after the \emph{<part of thing>} under consideration, assuming that
    there is not already a comment and line break before or after the \emph{<part of thing>}.
\end{itemize}
All poly-switches are \emph{off} by default; \texttt{latexindent.pl} searches first of all for per-name settings, and then followed by global per-thing settings.

\subsection{modifyLineBreaks for environments}
We start by viewing a snippet of \texttt{defaultSettings.yaml} in \cref{lst:environments-mlb}; note that it contains \emph{global} settings (immediately
after the \texttt{environments} field) and that \emph{per-name} settings are also allowed -- in the case of \cref{lst:environments-mlb}, settings 
for \texttt{equation*} have been specified. 

\cmhlistingsfromfile[firstnumber=347,linerange={347-356},style=yaml-LST,numbers=left,]{../defaultSettings.yaml}[width=.8\linewidth,before=\centering,MLB-TCB]{\texttt{environments}}{lst:environments-mlb}

\subsubsection{Adding line breaks (poly-switches set to $1$ or $2$)}
Let's begin with the simple example given in \cref{lst:env-mlb1}; note that we have annotated key parts of the file using $\BeginStartsOnOwnLine$, 
$\BodyStartsOnOwnLine$, $\EndStartsOnOwnLine$ and $\EndFinishesWithLineBreak$, these will be related to fields specified in \cref{lst:environments-mlb}.

\begin{cmhlistings}[escapeinside={(*@}{@*)}]{\texttt{env-mlb1.tex}}{lst:env-mlb1}
before words (*@$\BeginStartsOnOwnLine$@*)\begin{myenv}(*@$\BodyStartsOnOwnLine$@*)body of myenv(*@$\EndStartsOnOwnLine$@*)\end{myenv}(*@$\EndFinishesWithLineBreak$@*) after words
\end{cmhlistings}

Let's explore \texttt{BeginStartsOnOwnLine} and \texttt{BodyStartsOnOwnLine} in \cref{lst:env-mlb1,lst:env-mlb2}, and in particular, 
let's allow each of them in turn to take a value of $1$.

\begin{minipage}{.45\textwidth}
\cmhlistingsfromfile[style=yaml-LST]{demonstrations/env-mlb1.yaml}[MLB-TCB]{\texttt{env-mlb1.yaml}}{lst:env-mlb1}
\end{minipage}
\hfill
\begin{minipage}{.45\textwidth}
\cmhlistingsfromfile[style=yaml-LST]{demonstrations/env-mlb2.yaml}[MLB-TCB]{\texttt{env-mlb2.yaml}}{lst:env-mlb2}
\end{minipage}

After running the following commands,
\begin{commandshell}
latexindent.pl -m env-mlb.tex -l env-mlb1.yaml
latexindent.pl -m env-mlb.tex -l env-mlb2.yaml
\end{commandshell}
the output is as in \cref{lst:env-mlb-mod1,lst:env-mlb-mod2}.

\begin{sidebyside}
\begin{minipage}{.57\linewidth}
\cmhlistingsfromfile{demonstrations/env-mlb-mod1.tex}{\texttt{env-mlb.tex} using \cref{lst:env-mlb1}}{lst:env-mlb-mod1}
\end{minipage}
\hfill
\begin{minipage}{.42\linewidth}
\cmhlistingsfromfile{demonstrations/env-mlb-mod2.tex}{\texttt{env-mlb.tex} using \cref{lst:env-mlb2}}{lst:env-mlb-mod2}
\end{minipage}
\end{sidebyside}

There are a couple of points to note:
\begin{itemize}
  \item in \cref{lst:env-mlb-mod1} a line break has been added at the point denoted by $\BeginStartsOnOwnLine$ in \cref{lst:env-mlb1}; no 
    other line breaks have been changed;
  \item in \cref{lst:env-mlb-mod2} a line break has been added at the point denoted by $\BodyStartsOnOwnLine$ in \cref{lst:env-mlb1}; 
    furthermore, note that the \emph{body} of \texttt{myenv} has received the appropriate (default) indentation.
\end{itemize}

Let's now change each of the \texttt{1} values in \cref{lst:env-mlb1,lst:env-mlb2} so that they are $2$ and 
save them into \texttt{env-mlb3.yaml} and \texttt{env-mlb4.yaml} respectively (see \cref{lst:env-mlb3,lst:env-mlb4}).

\begin{minipage}{.45\textwidth}
\cmhlistingsfromfile[style=yaml-LST]{demonstrations/env-mlb3.yaml}[MLB-TCB]{\texttt{env-mlb3.yaml}}{lst:env-mlb3}
\end{minipage}
\hfill
\begin{minipage}{.45\textwidth}
\cmhlistingsfromfile[style=yaml-LST]{demonstrations/env-mlb4.yaml}[MLB-TCB]{\texttt{env-mlb4.yaml}}{lst:env-mlb4}
\end{minipage}

Upon running  commands analogous to the above, we obtain \cref{lst:env-mlb-mod3,lst:env-mlb-mod4}.

\begin{sidebyside}
\begin{minipage}{.57\linewidth}
\cmhlistingsfromfile{demonstrations/env-mlb-mod3.tex}{\texttt{env-mlb.tex} using \cref{lst:env-mlb3}}{lst:env-mlb-mod3}
\end{minipage}
\hfill
\begin{minipage}{.42\linewidth}
\cmhlistingsfromfile{demonstrations/env-mlb-mod4.tex}{\texttt{env-mlb.tex} using \cref{lst:env-mlb4}}{lst:env-mlb-mod4}
\end{minipage}
\end{sidebyside}

Note that line breaks have been added as in \cref{lst:env-mlb-mod1,lst:env-mlb-mod2}, but this time a comment symbol
has been added before adding the line break; in both cases, trailing horizontal 
space has been stripped before doing so.

Let's explore \texttt{EndStartsOnOwnLine} and \texttt{EndFinishesWithLineBreak} in \cref{lst:env-mlb5,lst:env-mlb6}, 
and in particular, let's allow each of them in turn to take a value of $1$.

\begin{minipage}{.49\textwidth}
\cmhlistingsfromfile[style=yaml-LST]{demonstrations/env-mlb5.yaml}[MLB-TCB]{\texttt{env-mlb5.yaml}}{lst:env-mlb5}
\end{minipage}
\hfill
\begin{minipage}{.49\textwidth}
\cmhlistingsfromfile[style=yaml-LST]{demonstrations/env-mlb6.yaml}[MLB-TCB]{\texttt{env-mlb6.yaml}}{lst:env-mlb6}
\end{minipage}

After running the following commands,
\begin{commandshell}
latexindent.pl -m env-mlb.tex -l env-mlb5.yaml
latexindent.pl -m env-mlb.tex -l env-mlb6.yaml
\end{commandshell}
the output is as in \cref{lst:env-mlb-mod5,lst:env-mlb-mod6}.

\begin{sidebyside}
\begin{minipage}{.42\linewidth}
\cmhlistingsfromfile{demonstrations/env-mlb-mod5.tex}{\texttt{env-mlb.tex} using \cref{lst:env-mlb5}}{lst:env-mlb-mod5}
\end{minipage}
\hfill
\begin{minipage}{.57\linewidth}
\cmhlistingsfromfile{demonstrations/env-mlb-mod6.tex}{\texttt{env-mlb.tex} using \cref{lst:env-mlb6}}{lst:env-mlb-mod6}
\end{minipage}
\end{sidebyside}

There are a couple of points to note:
\begin{itemize}
  \item in \cref{lst:env-mlb-mod5} a line break has been added at the point denoted by $\EndStartsOnOwnLine$ in \vref{lst:env-mlb1}; no 
    other line breaks have been changed and the \lstinline!\end{myenv}! statement has \emph{not} received indentation (as intended);
  \item in \cref{lst:env-mlb-mod6} a line break has been added at the point denoted by $\EndFinishesWithLineBreak$ in \vref{lst:env-mlb1}; 
\end{itemize}

Let's now change each of the \texttt{1} values in \cref{lst:env-mlb5,lst:env-mlb6} so that they are $2$ and 
save them into \texttt{env-mlb7.yaml} and \texttt{env-mlb8.yaml} respectively (see \cref{lst:env-mlb7,lst:env-mlb8}).

\begin{minipage}{.49\textwidth}
\cmhlistingsfromfile[style=yaml-LST]{demonstrations/env-mlb7.yaml}[MLB-TCB]{\texttt{env-mlb7.yaml}}{lst:env-mlb7}
\end{minipage}
\hfill
\begin{minipage}{.49\textwidth}
\cmhlistingsfromfile[style=yaml-LST]{demonstrations/env-mlb8.yaml}[MLB-TCB]{\texttt{env-mlb8.yaml}}{lst:env-mlb8}
\end{minipage}

Upon running  commands analogous to the above, we obtain \cref{lst:env-mlb-mod7,lst:env-mlb-mod8}.

\begin{sidebyside}
\begin{minipage}{.42\linewidth}
\cmhlistingsfromfile{demonstrations/env-mlb-mod7.tex}{\texttt{env-mlb.tex} using \cref{lst:env-mlb7}}{lst:env-mlb-mod7}
\end{minipage}
\hfill
\begin{minipage}{.57\linewidth}
\cmhlistingsfromfile{demonstrations/env-mlb-mod8.tex}{\texttt{env-mlb.tex} using \cref{lst:env-mlb8}}{lst:env-mlb-mod8}
\end{minipage}
\end{sidebyside}

Note that line breaks have been added as in \cref{lst:env-mlb-mod5,lst:env-mlb-mod6}, but this time a comment symbol
has been added before adding the line break; in both cases, trailing horizontal 
space has been stripped before doing so.

If you ask \texttt{latexindent.pl} to add a line break (possibly with a comment) using a poly-switch value of $1$ (or $2$),
it will only do so if necessary. For example, if you process the file in \cref{lst:env-mlb2} using any of the YAML 
files presented so far in this section, it will be left unchanged.

\begin{minipage}{.45\linewidth}
\cmhlistingsfromfile{demonstrations/env-mlb2.tex}{\texttt{env-mlb2.tex}}{lst:mlb2}
\end{minipage}
\hfill
\begin{minipage}{.45\linewidth}
\cmhlistingsfromfile{demonstrations/env-mlb3.tex}{\texttt{env-mlb3.tex}}{lst:mlb3}
\end{minipage}

In contrast, the output from processing the file in \cref{lst:mlb3} will vary depending 
on the poly-switches used; in \cref{lst:env-mlb3-mod2} you'll see that the comment symbol after 
the \lstinline!\begin{myenv}! has been moved to the next line, as \texttt{BodyStartsOnOwnLine} 
  is set to \texttt{1}. In \cref{lst:env-mlb3-mod4} you'll see that the comment has been accounted 
  for correctly, and that, because \texttt{BodyStartsOnOwnLine} has been set to \texttt{2}, 
  the comment symbol has \emph{not} been moved to its own line. You're encouraged to experiment 
  with \cref{lst:mlb3} and the other poly-switches considered so far.

\begin{minipage}{.45\linewidth}
\cmhlistingsfromfile{demonstrations/env-mlb3-mod2.tex}{\texttt{env-mlb3.tex} using \vref{lst:env-mlb2}}{lst:env-mlb3-mod2}
\end{minipage}
\hfill
\begin{minipage}{.45\linewidth}
\cmhlistingsfromfile{demonstrations/env-mlb3-mod4.tex}{\texttt{env-mlb3.tex} using \vref{lst:env-mlb4}}{lst:env-mlb3-mod4}
\end{minipage}

The details of the discussion in this section have concerned \emph{global} poly-switches in the \texttt{environments} field;
each switch can also be specified on a \emph{per-name} basis, which would take priority over the global values; with 
reference to \vref{lst:environments-mlb}, an example is shown for the \texttt{equation*} environment.

\subsubsection{Removing line breaks (poly-switches set to $0$)}
Setting poly-switches to $0$ tells \texttt{latexindent.pl} to remove line breaks, if necessary. We will consider the
example codes given in \cref{lst:mlb4,lst:mlb5}, noting in particular the positions of 
the line break highlighters, $\BeginStartsOnOwnLine$, $\BodyStartsOnOwnLine$, $\EndStartsOnOwnLine$
and $\EndFinishesWithLineBreak$.

\begin{minipage}{.45\linewidth}
\begin{cmhlistings}[escapeinside={(*@}{@*)}]{\texttt{env-mlb4.tex}}{lst:mlb4}
before words (*@$\BeginStartsOnOwnLine$@*)
\begin{myenv}  %(*@$\BodyStartsOnOwnLine$@*)
body of myenv%(*@$\EndStartsOnOwnLine$@*)
\end{myenv}% (*@$\EndFinishesWithLineBreak$@*)
after words
\end{cmhlistings}
\end{minipage}%
\hfill
\begin{minipage}{.45\linewidth}
\begin{cmhlistings}[escapeinside={(*@}{@*)}]{\texttt{env-mlb1.tex}}{lst:mlb5}
before words (*@$\BeginStartsOnOwnLine$@*)


\begin{myenv}  %(*@$\BodyStartsOnOwnLine$@*)


body of myenv%(*@$\EndStartsOnOwnLine$@*)


\end{myenv}% (*@$\EndFinishesWithLineBreak$@*)

after words
\end{cmhlistings}
\end{minipage}%
