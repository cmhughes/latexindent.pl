\title{%
	\begin{tcolorbox}[
			width=5.2cm,
			boxrule=0pt,
			colframe=white!40!black,
			colback=white,
			rightrule=2pt,
			sharp corners,
			enhanced,
			overlay={\node[anchor=north east,outer sep=2pt] at ([xshift=3cm,yshift=4mm]frame.north east) {\includegraphics[width=3cm]{logo}}; }]
		\centering\ttfamily\bfseries latexindent.pl\\[1cm] Version 3.24.1
	\end{tcolorbox}
}
\author{Chris Hughes \thanks{and contributors!
		See \vref{sec:contributors}.
		For
		all communication, please visit \cite{latexindent-home}.}}
\date{2024-05-12}
\maketitle
\begin{adjustwidth}{1cm}{1cm}
	\small
	\texttt{latexindent.pl} is a \texttt{Perl} script that indents \texttt{.tex} (and other) files according to an indentation scheme that the user can modify to suit their taste.
	Environments, including those with alignment delimiters (such as \texttt{tabular}), and commands, including those that can split braces and brackets across lines, are \emph{usually} handled correctly by the script.
	Options for \texttt{verbatim}-like environments and commands, together with indentation after headings (such as \lstinline!chapter!, \lstinline!section!, etc) are also available.
	The script also has the ability to modify line breaks, and to add comment symbols and blank lines; furthermore, it permits string or
	regex-based substitutions.
	All user options are customisable via the switches and the YAML interface. 

    tl;dr, a quick start guide is given in \vref{sec:quickstart}.
\end{adjustwidth}
