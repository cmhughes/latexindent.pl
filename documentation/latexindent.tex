% arara: pdflatex: {shell: yes}
% !arara: pdflatex: {shell: yes}
% !arara: bibtex
% !arara: pdflatex
% !arara: pdflatex
% !arara: pdflatex
% !arara: indent: {overwrite: yes, trace: yes, localSettings: yes, silent: yes}
\documentclass[10pt]{article}
%   This program is free software: you can redistribute it and/or modify
%   it under the terms of the GNU General Public License as published by
%   the Free Software Foundation, either version 3 of the License, or
%   (at your option) any later version.
%
%   This program is distributed in the hope that it will be useful,
%   but WITHOUT ANY WARRANTY; without even the implied warranty of
%   MERCHANTABILITY or FITNESS FOR A PARTICULAR PURPOSE.  See the
%   GNU General Public License for more details.
%
%   See <http://www.gnu.org/licenses/>.
\usepackage[left=4.5cm,right=2.5cm,showframe=false,
	top=2cm,bottom=1.5cm]{geometry}                      % page setup
\usepackage{lmodern}
\usepackage{parskip}                                 % paragraph skips
\usepackage{booktabs}                                % beautiful tables
\usepackage{listings}                                % nice verbatim environments
\usepackage{titlesec}                                % customize headings
\usepackage{changepage}                              % adjust width of page
\usepackage{fancyhdr}                                % headers & footers
\usepackage{wrapfig}
\usepackage[sc,format=hang,font=small]{caption}      % captions
\usepackage[backend=bibtex]{biblatex}                % bibliography
\usepackage{tcolorbox}                                % framed environments
\usepackage{xparse}
\usepackage[charter]{mathdesign}                     % changes font
\usepackage[expansion=false,kerning=true]{microtype} % better kerning
\usepackage{enumitem}                                % custom lists
\usepackage{longtable}
\usepackage{array}
% setup gitinfo2, as in the manual, details just above begin{document}
\usepackage[mark,grumpy]{gitinfo2}
% tcolorbox libraries
\tcbuselibrary{breakable,skins,listings,minted,xparse}
%\usepackage{varioref}                                % clever referencing
%\tcbuselibrary{documentation,breakable,skins,minted}
% tikz libraries
\usetikzlibrary{positioning}
\usetikzlibrary{decorations.pathmorphing}
\usetikzlibrary{decorations,shapes}
\usepackage{varioref}                                % clever referencing
\usepackage{hyperref}
\hypersetup{
	pdfauthor={Chris Hughes},
	pdftitle={latexindent.pl package},
	pdfkeywords={perl;beautify;indentation},
	bookmarksnumbered,
	bookmarksopen,
	linktocpage,
	colorlinks=true,
	linkcolor=blue,
	citecolor=black,
}
\usepackage{cleveref}

\addbibresource{latex-indent}
\addbibresource{contributors}


% http://tex.stackexchange.com/questions/122135/how-to-add-a-png-icon-on-the-right-side-of-a-tcolorbox-title
\newtcolorbox{warning}{parbox=false,breakable,enhanced,arc=0mm,colback=red!5,colframe=red,leftrule=12mm,%
overlay={\node[anchor=north west,outer sep=2pt] at (frame.north west) {\includegraphics[width=8mm]{warning}}; }
}

\definecolor{harvestgold}{cmyk}{0.00, 0.05, 0.51, 0.07}  %EDE275
\definecolor{cmhgold}{cmyk}{0,0.178,0.909,0.008}         %FDD017
\makeatletter
\tcbset{
	addtolol/.style={list entry={\kvtcb@title},add to list={lol}{lstlisting}},
	cmhlistings/.style={
			%	width=\linewidth,
			%breakable,
			colback=blue!5!white,
			colframe=white!85!black,
			top=0cm,
			bottom=0cm,
			left=0mm,
			listing only,
			center title,
			listing engine=minted,
			minted style=colorful,
			addtolol,
			boxrule=0pt,
			toprule=1pt,bottomrule=1pt,
			titlerule=1pt,
			colframe=white!25!black,colback=white,
			sharp corners,
			colbacktitle=white!75!black,
		},
	yaml-TCB/.style={
			listing only,
			listing engine=listings,
			left=0cm,
			boxrule=0pt,
			%leftrule=3pt,
			sharp corners,
			center title,
            %colbacktitle=white!75!black,
			colbacktitle=white!75!blue,
			colframe=white!25!blue,
            colback=white,
            toprule=2pt,
            titlerule=2pt,
            bottomrule=1pt,
		},
    MLB-TCB/.style={
            enhanced,
			listing only,
			listing engine=listings,
			left=0cm,
			boxrule=0pt,
			sharp corners,
			center title,
            colframe=cmhgold,
			colbacktitle=harvestgold,
            titlerule=2pt,
            toprule=2pt,
            width=0.9\linewidth,
            before=\centering,
            bottomrule=1pt,
            %overlay={\node[anchor=north east,outer sep=2pt] at ([xshift=8mm]frame.north east) {\includegraphics[width=8mm]{logo}}; }
    }
}

\newtcblisting[use counter=lstlisting]{cmhlistings}[3][]{%
	cmhlistings,
	center title,
	title={\color{black}{\scshape Listing \thetcbcounter}: ~#2},label={#3},
	listing engine=listings,
	listing options={#1},
}

\DeclareTCBInputListing[use counter=lstlisting]{\cmhlistingsfromfile}{O{} m O{} m m}{%
	cmhlistings,
	listing file={#2},
	listing options={#1},
	title={\color{black}{\scshape Listing \thetcbcounter}: ~#4},label={#5},
	#3,
}

% command shell
\newtcblisting{commandshell}{colback=black,colupper=white,colframe=yellow!75!black,
	listing only,listing options={style=tcblatex,language=sh,
			morekeywords={latexindent,pl},
			keywordstyle=\color{blue!35!white}\bfseries,
		},
	listing engine=listings,
	left=0cm,
	every listing line={\textcolor{red}{\small\ttfamily\fontseries{b}\selectfont cmh:$\sim$\$ }}}


\lstset{%
	basicstyle=\small\ttfamily,language={[LaTeX]TeX},
	%	numbers=left,
	numberstyle=\ttfamily%\small,
	breaklines=true,
	%   frame=single,framexleftmargin=8mm, xleftmargin=8mm,
	%	prebreak = \raisebox{0ex}[0ex][0ex]{\ensuremath{\hookrightarrow}},
	%	backgroundcolor=\color{green!5},frameround=fttt,
	%	rulecolor=\color{blue!70!black},
	keywordstyle=\color{blue},                    % keywords
	commentstyle=\color{purple},    % comments
	tabsize=3,
	%xleftmargin=1.5em,
}%
\DeclareTCBListing[use counter=lstlisting]{yaml}{O{} m O{} m}{
	yaml-TCB,
	listing options={
			style=tcblatex,
			numbers=none,
			numberstyle=\color{red},
			#1,
		},
	title={\color{black}{\scshape Listing \thetcbcounter}: ~#2},label={#4},
	#3,
}

\lstdefinestyle{yaml-LST}{
	style=tcblatex,
	numbers=none,
	%numbers=left,
	numberstyle=\color{red},
}

\lstdefinestyle{demo}{
	numbers=none,
	linewidth=1.25\textwidth,
	columns=fullflexible,
}



% stars around contributors
\pgfdeclaredecoration{stars}{initial}{
	\state{initial}[width=15pt]
	{
		\pgfmathparse{round(rnd*100)}
		\pgfsetfillcolor{yellow!\pgfmathresult!orange}
		\pgfsetstrokecolor{yellow!\pgfmathresult!red}
		\pgfnode{star}{center}{}{}{\pgfusepath{stroke,fill}}
	}
	\state{final}
	{
		\pgfpathmoveto{\pgfpointdecoratedpathlast}
	}
}

\newtcolorbox{stars}{%
enhanced jigsaw,
breakable, % allow page breaks
left=0cm,
top=0cm,
before skip=0.2cm,
boxsep=0cm,
frame style={draw=none,fill=none}, % hide the default frame
colback=white,
overlay={
\draw[inner sep=0,minimum size=rnd*15pt+2pt]
decorate[decoration={stars,segment length=2cm}] {
decorate[decoration={zigzag,segment length=2cm,amplitude=0.3cm}] {
([xshift=-.5cm,yshift=0.1cm]frame.south west) --  ([xshift=-.5cm,yshift=0.4cm]frame.north west)
}};
\draw[inner sep=0,minimum size=rnd*15pt+2pt]
decorate[decoration={stars,segment length=2cm}] {
decorate[decoration={zigzag,segment length=2cm,amplitude=0.3cm}] {
([xshift=.75cm,yshift=0.1cm]frame.south east) --  ([xshift=.75cm,yshift=0.6cm]frame.north east)
}};
},
% paragraph skips obeyed within tcolorbox
parbox=false,
}

% copied from /usr/local/texlive/2013/texmf-dist/tex/latex/biblatex/bbx/numeric.bbx
% the only modification is the \stars and \endstars
\defbibenvironment{specialbib}
{\stars\list
	{\printtext[labelnumberwidth]{%
			\printfield{prefixnumber}%
			\printfield{labelnumber}}}
	{\setlength{\labelwidth}{\labelnumberwidth}%
		\setlength{\leftmargin}{\labelwidth}%
		\setlength{\labelsep}{\biblabelsep}%
		\addtolength{\leftmargin}{\labelsep}%
		\setlength{\itemsep}{\bibitemsep}%
		\setlength{\parsep}{\bibparsep}}%
	\renewcommand*{\makelabel}[1]{\hss##1}}
{\endlist\endstars}
{\item}

\newtcbox{yamltitlebox}[1][]{colframe=black!50!white,boxrule=.5pt,sharp corners,#1}

\newcommand{\flagbox}[1]{%
	\par
	\makebox[30pt][l]{%
		\hspace{-2cm}%
		\ttfamily\fontseries{b}\selectfont #1
	}%
}

\NewDocumentCommand{\yamltitle}{O{} m s m}{%
	\par
	\makebox[30pt][l]{%
		\hspace{-2cm}%
		\yamltitlebox[#1]{%
				{\ttfamily\fontseries{b}\selectfont{\color{blue!80!white}#2}}: %
			\IfBooleanTF{#3}
			{$\langle$\itshape #4$\rangle$}
			{{\bfseries #4}}
		}}
	\par\nobreak%
}

\newcommand{\fixthis}[1]
{%
	\marginpar{\huge \color{red} \framebox{FIX}}%
	\typeout{FIXTHIS: p\thepage : #1^^J}%
}
% custom section
\titleformat{\section}
{\normalfont\Large\bfseries}
{\llap{\thesection\hskip.5cm}}
{0pt}
{}
% custom subsection
\titleformat{\subsection}
{\normalfont\bfseries}
{\llap{\thesubsection\hskip.5cm}}
{0pt}
{}
% custom subsubsection
\titleformat{\subsubsection}
{\normalfont\bfseries}
{\llap{\thesubsubsection\hskip.5cm}}
{0pt}
{}


\titlespacing\section{0pt}{12pt plus 4pt minus 2pt}{-5pt plus 2pt minus 2pt}
\titlespacing\subsection{0pt}{12pt plus 4pt minus 2pt}{-6pt plus 2pt minus 2pt}
\titlespacing\subsubsection{0pt}{12pt plus 4pt minus 2pt}{-6pt plus 2pt minus 2pt}


% cleveref settings
\crefname{table}{Table}{Tables}
\Crefname{table}{Table}{Tables}
\crefname{figure}{Figure}{Figures}
\Crefname{figure}{Figure}{Figures}
\crefname{section}{Section}{Sections}
\Crefname{section}{Section}{Sections}
\crefname{listing}{Listing}{Listings}
\Crefname{listing}{Listing}{Listings}

% headers and footers
\fancyhf{} % delete current header and footer
\fancyhead[R]{\bfseries\thepage%
\tikz[remember picture,overlay] {
\node at (1,0){\includegraphics{logo-bw}}; }
}
\fancyheadoffset[L]{3cm}
\pagestyle{fancy}

% renew plain style
\fancypagestyle{plain}{%
	\fancyhf{} % clear all header and footer fields
	\renewcommand{\headrulewidth}{0pt}
	\renewcommand{\footrulewidth}{0pt}}

% sidebyside environment
\newenvironment{sidebyside}{\begin{adjustwidth}{-3cm}{0cm}}{\end{adjustwidth}}

% symbols for the m switch
\newcommand{\BeginStartsOnOwnLine}{\color{red}\spadesuit}
\newcommand{\BodyStartsOnOwnLine}{\color{red}\heartsuit}
\newcommand{\EndStartsOnOwnLine}{\color{red}\diamondsuit}
\newcommand{\EndFinishesWithLineBreak}{\color{red}\clubsuit}
\newcommand{\ElseStartsOnOwnLine}{\color{red}\bigstar}
\newcommand{\ElseFinishesWithLineBreak}{\color{red}\square}
\newcommand{\EqualsStartsOnOwnLine}{\color{red}\bullet}

% gitinfo2 settings
\renewcommand{\gitMark}{\gitBranch\,@\,\gitAbbrevHash{}\,\textbullet{}\,\gitAuthorDate}

% setting up gitinfo2:
%   copy the file post-xxx-sample.txt from http://mirror.ctan.org/macros/latex/contrib/gitinfo2
%   and put it in .git/hooks/post-checkout
% then
%   cd .git/hooks
%   chmod g+x post-checkout
%   chmod +x post-checkout
%   cp post-checkout post-commit
%   cp post-checkout post-merge
%   cd ../..
%   git checkout master
%   git checkout develop
%   ls .git
% and you should see gitHeadInfo.gin
\begin{document}
% \begin{noindent}
\title{\ttfamily\bfseries latexindent.pl\\[1cm] Version 3.0}
% \end{noindent}
\author{Chris Hughes \footnote{and contributors! (See \vref{sec:contributors}.) For
		all communication, please visit \cite{latexindent-home}.}}
\maketitle
\begin{abstract}
	\texttt{latexindent.pl} is a \texttt{Perl} script that indents \texttt{.tex} (and other)
	files according to an indentation scheme that the user can modify to suit their
	taste. Environments, including those with alignment delimiters (such as \texttt{tabular}),
	and commands, including those that can split braces and brackets across lines,
	are \emph{usually} handled correctly by the script. Options for \texttt{verbatim}-like
	environments and indentation after headings (such as \lstinline!chapter!, \lstinline!section!, etc)
	are also available. The script also has the ability to modifiy line breaks, and add
	comment symbols.
\end{abstract}
\tableofcontents
\lstlistoflistings

%\section{Introduction}
\subsection{Thanks}
	I first created \texttt{latexindent.pl} to help me format chapter files
	in a big project. After I blogged about it on the
	\TeX{} stack exchange \cite{cmhblog} I received some positive feedback and
	follow-up feature requests. A big thank you to Harish Kumar who
	helped to develop and test the initial versions of the script.

	The \texttt{YAML}-based interface of \texttt{latexindent.pl} was inspired
	by the wonderful \texttt{arara} tool; any similarities are deliberate, and
	I hope that it is perceived as the compliment that it is. Thank you to Paulo Cereda and the
	team for releasing this awesome tool; I initially worried that I was going to
	have to make a GUI for \texttt{latexindent.pl}, but the release of \texttt{arara}
	has meant there is no need.

	There have been several contributors to the project so far (and hopefully more in
	the future!); thank you very much to the people detailed in \vref{sec:contributors}
	for their valued contributions, and thank you to those who report bugs and request features
	at \cite{latexindent-home}.

\subsection{License}
	\texttt{latexindent.pl} is free and open source, and it always will be.
	Before you start using it on any important files, bear in mind that \texttt{latexindent.pl} has the option to overwrite your \texttt{.tex} files.
	It will always make at least one backup (you can choose how many it makes, see \cpageref{page:onlyonebackup})
	but you should still be careful when using it. The script has been tested on many
	files, but there are some known limitations (see \cref{sec:knownlimitations}).
	You, the user, are responsible for ensuring that you maintain backups of your files
	before running \texttt{latexindent.pl} on them. I think it is important at this
	stage to restate an important part of the license here:
	\begin{quote}\itshape
		This program is distributed in the hope that it will be useful,
		but WITHOUT ANY WARRANTY; without even the implied warranty of
		MERCHANTABILITY or FITNESS FOR A PARTICULAR PURPOSE.  See the
		GNU General Public License for more details.
	\end{quote}
	There is certainly no malicious intent in releasing this script, and I do hope
	that it works as you expect it to; if it does not, please first of all
	make sure that you have the correct settings, and then feel free to let me know at \cite{latexindent-home} with a
	complete minimum working example as I would like to improve the code as much as possible.
	\begin{warning}
		Before you try the script on anything important (like your thesis), test it
		out on the sample files in the \texttt{test-case} directory \cite{latexindent-home}.
	\end{warning}

	\emph{If you have used any version 2.* of \texttt{latexindent.pl}, there
		are a few changes to the interface; see \vref{app:differences} and the comments
		throughout this document for details}.

%% arara: pdflatex: { files: [latexindent]}
\section{Demonstration: before and after}
 Let's give a demonstration of some before and after code -- after all, you probably won't
 want to try the script if you don't much like the results. You might also like to watch
 the video demonstration I made on youtube \cite{cmh:videodemo}

 As you look at \crefrange{lst:filecontentsbefore}{lst:pstricksafter}, remember that
 \texttt{latexindent.pl} is just following its rules, and there is nothing particular
 about these code snippets. All of the rules can be modified so that you can personalise
 your indentation scheme.

 In each of the samples given in \crefrange{lst:filecontentsbefore}{lst:pstricksafter} the
 `before' case is a `worst case scenario' with no effort to make indentation. The `after'
 result would be the same, regardless of the leading white space at the beginning of each
 line which is stripped by \texttt{latexindent.pl} (unless a \texttt{verbatim}-like
 environment or \texttt{noIndentBlock} is specified -- more on this in
 \cref{sec:defuseloc}).

 \begin{widepage}
	 \centering
	 \begin{minipage}{.45\linewidth}
		 \cmhlistingsfromfile{demonstrations/filecontents1.tex}{\texttt{filecontents1.tex}}{lst:filecontentsbefore}
	 \end{minipage}\hfill
	 \begin{minipage}{.45\linewidth}
		 \cmhlistingsfromfile{demonstrations/filecontents1-default.tex}{\texttt{filecontents1.tex} default output}{lst:filecontentsafter}
	 \end{minipage}%

	 \begin{minipage}{.45\linewidth}
		 \cmhlistingsfromfile{demonstrations/tikzset.tex}{\texttt{tikzset.tex}}{lst:tikzsetbefore}
	 \end{minipage}\hfill
	 \begin{minipage}{.45\linewidth}
		 \cmhlistingsfromfile{demonstrations/tikzset-default.tex}{\texttt{tikzset.tex} default output}{lst:tikzsetafter}
	 \end{minipage}%

	 \begin{minipage}{.45\linewidth}
		 \cmhlistingsfromfile{demonstrations/pstricks.tex}{\texttt{pstricks.tex}}{lst:pstricksbefore}
	 \end{minipage}\hfill
	 \begin{minipage}{.45\linewidth}
		 \cmhlistingsfromfile{demonstrations/pstricks-default.tex}{\texttt{pstricks.tex} default output}{lst:pstricksafter}
	 \end{minipage}%
 \end{widepage}


%% arara: pdflatex: { files: [latexindent]}
\section{How to use the script}
 \texttt{latexindent.pl} ships as part of the \TeX Live distribution for
 Linux and Mac users; \texttt{latexindent.exe} ships as part of the \TeX Live and
 MiK\TeX{} distributions for Windows users. These files are also available from github
 \cite{latexindent-home} should you wish to use them without a \TeX{} distribution; in
 this case, you may like to read \vref{sec:updating-path} which details how the
 \texttt{path} variable can be updated.

 In what follows, we will always refer to \texttt{latexindent.pl}, but depending on your
 operating system and preference, you might substitute \texttt{latexindent.exe} or simply
 \texttt{latexindent}.

 There are two ways to use \texttt{latexindent.pl}: from the command line, and using
 \texttt{arara}; we discuss these in \cref{sec:commandline} and \cref{sec:arara}
 respectively. We will discuss how to change the settings and behaviour of the script in
 \vref{sec:defuseloc}.

 \texttt{latexindent.pl} ships with \texttt{latexindent.exe} for Windows
 users, so that you can use the script with or without a Perl distribution. If you plan to
 use \texttt{latexindent.pl} (i.e, the original Perl script) then you will need a few
 standard Perl modules -- see \vref{sec:requiredmodules} for
 details;%
 \announce{2018-01-13}{perl module installer helper script} in particular, note that a module installer helper script is shipped with
 \texttt{latexindent.pl}.

\subsection{From the command line}\label{sec:commandline}
	\texttt{latexindent.pl} has a number of different switches/flags/options, which
	can be combined in any way that you like, either in short or long form as detailed below.
	\texttt{latexindent.pl} produces a \texttt{.log} file, \texttt{indent.log}, every time it
	is run; the name of the log file can be customised, but we will refer to the log file as
	\texttt{indent.log} throughout this document. There is a base of information that is
	written to \texttt{indent.log}, but other additional information will be written
	depending on which of the following options are used.

\flagbox{-v, --version}
	\index{switches!-v, --version definition and details}
	\announce{2017-06-25}{version}
	\begin{commandshell}
latexindent.pl -v
\end{commandshell}
	This will output only the version number to the terminal.

\flagbox{-h, --help}
	\index{switches!-h, --help definition and details}

	\begin{commandshell}
latexindent.pl -h
\end{commandshell}

	As above this will output a welcome message to the terminal, including the version number
	and available options.
	\begin{commandshell}
latexindent.pl myfile.tex
\end{commandshell}

	This will operate on \texttt{myfile.tex}, but will simply output to your terminal;
	\texttt{myfile.tex} will not be changed by \texttt{latexindent.pl} in any way using this
	command.

\flagbox{-w, --overwrite}
	\index{switches!-w, --overwrite definition and details}
	\index{backup files!overwrite switch, -w}
	\begin{commandshell}
latexindent.pl -w myfile.tex
latexindent.pl --overwrite myfile.tex
latexindent.pl myfile.tex --overwrite 
\end{commandshell}

	This \emph{will} overwrite \texttt{myfile.tex}, but it will make a copy of
	\texttt{myfile.tex} first. You can control the name of the extension (default is
	\texttt{.bak}), and how many different backups are made -- more on this in
	\cref{sec:defuseloc}, and in particular see \texttt{backupExtension} and
	\texttt{onlyOneBackUp}.

	Note that if \texttt{latexindent.pl} can not create the backup, then it will exit without
	touching your original file; an error message will be given asking you to check the
	permissions of the backup file.

\flagbox{-o=output.tex,--outputfile=output.tex}
	\index{switches!-o, --output definition and details}
	\begin{commandshell} 
latexindent.pl -o=output.tex myfile.tex
latexindent.pl myfile.tex -o=output.tex 
latexindent.pl --outputfile=output.tex myfile.tex
latexindent.pl --outputfile output.tex myfile.tex
\end{commandshell}

	This will indent \texttt{myfile.tex} and output it to \texttt{output.tex}, overwriting it
	(\texttt{output.tex}) if it already exists\footnote{Users of version 2.* should note the
		subtle change in syntax}. Note that if \texttt{latexindent.pl} is called with both the
	\texttt{-w} and \texttt{-o} switches, then \texttt{-w} will be ignored and \texttt{-o}
	will take priority (this seems safer than the other way round).

	Note that using \texttt{-o} as above is equivalent to using
	\begin{commandshell}
latexindent.pl myfile.tex > output.tex
\end{commandshell}

	You can call the \texttt{-o} switch with the name of the output file \emph{without} an
	extension; in%
	\announce{2017-06-25}{upgrade to -o switch}
	this case, \texttt{latexindent.pl} will use the extension from the original file. For
	example, the following two calls to \texttt{latexindent.pl} are equivalent:
	\begin{commandshell}
latexindent.pl myfile.tex -o=output
latexindent.pl myfile.tex -o=output.tex
\end{commandshell}

	You can call the \texttt{-o} switch using a \texttt{+} symbol at the beginning; this
	will%
	\announce{2017-06-25}{+ sign in o switch}
	concatenate the name of the input file and the text given to the \texttt{-o} switch. For
	example, the following two calls to \texttt{latexindent.pl} are equivalent:
	\begin{commandshell}
latexindent.pl myfile.tex -o=+new
latexindent.pl myfile.tex -o=myfilenew.tex
\end{commandshell}

	You can call the \texttt{-o} switch using a \texttt{++} symbol at the end of the
	name%
	\announce{2017-06-25}{++ in o switch} of your output
	file; this tells \texttt{latexindent.pl} to search successively for the name of your
	output file concatenated with $0, 1, \ldots$ while the name of the output file exists.
	For example,
	\begin{commandshell}
latexindent.pl myfile.tex -o=output++
\end{commandshell}
	tells \texttt{latexindent.pl} to output to \texttt{output0.tex}, but if it exists then
	output to \texttt{output1.tex}, and so on.

	Calling \texttt{latexindent.pl} with simply
	\begin{commandshell}
latexindent.pl myfile.tex -o=++
\end{commandshell}
	tells it to output to \texttt{myfile0.tex}, but if it exists then output to
	\texttt{myfile1.tex} and so on.

	The \texttt{+} and \texttt{++} feature of the \texttt{-o} switch can be combined; for
	example, calling
	\begin{commandshell}
latexindent.pl myfile.tex -o=+out++
\end{commandshell}
	tells \texttt{latexindent.pl} to output to \texttt{myfileout0.tex}, but if it exists,
	then try \texttt{myfileout1.tex}, and so on.

	There is no need to specify a file extension when using the \texttt{++} feature, but if
	you wish to, then you should include it \emph{after} the \texttt{++} symbols, for example
	\begin{commandshell}
latexindent.pl myfile.tex -o=+out++.tex
\end{commandshell}

	See \vref{app:differences} for details of how the interface has changed from Version 2.2
	to Version 3.0 for this flag.
\flagbox{-s, --silent}
	\index{switches!-s, --silent definition and details}
	\begin{commandshell}
latexindent.pl -s myfile.tex
latexindent.pl myfile.tex -s
\end{commandshell}

	Silent mode: no output will be given to the terminal.

\flagbox{-t, --trace}
	\index{switches!-t, --trace definition and details}
	\begin{commandshell}
latexindent.pl -t myfile.tex
latexindent.pl myfile.tex -t
\end{commandshell}

	\label{page:traceswitch}
	Tracing mode: verbose output will be given to \texttt{indent.log}. This is useful if
	\texttt{latexindent.pl} has made a mistake and you're trying to find out where and why.
	You might also be interested in learning about \texttt{latexindent.pl}'s thought process
	-- if so, this switch is for you, although it should be noted that, especially for large
	files, this does affect performance of the script.

\flagbox{-tt, --ttrace}
	\index{switches!-tt, --ttrace definition and details}
	\begin{commandshell}
latexindent.pl -tt myfile.tex
latexindent.pl myfile.tex -tt
\end{commandshell}

	\emph{More detailed} tracing mode: this option gives more details to
	\texttt{indent.log}
	than the standard \texttt{trace} option (note that, even more so than with \texttt{-t},
	especially for large files, performance of the script will be affected).

\flagbox{-l, --local[=myyaml.yaml,other.yaml,...]}
	\index{switches!-l, --local definition and details}
	\begin{commandshell}
latexindent.pl -l myfile.tex
latexindent.pl -l=myyaml.yaml myfile.tex
latexindent.pl -l myyaml.yaml myfile.tex
latexindent.pl -l first.yaml,second.yaml,third.yaml myfile.tex
latexindent.pl -l=first.yaml,second.yaml,third.yaml myfile.tex
latexindent.pl myfile.tex -l=first.yaml,second.yaml,third.yaml 
\end{commandshell}

	\label{page:localswitch}
	\texttt{latexindent.pl} will always load \texttt{defaultSettings.yaml} (rhymes with
	camel) and if it is called with the \texttt{-l} switch and it finds
	\texttt{localSettings.yaml} in the same directory as \texttt{myfile.tex}, then, if not
	found, it looks for \texttt{localSettings.yaml} (and friends, see
	\vref{sec:localsettings}) in the current working directory, then
	these%
	\announce{2021-03-14}*{-l switch: localSettings and
		friends} settings will be added to the indentation scheme. Information will be given in
	\texttt{indent.log} on the success or failure of loading \texttt{localSettings.yaml}.

	The \texttt{-l} flag can take an \emph{optional} parameter which details the name (or
	names separated by commas) of a YAML file(s) that resides in the same directory as
	\texttt{myfile.tex}; you can use this option if you would like to load a settings file in
	the current working directory that is \emph{not} called
	\texttt{localSettings.yaml}.%
	\announce{2017-08-21}*{-l
		switch absolute paths} In fact, you can specify both \emph{relative} and \emph{absolute
		paths} for your YAML files; for example
	\begin{commandshell}
latexindent.pl -l=../../myyaml.yaml myfile.tex
latexindent.pl -l=/home/cmhughes/Desktop/myyaml.yaml myfile.tex
latexindent.pl -l=C:\Users\cmhughes\Desktop\myyaml.yaml myfile.tex
\end{commandshell}
	You will find a lot of other explicit demonstrations of how to use the \texttt{-l} switch
	throughout this documentation,

	You can call the \texttt{-l} switch with a `+' symbol either before or after
	\announce{2017-06-25}{+ sign with -l switch} another YAML file; for example:
	\begin{commandshell}
latexindent.pl -l=+myyaml.yaml myfile.tex
latexindent.pl -l "+ myyaml.yaml" myfile.tex
latexindent.pl -l=myyaml.yaml+  myfile.tex
\end{commandshell}
	which translate, respectively, to
	\begin{commandshell}
latexindent.pl -l=localSettings.yaml,myyaml.yaml myfile.tex
latexindent.pl -l=localSettings.yaml,myyaml.yaml myfile.tex
latexindent.pl -l=myyaml.yaml,localSettings.yaml myfile.tex
\end{commandshell}
	Note that the following is \emph{not} allowed:
	\begin{commandshell}
latexindent.pl -l+myyaml.yaml myfile.tex
\end{commandshell}
	and
	\begin{commandshell}
latexindent.pl -l + myyaml.yaml myfile.tex
\end{commandshell}
	will \emph{only} load \texttt{localSettings.yaml}, and \texttt{myyaml.yaml} will be
	ignored. If you wish to use spaces between any of the YAML settings, then you must wrap
	the entire list of YAML files in quotes, as demonstrated above.

	You may also choose to omit the \texttt{yaml} extension, such
	as%
	\announce{2017-06-25}{no extension for -l switch}
	\begin{commandshell}
latexindent.pl -l=localSettings,myyaml myfile.tex
\end{commandshell}
\flagbox{-y, --yaml=yaml settings}
	\index{switches!-y, --yaml definition and details}
	\index{indentation!default}
	\index{indentation!defaultIndent using -y switch}
	\begin{commandshell}
latexindent.pl myfile.tex -y="defaultIndent: ' '"
latexindent.pl myfile.tex -y="defaultIndent: ' ',maximumIndentation:' '"
latexindent.pl myfile.tex -y="indentRules: one: '\t\t\t\t'"
latexindent.pl myfile.tex -y='modifyLineBreaks:environments:EndStartsOnOwnLine:3' -m
latexindent.pl myfile.tex -y='modifyLineBreaks:environments:one:EndStartsOnOwnLine:3' -m
\end{commandshell}
	\label{page:yamlswitch}You%
	\announce{2017-08-21}{the -y switch} can specify YAML settings from the command line
	using the \texttt{-y} or \texttt{--yaml} switch; sample demonstrations are given above.
	Note, in particular, that multiple settings can be specified by separating them via
	commas. There is a further option to use a \texttt{;} to separate fields, which is
	demonstrated in \vref{sec:yamlswitch}.

	Any settings specified via this switch will be loaded \emph{after} any specified using
	the \texttt{-l} switch. This is discussed further in \vref{sec:loadorder}.
\flagbox{-d, --onlydefault}
	\index{switches!-d, --onlydefault definition and details}
	\begin{commandshell}
latexindent.pl -d myfile.tex
\end{commandshell}

	Only \texttt{defaultSettings.yaml}: you might like to read \cref{sec:defuseloc} before
	using this switch. By default, \texttt{latexindent.pl} will always search for
	\texttt{indentconfig.yaml} or \texttt{.indentconfig.yaml} in your home directory. If you
	would prefer it not to do so then (instead of deleting or renaming
	\texttt{indentconfig.yaml} or \texttt{.indentconfig.yaml}) you can simply call the script
	with the \texttt{-d} switch; note that this will also tell the script to ignore
	\texttt{localSettings.yaml} even if it has been called with the \texttt{-l} switch;
	\texttt{latexindent.pl}%
	\announce{2017-08-21}*{updated -d switch} will also ignore any settings specified from the \texttt{-y} switch.

\flagbox{-c, --cruft=<directory>}
	\index{switches!-c, --cruft definition and details}
	\begin{commandshell}
latexindent.pl -c=/path/to/directory/ myfile.tex
\end{commandshell}

	If you wish to have backup files and \texttt{indent.log} written to a directory other
	than the current working directory, then you can send these `cruft' files to another
	directory. Note the use of a trailing forward slash. % this switch was made as a result of http://tex.stackexchange.com/questions/142652/output-latexindent-auxiliary-files-to-a-different-directory

\flagbox{-g, --logfile=<name of log file>}
	\index{switches!-g, --logfile definition and details}
	\begin{commandshell}
latexindent.pl -g=other.log myfile.tex
latexindent.pl -g other.log myfile.tex
latexindent.pl --logfile other.log myfile.tex
latexindent.pl myfile.tex -g other.log 
\end{commandshell}

	By default, \texttt{latexindent.pl} reports information to \texttt{indent.log}, but if
	you wish to change the name of this file, simply call the script with your chosen name
	after the \texttt{-g} switch as demonstrated above.

	\announce{2021-05-07}{log file creation updated} If \texttt{latexindent.pl} can not open
	the log file that you specify, then the script will operate, and no log file will be
	produced; this might be helpful to users who wish to specify the following, for example
	\begin{commandshell}
latexindent.pl -g /dev/null myfile.tex
\end{commandshell}

\flagbox{-sl, --screenlog}
	\index{switches!-sl, --screenlog definition and details}
	\begin{commandshell}
latexindent.pl -sl myfile.tex
latexindent.pl -screenlog myfile.tex
\end{commandshell}
	Using this%
	\announce{2018-01-13}{screenlog switch created} option tells \texttt{latexindent.pl} to output the log file to the screen, as
	well as to your chosen log file.

\flagbox{-m, --modifylinebreaks}
	\index{switches!-m, --modifylinebreaks definition and details}
	\begin{commandshell}
latexindent.pl -m myfile.tex
latexindent.pl -modifylinebreaks myfile.tex
\end{commandshell}

	One of the most exciting developments in Version~3.0 is the ability to modify line
	breaks; for full details see \vref{sec:modifylinebreaks}

	\texttt{latexindent.pl} can also be called on a file without the file extension, for
	example
	\begin{commandshell}
latexindent.pl myfile
\end{commandshell}
	and in which case, you can specify the order in which extensions are searched for; see
	\vref{lst:fileExtensionPreference} for full details.
\flagbox{STDIN}
	\begin{commandshell}
cat myfile.tex | latexindent.pl
cat myfile.tex | latexindent.pl -
\end{commandshell}
	\texttt{latexindent.pl} will%
	\announce{2018-01-13}{STDIN allowed} allow input from STDIN, which means that you can
	pipe output from other commands directly into the script. For example assuming that you
	have content in \texttt{myfile.tex}, then the above command will output the results of
	operating upon \texttt{myfile.tex}.

	If you wish to use this feature with your own local settings, via the \texttt{-l} switch,
	then you should finish your call to \texttt{latexindent.pl} with a \texttt{-} sign:
	\begin{commandshell}
cat myfile.tex | latexindent.pl -l=mysettings.yaml -
\end{commandshell}

	Similarly, if you%
	\announce{2018-01-13}*{no options/filename updated} simply type \texttt{latexindent.pl} at the command line, then
	it will expect (STDIN) input from the command line.
	\begin{commandshell}
latexindent.pl
\end{commandshell}

	Once you have finished typing your input, you can press
	\begin{itemize}
		\item \texttt{CTRL+D} on Linux
		\item \texttt{CTRL+Z} followed by \texttt{ENTER} on Windows
	\end{itemize}
	to signify that your input has finished. Thanks to \cite{xu-cheng} for an update to this
	feature.
\flagbox{-r, --replacement}
	\index{switches!-r, --replacement definition and details}
	\begin{commandshell}
latexindent.pl -r myfile.tex
latexindent.pl -replacement myfile.tex
\end{commandshell}
	You can%
	\announce{2019-07-13}{replacement mode switch}
	call \texttt{latexindent.pl} with the \texttt{-r} switch to instruct it to perform
	replacements/substitutions on your file; full details and examples are given in
	\vref{sec:replacements}.
	\index{verbatim!rv, replacementrespectverb switch}

\flagbox{-rv, --replacementrespectverb}
	\index{switches!-rv, --replacementrespectverb definition and details}
	\begin{commandshell}
latexindent.pl -rv myfile.tex
latexindent.pl -replacementrespectverb myfile.tex
\end{commandshell}
	You can%
	\announce{2019-07-13}{replacement mode switch, respecting verbatim} instruct \texttt{latexindent.pl} to perform
	replacements/substitutions by using the \texttt{-rv} switch, but will \emph{respect
		verbatim code blocks}; full details and examples are given in \vref{sec:replacements}.

\flagbox{-rr, --onlyreplacement}
	\index{switches!-rr, --onlyreplacement definition and details}
	\begin{commandshell}
latexindent.pl -rr myfile.tex
latexindent.pl -onlyreplacement myfile.tex
\end{commandshell}
	You can%
	\announce{2019-07-13}{replacement (only) mode switch} instruct \texttt{latexindent.pl} to skip all of its other indentation operations
	and \emph{only} perform replacements/substitutions by using the \texttt{-rr} switch; full
	details and examples are given in \vref{sec:replacements}.

\flagbox{-k, --check}
	\index{switches!-k, --check definition and details}
	\begin{commandshell}
latexindent.pl -k myfile.tex
latexindent.pl -check myfile.tex
\end{commandshell}
	You can%
	\announce{2021-09-16}{-k,-check switch} instruct
	\texttt{latexindent.pl} to check if the text after indentation matches that given in the
	original file.

	The \texttt{exit} code
	\index{exit code} of \texttt{latexindent.pl} is 0 by default. If
	you use the \texttt{-k} switch then
	\begin{itemize}
		\item if the text after indentation matches that given in the original file, then the exit code
		      is 0;
		\item if the text after indentation does \emph{not} match that given in the original file, then
		      the exit code is 1.
	\end{itemize}

	The value of the exit code may be important to those wishing to, for example, check the
	status of the indentation in continuous integration tools such as GitHub Actions. Full
	details of the exit codes of \texttt{latexindent.pl} are given in \cref{tab:exit-codes}.

	A simple \texttt{diff} will be given in \texttt{indent.log}.

\flagbox{-kv, --checkv}
	\index{switches!-kv, --checkv definition and details}
	\begin{commandshell}
latexindent.pl -kv myfile.tex
latexindent.pl -checkv myfile.tex
\end{commandshell}
	\announce{2021-09-16}{-kv, -checkv: check verbose switch} The \texttt{check verbose}
	switch is exactly the same as the \texttt{-k} switch, except that it is \emph{verbose},
	and it will output the (simple) diff to the terminal, as well as to \texttt{indent.log}.

\flagbox{-n, --lines=MIN-MAX}
	\index{switches!-n, --lines definition and details}
	\begin{commandshell}
latexindent.pl -n 5-8 myfile.tex
latexindent.pl -lines 5-8 myfile.tex
\end{commandshell}
	\announce{2021-09-16}{-n, -lines switch} The \texttt{lines} switch instructs
	\texttt{latexindent.pl} to operate only on specific line ranges within
	\texttt{myfile.tex}.

	Complete demonstrations are given in \cref{sec:line-switch}.

\subsection{From arara}\label{sec:arara}
	Using \texttt{latexindent.pl} from the command line is fine for some folks, but others
	may find it easier to use from \texttt{arara}; you can find the arara rule for
	\texttt{latexindent.pl} and its associated documentation at \cite{paulo}.

\subsection{Summary of exit codes}
	\index{exit code!summary}
	Assuming that you call \texttt{latexindent.pl} on \texttt{myfile.tex}
	\begin{commandshell}
latexindent.pl myfile.tex
\end{commandshell}
	then \texttt{latexindent.pl} can exit with the exit codes given in \cref{tab:exit-codes}.

	\begin{table}[!htb]
		\caption{Exit codes for \texttt{latexindent.pl}}
		\label{tab:exit-codes}
		\begin{tabular}{ccl}
			\toprule
			exit code & indentation & status                                                                                        \\
			\midrule
			0         & \faCheck    & success; if \texttt{-k} or \texttt{-kv} active, indented text matches original                \\
			0         & \faClose    & success; if \texttt{-version} or \texttt{-help}, no indentation performed                     \\
			1         & \faCheck    & success, and \texttt{-k} or \texttt{-kv} active; indented text \emph{different} from original \\
			\midrule
			2         & \faClose    & failure, \texttt{defaultSettings.yaml} could not be read                                      \\
			3         & \faClose    & failure, myfile.tex not found                                                                 \\
			4         & \faClose    & failure, myfile.tex exists but cannot be read                                                 \\
			5         & \faClose    & failure, \texttt{-w} active, and back-up file cannot be written                               \\
			6         & \faClose    & failure, \texttt{-c} active, and cruft directory does not exist                               \\
			\bottomrule
		\end{tabular}
	\end{table}

%% arara: pdflatex: {shell: yes, files: [latexindent]}
\section{default, user, and local settings}\label{sec:defuseloc}
 \texttt{latexindent.pl} loads its settings from \texttt{defaultSettings.yaml}
 (rhymes with camel). The idea is to separate the behaviour of the script
 from the internal working -- this is very similar to the way that we separate content
 from form when writing our documents in \LaTeX.

\subsection{defaultSettings.yaml}
	If you look in \texttt{defaultSettings.yaml} you'll find the switches
	that govern the behaviour of \texttt{latexindent.pl}. If you're not sure where
	\texttt{defaultSettings.yaml} resides on your computer, don't worry as \texttt{indent.log}
	will tell you where to find it.
	\texttt{defaultSettings.yaml} is commented,
	but here is a description of what each switch is designed to do. The default
	value is given in each case; whenever you see \emph{integer} in \emph{this}
	section, assume that it must be greater than or equal to \texttt{0} unless
	otherwise stated.

	You can certainly feel free to edit \texttt{defaultSettings.yaml}, but
	this is not ideal as it may be overwritten when you update your \TeX{} distribution --
	all of your hard work tweaking the script would be undone! Don't worry,
	there's a solution, feel free to peek ahead to \cref{sec:indentconfig} if you like.

\yamltitle{fileExtensionPreference}*{fields}
	\texttt{latexindent.pl} can be called to
	act on a file without
	specifying the file extension.  For example we can call
	\begin{commandshell}
latexindent.pl myfile
\end{commandshell}
	\begin{wrapfigure}[8]{r}[0pt]{6cm}
		\cmhlistingsfromfile[firstnumber=22,linerange={22-26},style=yaml-LST]{../defaultSettings.yaml}[width=.8\linewidth,before=\centering,yaml-TCB]{\texttt{fileExtensionPreference}}{lst:fileExtensionPreference}
	\end{wrapfigure}

	in which case the script will look for \texttt{myfile} with the extensions
	specified in \texttt{fileExtensionPreference} in their numeric order. If
	no match is found, the script will exit. As with all of the fields, you should
	change and/or add to this as necessary.

	Calling \texttt{latexindent.pl myfile} with the (default) settings specified in \cref{lst:fileExtensionPreference}
	means that the script will first look for \texttt{myfile.tex}, then \texttt{myfile.sty}, \texttt{myfile.cls},
	and finally \texttt{myfile.bib} in order.

\yamltitle{backupExtension}*{extension name}

	If you call \texttt{latexindent.pl} with the \texttt{-w} switch (to overwrite
	\texttt{myfile.tex}) then it will create a backup file before doing
	any indentation; the default extension is \texttt{.bak}, so, for example, \texttt{myfile.bak0}
	would be created when calling \texttt{latexindent.pl myfile.tex}.

	By default, every time you subsequently call \texttt{latexindent.pl} with
	the \texttt{-w} to act upon \texttt{myfile.tex}, it will create successive back up files: \texttt{myfile.bak1}, \texttt{myfile.bak2},
	etc.

\yamltitle{onlyOneBackUp}*{integer}
	\label{page:onlyonebackup}
	If you don't want a backup for every time that you call \texttt{latexindent.pl} (so
	you don't want \texttt{myfile.bak1}, \texttt{myfile.bak2}, etc) and you simply
	want \texttt{myfile.bak} (or whatever you chose \texttt{backupExtension} to be)
	then change \texttt{onlyOneBackUp} to \texttt{1}; the default value of
	\texttt{onlyOneBackUp} is \texttt{0}.

\yamltitle{maxNumberOfBackUps}*{integer}
	Some users may only want a finite number of backup files,
	say at most $3$, in which case, they can change this switch.
	The smallest value of \texttt{maxNumberOfBackUps} is $0$ which will \emph{not}
	prevent backup files being made; in this case, the behaviour will be dictated
	entirely by \texttt{onlyOneBackUp}. The default value of \texttt{maxNumberOfBackUps}
	is \texttt{0}.

\yamltitle{cycleThroughBackUps}*{integer}
	Some users may wish to cycle through backup files, by deleting the
	oldest backup file and keeping only the most recent; for example,
	with \texttt{maxNumberOfBackUps: 4}, and \texttt{cycleThroughBackUps}
	set to \texttt{1}  then the \texttt{copy} procedure given below
	would be obeyed.

	\begin{commandshell}
copy myfile.bak1 to myfile.bak0
copy myfile.bak2 to myfile.bak1
copy myfile.bak3 to myfile.bak2
copy myfile.bak4 to myfile.bak3
	\end{commandshell}
	The default value of \texttt{cycleThroughBackUps} is \texttt{0}.

\yamltitle{verbatimEnvironments}*{fields}

	\begin{wrapfigure}[14]{r}[0pt]{6cm}
		\cmhlistingsfromfile[firstnumber=64,linerange={64-66},style=yaml-LST]{../defaultSettings.yaml}[width=.8\linewidth,before=\centering,yaml-TCB]{\texttt{verbatimEnvironments}}{lst:verbatimEnvironments}

		\vspace{.2cm}
		\cmhlistingsfromfile[firstnumber=69,linerange={69-71},style=yaml-LST]{../defaultSettings.yaml}[width=.8\linewidth,before=\centering,yaml-TCB]{\texttt{verbatimCommands}}{lst:verbatimCommands}
	\end{wrapfigure}
	A field that contains a list of environments
	that you would like left completely alone -- no indentation will be performed
	on environments that you have specified in this field, see \cref{lst:verbatimEnvironments}.

	Note that if  you put an environment in \texttt{verbatimEnvironments}
	and in other fields such as \texttt{lookForAlignDelims} or \texttt{noAdditionalIndent}
	then \texttt{latexindent.pl} will \emph{always} prioritize \texttt{verbatimEnvironments}.

\yamltitle{verbatimCommands}*{fields}
	A field that contains a list of commands that are verbatim commands, for example
	\lstinline|\lstinline|; any commands populated in this field are protected from line breaking
	routines (only relevant if the \texttt{-m} is active, see \vref{sec:m-switch}).

\yamltitle{noIndentBlock}*{fields}

	\begin{wrapfigure}[8]{r}[0pt]{6cm}
		\cmhlistingsfromfile[firstnumber=77,linerange={77-79},style=yaml-LST]{../defaultSettings.yaml}[width=.8\linewidth,before=\centering,yaml-TCB]{\texttt{noIndentBlock}}{lst:noIndentBlock}
	\end{wrapfigure}
	If you have a block of code that you don't want \texttt{latexindent.pl} to touch (even if it is \emph{not} a verbatim-like
	environment) then you can wrap it in an environment from \texttt{noIndentBlock};
	you can use any name you like for this, provided you populate it as demonstrate in
	\cref{lst:noIndentBlock}.

	Of course, you don't want to have to specify these as null environments
	in your code, so you use them with a comment symbol, \lstinline!%!, followed
by as many spaces (possibly none) as you like; see \cref{lst:noIndentBlockdemo} for
example.

\begin{cmhlistings}[style=demo,escapeinside={(*@}{@*)}]{\texttt{noIndentBlock} demonstration}{lst:noIndentBlockdemo}
%(*@@*) \begin{noindent}
        this code
                won't
     be touched
                    by
             latexindent.pl!
%(*@@*)\end{noindent}
	\end{cmhlistings}

    \yamltitle{removeTrailingWhitespace}*{fields}\label{yaml:removeTrailingWhitespace}

\begin{wrapfigure}[12]{r}[0pt]{6cm}
\cmhlistingsfromfile[firstnumber=82,linerange={82-84},style=yaml-LST]{../defaultSettings.yaml}[width=.8\linewidth,before=\centering,yaml-TCB]{removeTrailingWhitespace}{lst:removeTrailingWhitespace}

\vspace{.2cm}
\cmhlistingsfromfile[firstnumber=88,linerange={88-90},style=yaml-LST]{../defaultSettings.yaml}[width=.8\linewidth,before=\centering,yaml-TCB]{\texttt{fileContentsEnvironments}}{lst:fileContentsEnvironments}
\end{wrapfigure}
Trailing white space can be removed both \emph{before} and \emph{after} processing 
the document, as detailed in \cref{lst:removeTrailingWhitespace}; each of the fields 
can take the values \texttt{0} or \texttt{1}. See \vref{lst:removeTWS-before,lst:env-mlb5-modAll,lst:env-mlb5-modAll-remove-WS} 
for before and after results.  Thanks to \cite{vosskuhle} for providing this feature.

\yamltitle{fileContentsEnvironments}*{field}

Before \texttt{latexindent.pl} determines the difference between preamble (if any) and the main document,
it first searches for any of the environments specified in \texttt{fileContentsEnvironments}, see
\cref{lst:fileContentsEnvironments}.
  The behaviour of \texttt{latexindent.pl} on these environments is determined by their location (preamble or not), and 
  the value \texttt{indentPreamble}, discussed next.

\yamltitle{indentPreamble}{0|1}

The preamble of a document can sometimes contain some trickier code
for \texttt{latexindent.pl} to operate upon. By default, \texttt{latexindent.pl}
won't try to operate on the preamble (as \texttt{indentPreamble} is set to \texttt{0},
by default), but if you'd like \texttt{latexindent.pl} to try then change \texttt{indentPreamble} to \texttt{1}.

\yamltitle{lookForPreamble}*{fields}

\begin{wrapfigure}[8]{r}[0pt]{5cm}
\cmhlistingsfromfile[firstnumber=96,linerange={96-100},style=yaml-LST]{../defaultSettings.yaml}[width=.8\linewidth,before=\centering,yaml-TCB]{lookForPreamble}{lst:lookForPreamble}
\end{wrapfigure}
Not all files contain preamble; for example, \texttt{sty}, \texttt{cls} and \texttt{bib} files typically do \emph{not}. Referencing
\cref{lst:lookForPreamble}, if you set, for example, \texttt{.tex} to \texttt{0}, then regardless of the setting of the value of \texttt{indentPreamble}, preamble
will not be assumed when operating upon \texttt{.tex} files.
\yamltitle{preambleCommandsBeforeEnvironments}{0|1}
Assuming that \texttt{latexindent.pl} is asked to operate upon the preamble of a document,
when this switch is set to \texttt{0} then environment code blocks will be sought first, 
and then command code blocks. When this switch is set to \texttt{1}, commands 
will be sought first. The example that first motivated this switch contained the code given in \cref{lst:motivatepreambleCommandsBeforeEnvironments}.

\begin{cmhlistings}{Motivating \texttt{preambleCommandsBeforeEnvironments}}{lst:motivatepreambleCommandsBeforeEnvironments}
...
preheadhook={\begin{mdframed}[style=myframedstyle]},
postfoothook=\end{mdframed},
...
\end{cmhlistings}

\yamltitle{defaultIndent}*{horizontal space}
This is the default indentation (\lstinline!\t! means a tab, and is the default value) used in the absence of other details
	for the command or environment we are working with; see \texttt{indentRules}
	for more details (\cpageref{page:indentRules}).

	If you're interested in experimenting with \texttt{latexindent.pl} then you
	can \emph{remove} all indentation by setting \texttt{defaultIndent: ""}



\yamltitle{lookForAlignDelims}*{fields}
	\begin{wrapfigure}[12]{r}[0pt]{5cm}
		\begin{yaml}[numbers=none]{\texttt{lookForAlignDelims} (basic)}[width=.8\linewidth,before=\centering]{lst:aligndelims:basic}
lookForAlignDelims:
   tabular: 1
   tabularx: 1
   longtable: 1
   array: 1
   matrix: 1
   ...
	\end{yaml}
	\end{wrapfigure}
	This contains a list of environments and/or commands that
	are operated upon in a special way by \texttt{latexindent.pl} (see \cref{lst:aligndelims:basic}).
	In fact, the fields in \texttt{lookForAlignDelims} can actually
	take two different forms: the \emph{basic} version is shown in \cref{lst:aligndelims:basic}
	and the \emph{advanced} version in \cref{lst:aligndelims:advanced}; we will discuss each in turn.

	The environments specified in this field will be operated on in a special way  by \texttt{latexindent.pl}. In particular, it will try and align each column by its alignment
	tabs. It does have some limitations (discussed further in \cref{sec:knownlimitations}),
	but in many cases it will produce results such as those in \cref{lst:tabularbefore:basic,lst:tabularafter:basic}.

	If you find that \texttt{latexindent.pl} does not perform satisfactorily on such
	environments then you can set the relevant key to \texttt{0}, for example \texttt{tabular: 0}; alternatively, if you just want to ignore \emph{specific}
	instances of the environment, you could wrap them in something from \texttt{noIndentBlock} (see \cref{lst:noIndentBlock}).

	\begin{minipage}{.45\textwidth}
		\cmhlistingsfromfile[columns=fixed]{demonstrations/tabular1.tex}{\texttt{tabular} before}{lst:tabularbefore:basic}
	\end{minipage}%
	\hfill
	\begin{minipage}{.45\textwidth}
		\cmhlistingsfromfile[columns=fixed]{demonstrations/tabular1-default.tex}{\texttt{tabular} after (basic)}{lst:tabularafter:basic}
	\end{minipage}%

	If you wish to remove the alignment of the \lstinline!\\! within a delimiter-aligned block, then the
	advanced form of \texttt{lookForAlignDelims} shown in \cref{lst:aligndelims:advanced} is for you.

	\cmhlistingsfromfile[firstnumber=114,linerange={114-121},style=yaml-LST]{../defaultSettings.yaml}[yaml-TCB]{\texttt{lookForAlignDelims} (advanced)}{lst:aligndelims:advanced}

	Note that you can use a mixture of the basic and advanced form: in \cref{lst:aligndelims:advanced} \texttt{tabular} and \texttt{tabularx}
	are advanced and \texttt{longtable} is basic. When using the advanced form, each field should receive at least 1 sub-field, and \emph{can} (but does not have to) receive up to 3 fields:
	\begin{itemize}
		\item \texttt{delims}: switch equivalent to simply specifying, for example, \texttt{tabular: 1} in
		      the basic version shown in \cref{lst:aligndelims:basic} (default: 1);
		\item \texttt{alignDoubleBackSlash}: switch to determine if \lstinline!\\! should be aligned (default: 1);
		\item \texttt{spacesBeforeDoubleBackSlash}: optionally, specifies the number of spaces to be inserted
		      before (non-aligned) \lstinline!\\!. In order to use this field, \texttt{alignDoubleBackSlash} needs
		      to be set to 0 (default: 0).
	\end{itemize}

	Assuming that you have the settings in \cref{lst:aligndelims:advanced} saved in \texttt{mysettings.yaml}, and the code
	from \cref{lst:tabularbefore:basic} in \texttt{myfile.tex} and you run
	\begin{commandshell}
latexindent.pl -l mysettings.yaml myfile.tex 
\end{commandshell}
	then you should receive the before-and-after results shown in
	\cref{lst:tabularbefore:advanced,lst:tabularafter:advanced}; note that the ampersands have been aligned, but
	the \lstinline!\\! have not (compare the alignment of \lstinline!\\! in \cref{lst:tabularafter:basic,lst:tabularafter:advanced}).

	\begin{minipage}{.45\textwidth}
		\cmhlistingsfromfile[columns=fixed]{demonstrations/tabular1.tex}{\texttt{tabular} before }{lst:tabularbefore:advanced}
	\end{minipage}%
	\hfill
	\begin{minipage}{.45\textwidth}
		\cmhlistingsfromfile[columns=fixed]{demonstrations/tabular1-advanced.tex}{\texttt{tabular} after (advanced)}{lst:tabularafter:advanced}
	\end{minipage}%

	Using  \texttt{spacesBeforeDoubleBackSlash: 3} gives \cref{lst:tabularbefore:spacing,lst:tabularafter:spacing},
	note the spacing before the \lstinline!\\! in \cref{lst:tabularafter:spacing}.

	\begin{minipage}{.45\textwidth}
		\cmhlistingsfromfile[columns=fixed]{demonstrations/tabular1.tex}{\texttt{tabular} before}{lst:tabularbefore:spacing}
	\end{minipage}%
	\hfill
	\begin{minipage}{.45\textwidth}
		\cmhlistingsfromfile[columns=fixed]{demonstrations/tabular1-advanced-3spaces.tex}{\texttt{tabular} after (spacing)}{lst:tabularafter:spacing}
	\end{minipage}%

	As of Version 3.0, the alignment routine works on mandatory and optional arguments within commands, and also within `special' code blocks
	(see \fixthis{need a reference to special section!}); for example, assuming that you have a command called \lstinline!\matrix!
	and that it is populated within \texttt{lookForAlignDelims} (which it is, by default), then the before-and-after results
	shown in \cref{lst:matrixbefore,lst:matrixafter} are achievable by default.

	\begin{minipage}{.45\textwidth}
		\cmhlistingsfromfile[columns=fixed]{demonstrations/matrix1.tex}{\texttt{matrix} before}{lst:matrixbefore}
	\end{minipage}%
	\hfill
	\begin{minipage}{.45\textwidth}
		\cmhlistingsfromfile[columns=fixed]{demonstrations/matrix1-default.tex}{\texttt{matrix} after}{lst:matrixafter}
	\end{minipage}%


	\fixthis{need to resolve alignment outside of environments}
	%If you have blocks of code that you wish to align at the \&  character that
	%are \emph{not} wrapped in, for example, \texttt{\begin{tabular}\ldots\end{tabular}}, then you use the mark up
	%  illustrated in \cref{lst:alignmentmarkup}. Note that the \texttt{%*} must be next to
	%  each other, but that there can be any number of spaces (possibly none) between the
	%  \texttt{*} and \texttt{\begin{tabular}}; note also that you may use any
	%    environment name that you have specified in \texttt{lookForAlignDelims}.
	%    \begin{cmhlistings}[style=demo,columns=fixed]{Mark up for aligning delimiters outside of environments}{lst:alignmentmarkup}
	%      \matrix{%
	%      %* \begin{tabular}
	%         1 & 2 & 3 & 4 \\
	%         5 &   & 6 &   \\
	%        %* \end{tabular}
	%        }

	\fixthis{final check: line numbers in yaml}

\yamltitle{indentAfterItems}*{fields}
	\begin{wrapfigure}[5]{r}[0pt]{7cm}
		\cmhlistingsfromfile[firstnumber=148,linerange={148-151},style=yaml-LST]{../defaultSettings.yaml}[width=.8\linewidth,before=\centering,yaml-TCB]{\texttt{indentAfterItems}}{lst:indentafteritems}
	\end{wrapfigure}
	The environments specified in \texttt{indentAfterItems}  tell
	\texttt{latexindent.pl} to look for \lstinline!\item! commands; if these switches are set to \texttt{1}
	then indentation will be performed so as indent the code after each \texttt{item}.
	A demonstration is given in \cref{lst:itemsbefore,lst:itemsafter}

	\begin{minipage}{.45\textwidth}
		\cmhlistingsfromfile{demonstrations/items1.tex}{\texttt{items} before}{lst:itemsbefore}
	\end{minipage}%
	\hfill
	\begin{minipage}{.45\textwidth}
		\cmhlistingsfromfile{demonstrations/items1-default.tex}{\texttt{items} after}{lst:itemsafter}
	\end{minipage}

\yamltitle{itemNames}*{fields}
	\begin{wrapfigure}[5]{r}[0pt]{5cm}
		\cmhlistingsfromfile[firstnumber=157,linerange={157-159},style=yaml-LST]{../defaultSettings.yaml}[width=.8\linewidth,before=\centering,yaml-TCB]{\texttt{itemNames}}{lst:itemNames}
	\end{wrapfigure}
	If you have your own \texttt{item} commands (perhaps you
	prefer to use \texttt{myitem}, for example)
	then you can put populate them in \texttt{itemNames}.
	For example, users of the \texttt{exam} document class might like to add
	\texttt{parts} to \texttt{indentAfterItems} and \texttt{part} to \texttt{itemNames}
	to their user settings--see \vref{sec:indentconfig} for details of how to configure user settings,
	and \vref{lst:mysettings} in particular.\label{page:examsettings}

\yamltitle{specialBeginEnd}*{fields}
	The fields specified in \texttt{specialBeginEnd} are, in their default state, focused on math mode begin and end statements, but
	there is no requirement for this to be the case; \cref{lst:specialBeginEnd} shows the
	default settings of \texttt{specialBeginEnd}.

	\cmhlistingsfromfile[firstnumber=163,linerange={163-175},style=yaml-LST]{../defaultSettings.yaml}[width=.8\linewidth,before=\centering,yaml-TCB]{\texttt{specialBeginEnd}}{lst:specialBeginEnd}

	The field \texttt{displayMath} represents \lstinline!\[...\]!, \texttt{inlineMath} represents
	\lstinline!$...$! and \texttt{displayMathTex} represents \lstinline!$$...$$!. You can, of course,
	rename these in your own YAML files (see \vref{sec:localsettings}); indeed, you
	might like to set up your own specil begin and end statements.

	A demonstration of the before-and-after results are shown in \cref{lst:specialbefore,lst:specialafter}.

	\begin{minipage}{.45\textwidth}
		\cmhlistingsfromfile{demonstrations/special1.tex}{\texttt{special1.tex} before}{lst:specialbefore}
	\end{minipage}%
	\hfill
	\begin{minipage}{.45\textwidth}
		\cmhlistingsfromfile{demonstrations/special1-default.tex}{\texttt{special1.tex} after}{lst:specialafter}
	\end{minipage}

	For each field, the \texttt{lookForThis} is set to \texttt{1} by default, which means that \texttt{latexindent.pl}
	will look for this pattern; you can tell \texttt{latexindent.pl} not to look for the pattern, by setting
	\texttt{lookForThis} to \texttt{0}.

\yamltitle{indentAfterHeadings}*{fields}
	\begin{wrapfigure}[17]{r}[0pt]{8cm}
		\cmhlistingsfromfile[firstnumber=185,linerange={185-194},style=yaml-LST]{../defaultSettings.yaml}[width=.8\linewidth,before=\centering,yaml-TCB]{\texttt{indentAfterHeadings}}{lst:indentAfterHeadings}
	\end{wrapfigure}
	This field enables the user to specify
	indentation rules that take effect after heading commands such as \lstinline!\part!, \lstinline!\chapter!,
	\lstinline!\section!, \lstinline!\subsection*!, or indeed any user-specified command written in this field.

	This field is slightly different from most
	of the fields that we have considered previously, because each element is
	itself a field which has two elements: \texttt{indent} and \texttt{level}. (Similar
	in structure to the advanced form of  \texttt{lookForAlignDelims} in \cref{lst:aligndelims:advanced}.)

	The default settings do \emph{not} place indentation after a heading, but you
	can easily switch them on by changing \texttt{indentAfterThisHeading: 0} to \texttt{indentAfterThisHeading: 1}.
	The \texttt{level} field tells \texttt{latexindent.pl} the hierarchy of the heading
	structure in your document. You might, for example, like to have both \texttt{section}
	and \texttt{subsection} set with \texttt{level: 3} because you do not want the indentation to go too deep.

	You can add any of your own custom heading commands to this field, specifying the \texttt{level}
	as appropriate.  You can also specify your own indentation in \texttt{indentRules} \fixthis{need a reference to indentRules};
	you will find the default \texttt{indentRules} contains \lstinline!chapter: " "! which
	tells \texttt{latexindent.pl} simply to use a space character after \texttt{\chapter} headings
	(once \texttt{indent} is set to \texttt{1} for \texttt{chapter}).

	For example, assuming that you have read \vref{sec:localsettings}, say that
	you have the code in \cref{lst:headings1} saved into \texttt{headings1.yaml},
	and that you have the text from \cref{lst:headings1} saved into \texttt{headings1.tex}.

	\begin{minipage}{.45\textwidth}
		\cmhlistingsfromfile[style=yaml-LST]{demonstrations/headings1.yaml}[yaml-TCB]{\texttt{headings1.yaml}}{lst:headings1}
	\end{minipage}%
	\hfill
	\begin{minipage}{.45\textwidth}
		\cmhlistingsfromfile{demonstrations/headings1.tex}{\texttt{headings1.tex}}{lst:headings1}
	\end{minipage}

	If you run the command
	\begin{commandshell}
latexindent.pl headings1.tex -l=headings1.yaml
\end{commandshell}
	then you should receive the output given in \cref{lst:headings1-mod1}.

	\begin{minipage}{.45\textwidth}
		\cmhlistingsfromfile{demonstrations/headings1-mod1.tex}{\texttt{headings1.tex} first modification}{lst:headings1-mod1}
	\end{minipage}%
	\hfill
	\begin{minipage}{.45\textwidth}
		\cmhlistingsfromfile{demonstrations/headings1-mod2.tex}{\texttt{headings1.tex} second modification}{lst:headings1-mod2}
	\end{minipage}

	Now say that you modify the \texttt{YAML} from \cref{lst:headings1} so that the \texttt{paragraph} \texttt{level} is \texttt{1}; after
	running
	\begin{commandshell}
latexindent.pl headings1.tex -l=headings1.yaml
\end{commandshell}
	you should now receive the code given in \cref{lst:headings1-mod2}; notice that
	the \texttt{paragraph} and \texttt{subsection} are at the same indentation level.

%% arara: pdflatex: {shell: yes, files: [latexindent]}
% arara: pdflatex: {shell: yes, files: [latexindent]}
\subsection{\texttt{noAdditionalIndent} and \texttt{indentRules}}\label{sec:noadd-indent-rules}
	\texttt{latexindent.pl} operates on files by looking for code blocks, as detailed in \vref{subsubsec:code-blocks};
	for each type of code block  in \vref{tab:code-blocks} (which we will call a \emph{$\langle$thing$\rangle$} in what follows)
	it searches YAML fields for information in the following order:
	\begin{enumerate}
		\item \texttt{noAdditionalIndent} for the \emph{name} of the current \emph{$\langle$thing$\rangle$};
		\item \texttt{indentRules} for the \emph{name} of the current \emph{$\langle$thing$\rangle$};
		\item \texttt{noAdditionalIndentGlobal} for the \emph{type} of the current \emph{$\langle$thing$\rangle$};
		\item \texttt{indentRulesGlobal} for the \emph{type} of the current \emph{$\langle$thing$\rangle$}.
	\end{enumerate}

	Using the above list, the first piece of information to be found will be used; failing that,
	the value of \texttt{defaultIndent} is used.
	If information is found in multiple fields, the first one according to the list above will be used; for example,
	if information is present in both \texttt{indentRules} and in \texttt{noAdditionalIndentGlobal}, then the information from \texttt{indentRules}
	takes priority.

	We now present details for the different type of code blocks known to \texttt{latexindent.pl}, as detailed in \vref{tab:code-blocks}; for
	reference, there follows a list of the code blocks covered.

	\startcontents[noAdditionalIndent]
	\printcontents[noAdditionalIndent]{}{0}{}

%% arara: pdflatex: { files: [latexindent]}
\subsubsection{Environments and their arguments}\label{subsubsec:env-and-their-args}
	There are a few different YAML switches governing the indentation of environments; let's
	start with the code shown in \cref{lst:myenvtex}.

	\cmhlistingsfromfile{demonstrations/myenvironment-simple.tex}{\texttt{myenv.tex}}{lst:myenvtex}

\yamltitle{noAdditionalIndent}*{fields}
	If we do not wish \texttt{myenv} to receive any additional indentation, we have a few
	choices available to us, as demonstrated in \cref{lst:myenv-noAdd1,lst:myenv-noAdd2}.

	\begin{minipage}{.45\textwidth}
		\cmhlistingsfromfile[style=yaml-LST]{demonstrations/myenv-noAdd1.yaml}[width=.8\linewidth,before=\centering,yaml-TCB]{\texttt{myenv-noAdd1.yaml}}{lst:myenv-noAdd1}
	\end{minipage}
	\hfill
	\begin{minipage}{.45\textwidth}
		\cmhlistingsfromfile[style=yaml-LST]{demonstrations/myenv-noAdd2.yaml}[width=.8\linewidth,before=\centering,yaml-TCB]{\texttt{myenv-noAdd2.yaml}}{lst:myenv-noAdd2}
	\end{minipage}

	On applying either of the following commands,
	\index{switches!-l demonstration}
	\begin{commandshell}
latexindent.pl myenv.tex -l myenv-noAdd1.yaml  
latexindent.pl myenv.tex -l myenv-noAdd2.yaml  
\end{commandshell}
	we obtain the output given in \cref{lst:myenv-output}; note in particular that the
	environment \texttt{myenv} has not received any \emph{additional} indentation, but that
	the \texttt{outer} environment \emph{has} still received indentation.

	\cmhlistingsfromfile{demonstrations/myenvironment-simple-noAdd-body1.tex}{\texttt{myenv.tex} output (using either \cref{lst:myenv-noAdd1} or \cref{lst:myenv-noAdd2})}{lst:myenv-output}

	Upon changing the YAML files to those shown in \cref{lst:myenv-noAdd3,lst:myenv-noAdd4},
	and running either
	\index{switches!-l demonstration}
	\begin{commandshell}
latexindent.pl myenv.tex -l myenv-noAdd3.yaml  
latexindent.pl myenv.tex -l myenv-noAdd4.yaml  
\end{commandshell}
	we obtain the output given in \cref{lst:myenv-output-4}.

	\begin{minipage}{.45\textwidth}
		\cmhlistingsfromfile[style=yaml-LST]{demonstrations/myenv-noAdd3.yaml}[width=.8\linewidth,before=\centering,yaml-TCB]{\texttt{myenv-noAdd3.yaml}}{lst:myenv-noAdd3}
	\end{minipage}
	\hfill
	\begin{minipage}{.45\textwidth}
		\cmhlistingsfromfile[style=yaml-LST]{demonstrations/myenv-noAdd4.yaml}[width=.8\linewidth,before=\centering,yaml-TCB]{\texttt{myenv-noAdd4.yaml}}{lst:myenv-noAdd4}
	\end{minipage}

	\cmhlistingsfromfile{demonstrations/myenvironment-simple-noAdd-body4.tex}{\texttt{myenv.tex output} (using either \cref{lst:myenv-noAdd3} or \cref{lst:myenv-noAdd4})}{lst:myenv-output-4}

	Let's now allow \texttt{myenv} to have some optional and mandatory arguments, as in
	\cref{lst:myenv-args}.

	\cmhlistingsfromfile{demonstrations/myenvironment-args.tex}{\texttt{myenv-args.tex}}{lst:myenv-args}

	Upon running
	\index{switches!-l demonstration}
	\begin{commandshell}
latexindent.pl -l=myenv-noAdd1.yaml myenv-args.tex  
\end{commandshell}
	we obtain the output shown in \cref{lst:myenv-args-noAdd1}; note that the optional
	argument, mandatory argument and body \emph{all} have received no additional indent. This
	is because, when \texttt{noAdditionalIndent} is specified in `scalar' form (as in
	\cref{lst:myenv-noAdd1}), then \emph{all} parts of the environment (body, optional and
	mandatory arguments) are assumed to want no additional indent.
	\cmhlistingsfromfile{demonstrations/myenvironment-args-noAdd-body1.tex}{\texttt{myenv-args.tex} using \cref{lst:myenv-noAdd1}}{lst:myenv-args-noAdd1}

	We may customise \texttt{noAdditionalIndent} for optional and mandatory arguments of the
	\texttt{myenv} environment, as shown in, for example,
	\cref{lst:myenv-noAdd5,lst:myenv-noAdd6}.

	\begin{minipage}{.49\textwidth}
		\cmhlistingsfromfile[style=yaml-LST]{demonstrations/myenv-noAdd5.yaml}[width=.8\linewidth,before=\centering,yaml-TCB]{\texttt{myenv-noAdd5.yaml}}{lst:myenv-noAdd5}
	\end{minipage}
	\hfill
	\begin{minipage}{.49\textwidth}
		\cmhlistingsfromfile[style=yaml-LST]{demonstrations/myenv-noAdd6.yaml}[width=.8\linewidth,before=\centering,yaml-TCB]{\texttt{myenv-noAdd6.yaml}}{lst:myenv-noAdd6}
	\end{minipage}

	Upon running
	\index{switches!-l demonstration}
	\begin{commandshell}
latexindent.pl myenv.tex -l myenv-noAdd5.yaml  
latexindent.pl myenv.tex -l myenv-noAdd6.yaml  
\end{commandshell}
	we obtain the respective outputs given in
	\cref{lst:myenv-args-noAdd5,lst:myenv-args-noAdd6}. Note that in
	\cref{lst:myenv-args-noAdd5} the text for the \emph{optional} argument has not received
	any additional indentation, and that in \cref{lst:myenv-args-noAdd6} the \emph{mandatory}
	argument has not received any additional indentation; in both cases, the \emph{body} has
	not received any additional indentation.

	\begin{minipage}{.45\textwidth}
		\cmhlistingsfromfile{demonstrations/myenvironment-args-noAdd5.tex}{\texttt{myenv-args.tex} using \cref{lst:myenv-noAdd5}}{lst:myenv-args-noAdd5}
	\end{minipage}
	\hfill
	\begin{minipage}{.45\textwidth}
		\cmhlistingsfromfile{demonstrations/myenvironment-args-noAdd6.tex}{\texttt{myenv-args.tex} using \cref{lst:myenv-noAdd6}}{lst:myenv-args-noAdd6}
	\end{minipage}

\yamltitle{indentRules}*{fields}
	We may also specify indentation rules for environment code blocks using the
	\texttt{indentRules} field; see, for example, \cref{lst:myenv-rules1,lst:myenv-rules2}.

	\begin{cmhtcbraster}[raster column skip=.1\linewidth]
		\cmhlistingsfromfile[style=yaml-LST]{demonstrations/myenv-rules1.yaml}[width=.8\linewidth,before=\centering,yaml-TCB]{\texttt{myenv-rules1.yaml}}{lst:myenv-rules1}
		\cmhlistingsfromfile[style=yaml-LST]{demonstrations/myenv-rules2.yaml}[width=.8\linewidth,before=\centering,yaml-TCB]{\texttt{myenv-rules2.yaml}}{lst:myenv-rules2}
	\end{cmhtcbraster}

	On applying either of the following commands,
	\index{switches!-l demonstration}
	\begin{commandshell}
latexindent.pl myenv.tex -l myenv-rules1.yaml  
latexindent.pl myenv.tex -l myenv-rules2.yaml  
\end{commandshell}
	we obtain the output given in \cref{lst:myenv-rules-output}; note in particular that the
	environment \texttt{myenv} has received one tab (from the \texttt{outer} environment)
	plus three spaces from \cref{lst:myenv-rules1} or \ref{lst:myenv-rules2}.

	\cmhlistingsfromfile[showtabs=true,showspaces=true]{demonstrations/myenv-rules1.tex}{\texttt{myenv.tex} output (using either \cref{lst:myenv-rules1} or \cref{lst:myenv-rules2})}{lst:myenv-rules-output}

	If you specify a field in \texttt{indentRules} using anything other than horizontal
	space, it will be ignored.

	Returning to the example in \cref{lst:myenv-args} that contains optional and mandatory
	arguments. Upon using \cref{lst:myenv-rules1} as in
	\index{switches!-l demonstration}
	\begin{commandshell}
latexindent.pl myenv-args.tex -l=myenv-rules1.yaml  
\end{commandshell}
	we obtain the output in \cref{lst:myenv-args-rules1}; note that the body, optional
	argument and mandatory argument of \texttt{myenv} have \emph{all} received the same
	customised indentation.
	\cmhlistingsfromfile[showtabs=true,showspaces=true]{demonstrations/myenvironment-args-rules1.tex}{\texttt{myenv-args.tex} using \cref{lst:myenv-rules1}}{lst:myenv-args-rules1}

	You can specify different indentation rules for the different features using, for
	example, \cref{lst:myenv-rules3,lst:myenv-rules4}

	\begin{minipage}{.49\textwidth}
		\cmhlistingsfromfile[style=yaml-LST]{demonstrations/myenv-rules3.yaml}[width=.9\linewidth,before=\centering,yaml-TCB]{\texttt{myenv-rules3.yaml}}{lst:myenv-rules3}
	\end{minipage}
	\hfill
	\begin{minipage}{.49\textwidth}
		\cmhlistingsfromfile[style=yaml-LST]{demonstrations/myenv-rules4.yaml}[width=.9\linewidth,before=\centering,yaml-TCB]{\texttt{myenv-rules4.yaml}}{lst:myenv-rules4}
	\end{minipage}

	After running
	\index{switches!-l demonstration}
	\begin{commandshell}
latexindent.pl myenv-args.tex -l myenv-rules3.yaml  
latexindent.pl myenv-args.tex -l myenv-rules4.yaml  
\end{commandshell}
	then we obtain the respective outputs given in
	\cref{lst:myenv-args-rules3,lst:myenv-args-rules4}.

	\begin{widepage}
		\begin{minipage}{.5\textwidth}
			\cmhlistingsfromfile[showtabs=true,showspaces=true]{demonstrations/myenvironment-args-rules3.tex}{\texttt{myenv-args.tex} using \cref{lst:myenv-rules3}}{lst:myenv-args-rules3}
		\end{minipage}%
		\hfill
		\begin{minipage}{.5\textwidth}
			\cmhlistingsfromfile[showtabs=true,showspaces=true]{demonstrations/myenvironment-args-rules4.tex}{\texttt{myenv-args.tex} using \cref{lst:myenv-rules4}}{lst:myenv-args-rules4}
		\end{minipage}
	\end{widepage}

	Note that in \cref{lst:myenv-args-rules3}, the optional argument has only received a
	single space of indentation, while the mandatory argument has received the default (tab)
	indentation; the environment body has received three spaces of indentation.

	In \cref{lst:myenv-args-rules4}, the optional argument has received the default (tab)
	indentation, the mandatory argument has received two tabs of indentation, and the body
	has received three spaces of indentation.

\yamltitle{noAdditionalIndentGlobal}*{fields}
	Assuming that your environment name is not found within neither
	\texttt{noAdditionalIndent} nor \texttt{indentRules}, the next place that
	\texttt{latexindent.pl} will look is \texttt{noAdditionalIndentGlobal}, and in particular
	\emph{for the environments} key (see \cref{lst:noAdditionalIndentGlobal:environments}).

	\cmhlistingsfromfile[style=noAdditionalIndentGlobalEnv]{../defaultSettings.yaml}[width=.5\linewidth,before=\centering,yaml-TCB]{\texttt{noAdditionalIndentGlobal}}{lst:noAdditionalIndentGlobal:environments}

	Let's say that you change the value of \texttt{environments} to \texttt{1} in
	\cref{lst:noAdditionalIndentGlobal:environments}, and that you run
	\index{switches!-l demonstration}

	\begin{widepage}
		\begin{commandshell}
latexindent.pl myenv-args.tex -l env-noAdditionalGlobal.yaml
latexindent.pl myenv-args.tex -l myenv-rules1.yaml,env-noAdditionalGlobal.yaml
\end{commandshell}
	\end{widepage}

	The respective output from these two commands are in
	\cref{lst:myenv-args-no-add-global1,lst:myenv-args-no-add-global2}; in
	\cref{lst:myenv-args-no-add-global1} notice that \emph{both} environments receive no
	additional indentation but that the arguments of \texttt{myenv} still \emph{do} receive
	indentation. In \cref{lst:myenv-args-no-add-global2} notice that the \emph{outer}
	environment does not receive additional indentation, but because of the settings from
	\texttt{myenv-rules1.yaml} (in \vref{lst:myenv-rules1}), the \texttt{myenv} environment
	still \emph{does} receive indentation.

	\begin{minipage}{.45\textwidth}
		\cmhlistingsfromfile{demonstrations/myenvironment-args-rules1-noAddGlobal1.tex}{\texttt{myenv-args.tex} using \cref{lst:noAdditionalIndentGlobal:environments}}{lst:myenv-args-no-add-global1}
	\end{minipage}
	\hfill
	\begin{minipage}{.45\textwidth}
		\cmhlistingsfromfile{demonstrations/myenvironment-args-rules1-noAddGlobal2.tex}{\texttt{myenv-args.tex} using \cref{lst:noAdditionalIndentGlobal:environments,lst:myenv-rules1}}{lst:myenv-args-no-add-global2}
	\end{minipage}

	In fact, \texttt{noAdditionalIndentGlobal} also contains keys that control the
	indentation of optional and mandatory arguments; on referencing
	\cref{lst:opt-args-no-add-glob,lst:mand-args-no-add-glob}

	\begin{minipage}{.49\textwidth}
		\cmhlistingsfromfile[style=yaml-LST]{demonstrations/opt-args-no-add-glob.yaml}[width=.8\linewidth,before=\centering,yaml-TCB]{\texttt{opt-args-no-add-glob.yaml}}{lst:opt-args-no-add-glob}
	\end{minipage}
	\hfill
	\begin{minipage}{.49\textwidth}
		\cmhlistingsfromfile[style=yaml-LST]{demonstrations/mand-args-no-add-glob.yaml}[width=.8\linewidth,before=\centering,yaml-TCB]{\texttt{mand-args-no-add-glob.yaml}}{lst:mand-args-no-add-glob}
	\end{minipage}

	we may run the commands
	\index{switches!-l demonstration}
	\begin{commandshell}
latexindent.pl myenv-args.tex -local opt-args-no-add-glob.yaml
latexindent.pl myenv-args.tex -local mand-args-no-add-glob.yaml
\end{commandshell}
	which produces the respective outputs given in
	\cref{lst:myenv-args-no-add-opt,lst:myenv-args-no-add-mand}. Notice that in
	\cref{lst:myenv-args-no-add-opt} the \emph{optional} argument has not received any
	additional indentation, and in \cref{lst:myenv-args-no-add-mand} the \emph{mandatory}
	argument has not received any additional indentation.

	\begin{minipage}{.45\textwidth}
		\cmhlistingsfromfile{demonstrations/myenvironment-args-rules1-noAddGlobal3.tex}{\texttt{myenv-args.tex} using \cref{lst:opt-args-no-add-glob}}{lst:myenv-args-no-add-opt}
	\end{minipage}
	\hfill
	\begin{minipage}{.45\textwidth}
		\cmhlistingsfromfile{demonstrations/myenvironment-args-rules1-noAddGlobal4.tex}{\texttt{myenv-args.tex} using \cref{lst:mand-args-no-add-glob}}{lst:myenv-args-no-add-mand}
	\end{minipage}

\yamltitle{indentRulesGlobal}*{fields}
	The final check that \texttt{latexindent.pl} will make is to look for
	\texttt{indentRulesGlobal} as detailed in \cref{lst:indentRulesGlobal:environments}.

	\cmhlistingsfromfile[style=indentRulesGlobalEnv]{../defaultSettings.yaml}[width=.5\linewidth,before=\centering,yaml-TCB]{\texttt{indentRulesGlobal}}{lst:indentRulesGlobal:environments}

	If you change the \texttt{environments} field to anything involving horizontal space, say
	\lstinline!" "!, and then run the following commands
	\index{switches!-l demonstration}

	\begin{commandshell}
latexindent.pl myenv-args.tex -l env-indentRules.yaml
latexindent.pl myenv-args.tex -l myenv-rules1.yaml,env-indentRules.yaml
\end{commandshell}
	then the respective output is shown in
	\cref{lst:myenv-args-indent-rules-global1,lst:myenv-args-indent-rules-global2}. Note that
	in \cref{lst:myenv-args-indent-rules-global1}, both the environment blocks have received
	a single-space indentation, whereas in \cref{lst:myenv-args-indent-rules-global2} the
	\texttt{outer} environment has received single-space indentation (specified by
	\texttt{indentRulesGlobal}), but \texttt{myenv} has received \lstinline!"   "!, as
	specified by the particular \texttt{indentRules} for \texttt{myenv}
	\vref{lst:myenv-rules1}.

	\begin{minipage}{.45\textwidth}
		\cmhlistingsfromfile[showspaces=true]{demonstrations/myenvironment-args-global-rules1.tex}{\texttt{myenv-args.tex} using \cref{lst:indentRulesGlobal:environments}}{lst:myenv-args-indent-rules-global1}
	\end{minipage}
	\hfill
	\begin{minipage}{.45\textwidth}
		\cmhlistingsfromfile[showspaces=true]{demonstrations/myenvironment-args-global-rules2.tex}{\texttt{myenv-args.tex} using \cref{lst:myenv-rules1,lst:indentRulesGlobal:environments}}{lst:myenv-args-indent-rules-global2}
	\end{minipage}

	You can specify \texttt{indentRulesGlobal} for both optional and mandatory arguments, as
	detailed in \cref{lst:opt-args-indent-rules-glob,lst:mand-args-indent-rules-glob}

	\begin{minipage}{.49\textwidth}
		\cmhlistingsfromfile[style=yaml-LST]{demonstrations/opt-args-indent-rules-glob.yaml}[width=.9\linewidth,before=\centering,yaml-TCB]{\texttt{opt-args-indent-rules-glob.yaml}}{lst:opt-args-indent-rules-glob}
	\end{minipage}
	\hfill
	\begin{minipage}{.49\textwidth}
		\cmhlistingsfromfile[style=yaml-LST]{demonstrations/mand-args-indent-rules-glob.yaml}[width=.9\linewidth,before=\centering,yaml-TCB]{\texttt{mand-args-indent-rules-glob.yaml}}{lst:mand-args-indent-rules-glob}
	\end{minipage}

	Upon running the following commands
	\index{switches!-l demonstration}
	\begin{commandshell}
latexindent.pl myenv-args.tex -local opt-args-indent-rules-glob.yaml
latexindent.pl myenv-args.tex -local mand-args-indent-rules-glob.yaml
\end{commandshell}
	we obtain the respective outputs in
	\cref{lst:myenv-args-indent-rules-global3,lst:myenv-args-indent-rules-global4}. Note that
	the \emph{optional} argument in \cref{lst:myenv-args-indent-rules-global3} has received
	two tabs worth of indentation, while the \emph{mandatory} argument has done so in
	\cref{lst:myenv-args-indent-rules-global4}.

	\begin{widepage}
		\begin{minipage}{.55\textwidth}
			\cmhlistingsfromfile[showtabs=true]{demonstrations/myenvironment-args-global-rules3.tex}{\texttt{myenv-args.tex} using \cref{lst:opt-args-indent-rules-glob}}{lst:myenv-args-indent-rules-global3}
		\end{minipage}
		\hfill
		\begin{minipage}{.55\textwidth}
			\cmhlistingsfromfile[showtabs=true]{demonstrations/myenvironment-args-global-rules4.tex}{\texttt{myenv-args.tex} using \cref{lst:mand-args-indent-rules-glob}}{lst:myenv-args-indent-rules-global4}
		\end{minipage}
	\end{widepage}

%% arara: pdflatex: {shell: yes, files: [latexindent]}
\subsubsection{Environments with items}
	With reference to \vref{lst:indentafteritems,lst:itemNames}, some commands
	may contain \texttt{item} commands; for the purposes of this discussion,
	we will use the code from \vref{lst:itemsbefore}.

	Assuming that you've populated \texttt{itemNames} with the name of your
	\texttt{item}, you can put the item name into \texttt{noAdditionalIndent}
	as in \cref{lst:item-noAdd1}, although a more efficient approach may be
	to change the relevant field in \texttt{itemNames} to \texttt{0}. Similarly,
	you can customise the indentation that your \texttt{item} receives using
	\texttt{indentRules}, as in \cref{lst:item-rules1}

	\begin{minipage}{.45\textwidth}
		\cmhlistingsfromfile[style=yaml-LST]{demonstrations/item-noAdd1.yaml}[yaml-TCB]{\texttt{item-noAdd1.yaml}}{lst:item-noAdd1}
	\end{minipage}%
	\hfill
	\begin{minipage}{.45\textwidth}
		\cmhlistingsfromfile[style=yaml-LST]{demonstrations/item-rules1.yaml}[yaml-TCB]{\texttt{item-rules1.yaml}}{lst:item-rules1}
	\end{minipage}

	Upon running the following commands
	\begin{commandshell}
latexindent.pl items1.tex -local item-noAdd1.yaml  
latexindent.pl items1.tex -local item-rules1.yaml  
\end{commandshell}
	the respective outputs are given in \cref{lst:items1-noAdd1,lst:items1-rules1}; note that in \cref{lst:items1-noAdd1}
	that the text after each \texttt{item} has not received any additional indentation, and in \cref{lst:items1-rules1},
	the text after each \texttt{item} has received a single space of indentation, specified by \cref{lst:item-rules1}.

	\begin{minipage}{.45\textwidth}
		\cmhlistingsfromfile{demonstrations/items1-noAdd1.tex}{\texttt{items1.tex} using \cref{lst:item-noAdd1}}{lst:items1-noAdd1}
	\end{minipage}
	\hfill
	\begin{minipage}{.45\textwidth}
		\cmhlistingsfromfile{demonstrations/items1-rules1.tex}{\texttt{items1.tex} using \cref{lst:item-rules1}}{lst:items1-rules1}
	\end{minipage}

	Alternatively, you might like to populate \texttt{noAdditionalIndentGlobal} or \texttt{indentRulesGlobal} using the \texttt{items}
	key, as demonstrated in \cref{lst:items-noAdditionalGlobal,lst:items-indentRulesGlobal}. Note that there is a need to
	`reset/remove' the \texttt{item} field from \texttt{indentRules} in both cases (see the hierarchy description given on \cpageref{sec:noadd-indent-rules})
	as the \texttt{item} command is a member of \texttt{indentRules} by default.

	\begin{minipage}{.45\textwidth}
		\cmhlistingsfromfile[style=yaml-LST]{demonstrations/items-noAdditionalGlobal.yaml}[yaml-TCB]{\texttt{items-noAdditionalGlobal.yaml}}{lst:items-noAdditionalGlobal}
	\end{minipage}%
	\hfill
	\begin{minipage}{.45\textwidth}
		\cmhlistingsfromfile[style=yaml-LST]{demonstrations/items-indentRulesGlobal.yaml}[yaml-TCB]{\texttt{items-indentRulesGlobal.yaml}}{lst:items-indentRulesGlobal}
	\end{minipage}

	Upon running the following commands,
	\begin{commandshell}
latexindent.pl items1.tex -local items-noAdditionalGlobal.yaml
latexindent.pl items1.tex -local items-indentRulesGlobal.yaml
\end{commandshell}
	the respective outputs from \cref{lst:items1-noAdd1,lst:items1-rules1} are obtained; note, however, that
	\emph{all} such \texttt{item} commands without their own individual \texttt{noAdditionalIndent} or \texttt{indentRules}
	settings would behave as in these listings.

%% arara: pdflatex: { files: [latexindent]}
\subsubsection{Commands with arguments}\label{subsubsec:commands-arguments}
	Let's begin with the simple example in \cref{lst:mycommand}; when \texttt{latexindent.pl}
	operates on this file, the default output is shown in \cref{lst:mycommand-default}.
	\footnote{The command code blocks have quite a few subtleties, described in
		\vref{subsec:commands-string-between}.}

	\begin{cmhtcbraster}[raster column skip=.1\linewidth]
		\cmhlistingsfromfile{demonstrations/mycommand.tex}{\texttt{mycommand.tex}}{lst:mycommand}
		\cmhlistingsfromfile{demonstrations/mycommand-default.tex}{\texttt{mycommand.tex} default output}{lst:mycommand-default}
	\end{cmhtcbraster}

	As in the environment-based case (see \vref{lst:myenv-noAdd1,lst:myenv-noAdd2}) we may
	specify \texttt{noAdditionalIndent} either in `scalar' form, or in `field' form, as shown
	in \cref{lst:mycommand-noAdd1,lst:mycommand-noAdd2}

	\begin{minipage}{.45\textwidth}
		\cmhlistingsfromfile[style=yaml-LST]{demonstrations/mycommand-noAdd1.yaml}[width=.8\linewidth,before=\centering,yaml-TCB]{\texttt{mycommand-noAdd1.yaml}}{lst:mycommand-noAdd1}
	\end{minipage}
	\hfill
	\begin{minipage}{.45\textwidth}
		\cmhlistingsfromfile[style=yaml-LST]{demonstrations/mycommand-noAdd2.yaml}[width=.8\linewidth,before=\centering,yaml-TCB]{\texttt{mycommand-noAdd2.yaml}}{lst:mycommand-noAdd2}
	\end{minipage}

	After running the following commands,
	\index{switches!-l demonstration}
	\begin{commandshell}
latexindent.pl mycommand.tex -l mycommand-noAdd1.yaml  
latexindent.pl mycommand.tex -l mycommand-noAdd2.yaml  
\end{commandshell}
	we receive the respective output given in
	\cref{lst:mycommand-output-noAdd1,lst:mycommand-output-noAdd2}

	\begin{minipage}{.45\textwidth}
		\cmhlistingsfromfile{demonstrations/mycommand-noAdd1.tex}{\texttt{mycommand.tex} using \cref{lst:mycommand-noAdd1}}{lst:mycommand-output-noAdd1}
	\end{minipage}
	\hfill
	\begin{minipage}{.45\textwidth}
		\cmhlistingsfromfile{demonstrations/mycommand-noAdd2.tex}{\texttt{mycommand.tex} using \cref{lst:mycommand-noAdd2}}{lst:mycommand-output-noAdd2}
	\end{minipage}

	Note that in \cref{lst:mycommand-output-noAdd1} that the `body', optional argument
	\emph{and} mandatory argument have \emph{all} received no additional indentation, while
	in \cref{lst:mycommand-output-noAdd2}, only the `body' has not received any additional
	indentation. We define the `body' of a command as any lines following the command name
	that include its optional or mandatory arguments.

	We may further customise \texttt{noAdditionalIndent} for \texttt{mycommand} as we did in
	\vref{lst:myenv-noAdd5,lst:myenv-noAdd6}; explicit examples are given in
	\cref{lst:mycommand-noAdd3,lst:mycommand-noAdd4}.

	\begin{minipage}{.45\textwidth}
		\cmhlistingsfromfile[style=yaml-LST]{demonstrations/mycommand-noAdd3.yaml}[width=.9\linewidth,before=\centering,yaml-TCB]{\texttt{mycommand-noAdd3.yaml}}{lst:mycommand-noAdd3}
	\end{minipage}
	\hfill
	\begin{minipage}{.45\textwidth}
		\cmhlistingsfromfile[style=yaml-LST]{demonstrations/mycommand-noAdd4.yaml}[width=.9\linewidth,before=\centering,yaml-TCB]{\texttt{mycommand-noAdd4.yaml}}{lst:mycommand-noAdd4}
	\end{minipage}

	After running the following commands,
	\index{switches!-l demonstration}
	\begin{commandshell}
latexindent.pl mycommand.tex -l mycommand-noAdd3.yaml  
latexindent.pl mycommand.tex -l mycommand-noAdd4.yaml  
\end{commandshell}
	we receive the respective output given in
	\cref{lst:mycommand-output-noAdd3,lst:mycommand-output-noAdd4}.

	\begin{minipage}{.45\textwidth}
		\cmhlistingsfromfile{demonstrations/mycommand-noAdd3.tex}{\texttt{mycommand.tex} using \cref{lst:mycommand-noAdd3}}{lst:mycommand-output-noAdd3}
	\end{minipage}
	\hfill
	\begin{minipage}{.45\textwidth}
		\cmhlistingsfromfile{demonstrations/mycommand-noAdd4.tex}{\texttt{mycommand.tex} using \cref{lst:mycommand-noAdd4}}{lst:mycommand-output-noAdd4}
	\end{minipage}

	Attentive readers will note that the body of \texttt{mycommand} in both
	\cref{lst:mycommand-output-noAdd3,lst:mycommand-output-noAdd4} has received no additional
	indent, even though \texttt{body} is explicitly set to \texttt{0} in both
	\cref{lst:mycommand-noAdd3,lst:mycommand-noAdd4}. This is because, by default,
	\texttt{noAdditionalIndentGlobal} for \texttt{commands} is set to \texttt{1} by default;
	this can be easily fixed as in
	\cref{lst:mycommand-noAdd5,lst:mycommand-noAdd6}.\label{page:command:noAddGlobal}

	\begin{minipage}{.45\textwidth}
		\cmhlistingsfromfile[style=yaml-LST]{demonstrations/mycommand-noAdd5.yaml}[width=.9\linewidth,before=\centering,yaml-TCB]{\texttt{mycommand-noAdd5.yaml}}{lst:mycommand-noAdd5}
	\end{minipage}
	\hfill
	\begin{minipage}{.45\textwidth}
		\cmhlistingsfromfile[style=yaml-LST]{demonstrations/mycommand-noAdd6.yaml}[width=.9\linewidth,before=\centering,yaml-TCB]{\texttt{mycommand-noAdd6.yaml}}{lst:mycommand-noAdd6}
	\end{minipage}

	After running the following commands,
	\index{switches!-l demonstration}
	\begin{commandshell}
latexindent.pl mycommand.tex -l mycommand-noAdd5.yaml  
latexindent.pl mycommand.tex -l mycommand-noAdd6.yaml  
\end{commandshell}
	we receive the respective output given in
	\cref{lst:mycommand-output-noAdd5,lst:mycommand-output-noAdd6}.

	\begin{minipage}{.45\textwidth}
		\cmhlistingsfromfile{demonstrations/mycommand-noAdd5.tex}{\texttt{mycommand.tex} using \cref{lst:mycommand-noAdd5}}{lst:mycommand-output-noAdd5}
	\end{minipage}
	\hfill
	\begin{minipage}{.45\textwidth}
		\cmhlistingsfromfile{demonstrations/mycommand-noAdd6.tex}{\texttt{mycommand.tex} using \cref{lst:mycommand-noAdd6}}{lst:mycommand-output-noAdd6}
	\end{minipage}

	Both \texttt{indentRules} and \texttt{indentRulesGlobal} can be adjusted as they were for
	\emph{environment} code blocks, as in \vref{lst:myenv-rules3,lst:myenv-rules4} and
	\vref{lst:indentRulesGlobal:environments,lst:opt-args-indent-rules-glob,lst:mand-args-indent-rules-glob}.

%% arara: pdflatex: { files: [latexindent]}
\subsubsection{ifelsefi code blocks}
 \begin{example}
 Let's use the simple example shown in \cref{lst:ifelsefi1}; when \texttt{latexindent.pl}
 operates on this file, the output as in \cref{lst:ifelsefi1-default}; note that the body
 of each of the \lstinline!\if! statements have been indented, and that the
 \lstinline!\else! statement has been accounted for correctly.

 \begin{cmhtcbraster}
  \cmhlistingsfromfile{demonstrations/ifelsefi1.tex}{\texttt{ifelsefi1.tex}}{lst:ifelsefi1}
  \cmhlistingsfromfile{demonstrations/ifelsefi1-default.tex}{\texttt{ifelsefi1.tex} default output}{lst:ifelsefi1-default}
 \end{cmhtcbraster}

 It is recommended to specify \texttt{noAdditionalIndent} and \texttt{indentRules} in the
 `scalar' form only for these type of code blocks, although the `field' form would work,
 assuming that \texttt{body} was specified. Examples are shown in
 \cref{lst:ifnum-noAdd,lst:ifnum-indent-rules}.

 \begin{cmhtcbraster}
  \cmhlistingsfromfile[style=yaml-LST]{demonstrations/ifnum-noAdd.yaml}[width=.8\linewidth,before=\centering,yaml-TCB]{\texttt{ifnum-noAdd.yaml}}{lst:ifnum-noAdd}
  \cmhlistingsfromfile[style=yaml-LST]{demonstrations/ifnum-indent-rules.yaml}[width=.8\linewidth,before=\centering,yaml-TCB]{\texttt{ifnum-indent-rules.yaml}}{lst:ifnum-indent-rules}
 \end{cmhtcbraster}

 After running the following commands, \index{switches!-l demonstration}

 \begin{commandshell}
latexindent.pl ifelsefi1.tex -local ifnum-noAdd.yaml  
latexindent.pl ifelsefi1.tex -l ifnum-indent-rules.yaml  
\end{commandshell}

 we receive the respective output given in
 \cref{lst:ifelsefi1-output-noAdd,lst:ifelsefi1-output-indent-rules}; note that in
 \cref{lst:ifelsefi1-output-noAdd}, the \texttt{ifnum} code block has \emph{not} received
 any additional indentation, while in \cref{lst:ifelsefi1-output-indent-rules}, the
 \texttt{ifnum} code block has received one tab and two spaces of indentation.

 \begin{cmhtcbraster}
  \cmhlistingsfromfile{demonstrations/ifelsefi1-noAdd.tex}{\texttt{ifelsefi1.tex} using \cref{lst:ifnum-noAdd}}{lst:ifelsefi1-output-noAdd}
  \cmhlistingsfromfile[showspaces=true,showtabs=true]{demonstrations/ifelsefi1-indent-rules.tex}{\texttt{ifelsefi1.tex} using \cref{lst:ifnum-indent-rules}}{lst:ifelsefi1-output-indent-rules}
 \end{cmhtcbraster}
 \end{example}

 \begin{example}
 We may specify \texttt{noAdditionalIndentGlobal} and \texttt{indentRulesGlobal} as in
 \cref{lst:ifelsefi-noAdd-glob,lst:ifelsefi-indent-rules-global}.

 \begin{cmhtcbraster}
  \cmhlistingsfromfile[style=yaml-LST]{demonstrations/ifelsefi-noAdd-glob.yaml}[width=.9\linewidth,before=\centering,yaml-TCB]{\texttt{ifelsefi-noAdd-glob.yaml}}{lst:ifelsefi-noAdd-glob}
  \cmhlistingsfromfile[style=yaml-LST]{demonstrations/ifelsefi-indent-rules-global.yaml}[width=.9\linewidth,before=\centering,yaml-TCB]{\texttt{ifelsefi-indent-rules-global.yaml}}{lst:ifelsefi-indent-rules-global}
 \end{cmhtcbraster}

 Upon running the following commands \index{switches!-l demonstration}
 \begin{commandshell}
latexindent.pl ifelsefi1.tex -local ifelsefi-noAdd-glob.yaml  
latexindent.pl ifelsefi1.tex -l ifelsefi-indent-rules-global.yaml  
\end{commandshell}
 we receive the outputs in
 \cref{lst:ifelsefi1-output-noAdd-glob,lst:ifelsefi1-output-indent-rules-global}; notice
 that in \cref{lst:ifelsefi1-output-noAdd-glob} neither of the \texttt{ifelsefi} code
 blocks have received indentation, while in
 \cref{lst:ifelsefi1-output-indent-rules-global} both code blocks have received a single
 space of indentation.

 \begin{cmhtcbraster}
  \cmhlistingsfromfile{demonstrations/ifelsefi1-noAdd-glob.tex}{\texttt{ifelsefi1.tex} using \cref{lst:ifelsefi-noAdd-glob}}{lst:ifelsefi1-output-noAdd-glob}
  \cmhlistingsfromfile[showspaces=true]{demonstrations/ifelsefi1-indent-rules-global.tex}{\texttt{ifelsefi1.tex} using \cref{lst:ifelsefi-indent-rules-global}}{lst:ifelsefi1-output-indent-rules-global}
 \end{cmhtcbraster}
 \end{example}

 \begin{example}
 We can further explore the treatment of \texttt{ifElseFi} code blocks in
 \cref{lst:ifelsefi2}, and the associated default output given in
 \cref{lst:ifelsefi2-default}; note, in particular,\announce{2018-04-27}*{updates to
 ifElseFi code blocks} that the bodies of each of the `or statements' have been indented.%

 \begin{cmhtcbraster}[raster column skip=.1\linewidth]
  \cmhlistingsfromfile{demonstrations/ifelsefi2.tex}{\texttt{ifelsefi2.tex}}{lst:ifelsefi2}
  \cmhlistingsfromfile{demonstrations/ifelsefi2-default.tex}{\texttt{ifelsefi2.tex} default output}{lst:ifelsefi2-default}
 \end{cmhtcbraster}
 \end{example}

%% arara: pdflatex: {files: [latexindent]}
\subsubsection{specialBeginEnd code blocks}
	Let's use the example from \vref{lst:specialbefore} which has default output shown in
	\vref{lst:specialafter}.

	It is recommended to specify \texttt{noAdditionalIndent} and \texttt{indentRules} in the
	`scalar' form for these type of code blocks, although the `field' form would work,
	assuming that \texttt{body} was specified. Examples are shown in
	\cref{lst:displayMath-noAdd,lst:displayMath-indent-rules}.
	\index{specialBeginEnd!noAdditionalIndent}
	\index{specialBeginEnd!indentRules example}

	\begin{minipage}{.49\textwidth}
		\cmhlistingsfromfile[style=yaml-LST]{demonstrations/displayMath-noAdd.yaml}[width=.9\linewidth,before=\centering,yaml-TCB]{\texttt{displayMath-noAdd.yaml}}{lst:displayMath-noAdd}
	\end{minipage}
	\hfill
	\begin{minipage}{.49\textwidth}
		\cmhlistingsfromfile[style=yaml-LST]{demonstrations/displayMath-indent-rules.yaml}[width=.9\linewidth,before=\centering,yaml-TCB]{\texttt{displayMath-indent-rules.yaml}}{lst:displayMath-indent-rules}
	\end{minipage}

	After running the following commands,
	\index{switches!-l demonstration}
	\begin{commandshell}
latexindent.pl special1.tex -local displayMath-noAdd.yaml  
latexindent.pl special1.tex -l displayMath-indent-rules.yaml  
\end{commandshell}
	we receive the respective output given in
	\cref{lst:special1-output-noAdd,lst:special1-output-indent-rules}; note that in
	\cref{lst:special1-output-noAdd}, the \texttt{displayMath} code block has \emph{not}
	received any additional indentation, while in \cref{lst:special1-output-indent-rules},
	the \texttt{displayMath} code block has received three tabs worth of indentation.

	\begin{minipage}{.45\textwidth}
		\cmhlistingsfromfile{demonstrations/special1-noAdd.tex}{\texttt{special1.tex} using \cref{lst:displayMath-noAdd}}{lst:special1-output-noAdd}
	\end{minipage}
	\hfill
	\begin{minipage}{.45\textwidth}
		\cmhlistingsfromfile[showtabs=true]{demonstrations/special1-indent-rules.tex}{\texttt{special1.tex} using \cref{lst:displayMath-indent-rules}}{lst:special1-output-indent-rules}
	\end{minipage}

	We may specify \texttt{noAdditionalIndentGlobal} and \texttt{indentRulesGlobal} as in
	\cref{lst:special-noAdd-glob,lst:special-indent-rules-global}.

	\begin{minipage}{.49\textwidth}
		\cmhlistingsfromfile[style=yaml-LST]{demonstrations/special-noAdd-glob.yaml}[width=.9\linewidth,before=\centering,yaml-TCB]{\texttt{special-noAdd-glob.yaml}}{lst:special-noAdd-glob}
	\end{minipage}
	\hfill
	\begin{minipage}{.49\textwidth}
		\cmhlistingsfromfile[style=yaml-LST]{demonstrations/special-indent-rules-global.yaml}[width=.9\linewidth,before=\centering,yaml-TCB]{\texttt{special-indent-rules-global.yaml}}{lst:special-indent-rules-global}
	\end{minipage}

	Upon running the following commands
	\index{switches!-l demonstration}
	\begin{commandshell}
latexindent.pl special1.tex -local special-noAdd-glob.yaml  
latexindent.pl special1.tex -l special-indent-rules-global.yaml  
\end{commandshell}
	we receive the outputs in
	\cref{lst:special1-output-noAdd-glob,lst:special1-output-indent-rules-global}; notice
	that in \cref{lst:special1-output-noAdd-glob} neither of the \texttt{special} code blocks
	have received indentation, while in \cref{lst:special1-output-indent-rules-global} both
	code blocks have received a single space of indentation.

	\begin{minipage}{.45\textwidth}
		\cmhlistingsfromfile{demonstrations/special1-noAdd-glob.tex}{\texttt{special1.tex} using \cref{lst:special-noAdd-glob}}{lst:special1-output-noAdd-glob}
	\end{minipage}
	\hfill
	\begin{minipage}{.45\textwidth}
		\cmhlistingsfromfile[showspaces=true]{demonstrations/special1-indent-rules-global.tex}{\texttt{special1.tex} using \cref{lst:special-indent-rules-global}}{lst:special1-output-indent-rules-global}
	\end{minipage}

% arara: pdflatex: {shell: yes, files: [latexindent]}
\fancyhead[R]{\bfseries\thepage%
	\tikz[remember picture,overlay] {
		\node at (1,0){\includegraphics{logo}};
	}}
\section{The \texttt{-m} (\texttt{modifylinebreaks}) switch}\label{sec:modifylinebreaks}
 All features described in this section will only be relevant if the \texttt{-m} switch
 is used.

\yamltitle{modifylinebreaks}*{fields}
	\begin{wrapfigure}[7]{r}[0pt]{8cm}
		\cmhlistingsfromfile[style=modifylinebreaks]{../defaultSettings.yaml}[MLB-TCB,width=.85\linewidth,before=\centering]{\texttt{modifyLineBreaks}}{lst:modifylinebreaks}
	\end{wrapfigure}
	\makebox[0pt][r]{%
		\raisebox{-\totalheight}[0pt][0pt]{%
			\tikz\node[opacity=1] at (0,0) {\includegraphics[width=4cm]{logo}};}}%	
	As of Version 3.0, \texttt{latexindent.pl} has the \texttt{-m} switch, which
	permits \texttt{latexindent.pl} to modify line breaks, according to the
	specifications in the \texttt{modifyLineBreaks} field. \emph{The settings
		in this field will only be considered if the \texttt{-m} switch has been used}.
	A snippet of the default settings of this field is shown in \cref{lst:modifylinebreaks}.

	Having read the previous paragraph, it should sound reasonable that, if you call \texttt{latexindent.pl}
	using the \texttt{-m} switch, then you give it permission to modify line breaks in your file,
	but let's be clear:

	\begin{warning}
		If you call \texttt{latexindent.pl} with the \texttt{-m} switch, then you
		are giving it permission to modify line breaks. By default, the only
		thing that will happen is that multiple blank lines will be condensed into
		one blank line; many other settings are possible, discussed next.
	\end{warning}

\yamltitle{preserveBlankLines}{0|1}
	This field is directly related to \emph{poly-switches}, discussed below.
	By default, it is set to \texttt{1}, which means that blank lines will
	be protected from removal; however, regardless of this setting, multiple
	blank lines can be condensed if \texttt{condenseMultipleBlankLinesInto} is
	greater than \texttt{0}, discussed next.

\yamltitle{condenseMultipleBlankLinesInto}*{integer $\geq 0$}
	Assuming that this switch takes an integer value greater than \texttt{0}, \texttt{latexindent.pl} will condense multiple blank lines into
	the number of blank lines illustrated by this switch. As an example, \cref{lst:mlb-bl} shows a sample file
	with blank lines; upon running
	\begin{commandshell}
latexindent.pl myfile.tex -m  
\end{commandshell}
	the output is shown in \cref{lst:mlb-bl-out}; note that the multiple blank lines have been
	condensed into one blank line, and note also that we have used the \texttt{-m} switch!

	\begin{minipage}{.45\textwidth}
		\cmhlistingsfromfile{demonstrations/mlb1.tex}{\texttt{mlb1.tex}}{lst:mlb-bl}
	\end{minipage}%
	\hfill
	\begin{minipage}{.45\textwidth}
		\cmhlistingsfromfile{demonstrations/mlb1-out.tex}{\texttt{mlb1.tex} out output}{lst:mlb-bl-out}
	\end{minipage}

\yamltitle{textWrapOptions}*{fields}
	When the \texttt{-m} switch is active \texttt{latexindent.pl} has the ability to wrap text using the options
	specified in the \texttt{textWrapOptions} field, see \cref{lst:textWrapOptions}. The value of
	\texttt{columns} specifies the column at which the text should be wrapped.
	By default, the value of \texttt{columns} is \texttt{0}, so \texttt{latexindent.pl}
	will \emph{not} wrap text; if you change it to a value of \texttt{2} or more, then
	text will be wrapped after the character in the specified column.

	\cmhlistingsfromfile[style=textWrapOptions]{../defaultSettings.yaml}[MLB-TCB,width=.85\linewidth,before=\centering]{\texttt{textWrapOptions}}{lst:textWrapOptions}

	For example, consider the file give in \cref{lst:textwrap1}.

	\begin{widepage}
		\cmhlistingsfromfile{demonstrations/textwrap1.tex}{\texttt{textwrap1.tex}}{lst:textwrap1}
	\end{widepage}

	Using the file \texttt{textwrap1.yaml} in \cref{lst:textwrap1-yaml}, and running the command
	\begin{commandshell}
latexindent.pl -m textwrap1.tex -o textwrap1-mod1.tex -l textwrap1.yaml
\end{commandshell}
	we obtain the output in \cref{lst:textwrap1-mod1}.

	\begin{minipage}{.45\linewidth}
		\cmhlistingsfromfile{demonstrations/textwrap1-mod1.tex}{\texttt{textwrap1-mod1.tex}}{lst:textwrap1-mod1}
	\end{minipage}
	\hfill
	\begin{minipage}{.45\linewidth}
		\cmhlistingsfromfile{demonstrations/textwrap1.yaml}[MLB-TCB]{\texttt{textwrap1.yaml}}{lst:textwrap1-yaml}
	\end{minipage}

	The text wrapping routine is performed \emph{after} verbatim environments have been stored, so verbatim
	environments and verbatim commands are exempt from the routine. For example, using the file in
	\cref{lst:textwrap2},
	\begin{widepage}
		\cmhlistingsfromfile{demonstrations/textwrap2.tex}{\texttt{textwrap2.tex}}{lst:textwrap2}
	\end{widepage}
	and running the following command and continuing to use \texttt{textwrap1.yaml} from \cref{lst:textwrap1-yaml},
	\begin{commandshell}
latexindent.pl -m textwrap2.tex -o textwrap2-mod1.tex -l textwrap1.yaml
\end{commandshell}
	then the output is as in \cref{lst:textwrap2-mod1}.
	\begin{widepage}
		\cmhlistingsfromfile{demonstrations/textwrap2-mod1.tex}{\texttt{textwrap2-mod1.tex}}{lst:textwrap2-mod1}
	\end{widepage}
	Furthermore, the text wrapping routine is performed after the trailing comments have been
	stored, and they are also exempt from text wrapping. For example, using the file in \cref{lst:textwrap3}
	\begin{widepage}
		\cmhlistingsfromfile{demonstrations/textwrap3.tex}{\texttt{textwrap3.tex}}{lst:textwrap3}
	\end{widepage}
	and running the following command and continuing to use \texttt{textwrap1.yaml} from \cref{lst:textwrap1-yaml},
	\begin{commandshell}
latexindent.pl -m textwrap3.tex -o textwrap3-mod1.tex -l textwrap1.yaml
\end{commandshell}
	then the output is as in \cref{lst:textwrap3-mod1}.

	\cmhlistingsfromfile{demonstrations/textwrap3-mod1.tex}{\texttt{textwrap3-mod1.tex}}{lst:textwrap3-mod1}

	\begin{wrapfigure}[6]{r}[0pt]{8cm}
		\cmhlistingsfromfile[style=textWrapOptionsAll]{../defaultSettings.yaml}[MLB-TCB,width=.85\linewidth,before=\centering]{\texttt{textWrapOptions}}{lst:textWrapOptionsAll}
	\end{wrapfigure}
	The text wrapping routine of \texttt{latexindent.pl} is performed by the \texttt{Text::Wrap} module, which provides a
	\texttt{separator} feature to separate lines with characters other than a new line (see \cite{textwrap}). By default,
	the separator is empty (see \cref{lst:textWrapOptionsAll}) which means that a new line token will be used, but you can change it as you see fit.

	For example starting with the file in \cref{lst:textwrap4}

	\cmhlistingsfromfile{demonstrations/textwrap4.tex}{\texttt{textwrap4.tex}}{lst:textwrap4}
	and using \texttt{textwrap2.yaml} from \cref{lst:textwrap2-yaml} with the following command
	\begin{commandshell}
latexindent.pl -m textwrap4.tex -o textwrap4-mod2.tex -l textwrap2.yaml
\end{commandshell}
	then we obtain the output in \cref{lst:textwrap4-mod2}.

	\begin{minipage}{.45\linewidth}
		\cmhlistingsfromfile{demonstrations/textwrap4-mod2.tex}{\texttt{textwrap4-mod2.tex}}{lst:textwrap4-mod2}
	\end{minipage}
	\hfill
	\begin{minipage}{.45\linewidth}
		\cmhlistingsfromfile{demonstrations/textwrap2.yaml}[MLB-TCB]{\texttt{textwrap2.yaml}}{lst:textwrap2-yaml}
	\end{minipage}

	\paragraph{Summary of text wrapping}
		It is important to note the following:
		\begin{itemize}
			\item Verbatim environments (\vref{lst:verbatimEnvironments}) and verbatim commands (\vref{lst:verbatimCommands}) will \emph{not} be affected by the text wrapping routine (see \vref{lst:textwrap2-mod1});
			\item comments will \emph{not} be affected by the text wrapping routine (see \vref{lst:textwrap3-mod1});
			\item indentation is performed \emph{after} the text wrapping routine; as such, indented code
			      will likely exceed any maximum value set in the \texttt{columns} field.
		\end{itemize}

\yamltitle{removeParagraphLineBreaks}*{fields}
	When the \texttt{-m} switch is active \texttt{latexindent.pl} has the ability to remove line breaks
	from within paragraphs; the behaviour is controlled by the \texttt{removeParagraphLineBreaks} field, detailed in
	\cref{lst:removeParagraphLineBreaks}. Thank you to \cite{jowens} for shaping and assisting with the testing of this feature.

	\cmhlistingsfromfile[style=removeParagraphLineBreaks]{../defaultSettings.yaml}[MLB-TCB,width=.85\linewidth,before=\centering]{\texttt{removeParagraphLineBreaks}}{lst:removeParagraphLineBreaks}

	This routine can be turned on \emph{globally} for \emph{every} code block type known to \texttt{latexindent.pl}
	(see \vref{tab:code-blocks}) by using the \texttt{all} switch; by default, this switch is \emph{off}. Assuming
	that the \texttt{all} switch is off, then the routine can be controlled on a per-code-block-type basis, and
	within that, on a per-name basis. We will consider examples of each of these in turn, but
	before we do, let's specify what \texttt{latexindent.pl} considers as a paragraph:
	\begin{itemize}
		\item it must begin on its own line with either an alphabetic or numeric character, and not with any of the
		      code-block types detailed in \vref{tab:code-blocks};
		\item it can include line breaks, but finishes when it meets either a blank line, a \lstinline!\par!
		      command, or any of the user-specified settings in the \texttt{paragraphsStopAt} field,
		      detailed in \vref{lst:paragraphsStopAt}.
	\end{itemize}

	Let's start with the \texttt{.tex} file in \cref{lst:shortlines}, together with the YAML settings in
	\cref{lst:remove-para1-yaml}.

	\begin{minipage}{.45\linewidth}
		\cmhlistingsfromfile{demonstrations/shortlines.tex}{\texttt{shortlines.tex}}{lst:shortlines}
	\end{minipage}
	\hfill
	\begin{minipage}{.49\linewidth}
		\cmhlistingsfromfile{demonstrations/remove-para1.yaml}[MLB-TCB]{\texttt{remove-para1.yaml}}{lst:remove-para1-yaml}
	\end{minipage}

	Upon running the command
	\begin{commandshell}
latexindent.pl -m shortlines.tex -o shortlines1.tex -l remove-para1.yaml
\end{commandshell}
	then we obtain the output given in \cref{lst:shortlines1}.

	\cmhlistingsfromfile{demonstrations/shortlines1.tex}{\texttt{shortlines1.tex}}{lst:shortlines1}

	Keen readers may notice that some trailing white space must be present in the file in \cref{lst:shortlines} which
	has crept in to the output in \cref{lst:shortlines1}. This can be fixed using the YAML file in
	\vref{lst:removeTWS-before} and running, for example,
	\begin{commandshell}
latexindent.pl -m shortlines.tex -o shortlines1-tws.tex -l remove-para1.yaml,removeTWS-before.yaml  
    \end{commandshell}
	in which case the output is as in \cref{lst:shortlines1-tws}; notice that the double spaces present in \cref{lst:shortlines1}
	have been addressed.

	\cmhlistingsfromfile{demonstrations/shortlines1-tws.tex}{\texttt{shortlines1-tws.tex}}{lst:shortlines1-tws}

	Keeping with the settings in \cref{lst:remove-para1-yaml}, we note that the \texttt{all} switch applies
	to \emph{all} code block types. So, for example, let's consider the files in \cref{lst:shortlines-mand,lst:shortlines-opt}

	\begin{minipage}{.45\linewidth}
		\cmhlistingsfromfile{demonstrations/shortlines-mand.tex}{\texttt{shortlines-mand.tex}}{lst:shortlines-mand}
	\end{minipage}
	\hfill
	\begin{minipage}{.45\linewidth}
		\cmhlistingsfromfile{demonstrations/shortlines-opt.tex}{\texttt{shortlines-opt.tex}}{lst:shortlines-opt}
	\end{minipage}

	Upon running the commands
	\begin{widepage}
		\begin{commandshell}
latexindent.pl -m shortlines-mand.tex -o shortlines-mand1.tex -l remove-para1.yaml
latexindent.pl -m shortlines-opt.tex -o shortlines-opt1.tex -l remove-para1.yaml
\end{commandshell}
	\end{widepage}

	then we obtain the respective output given in \cref{lst:shortlines-mand1,lst:shortlines-opt1}.

	\cmhlistingsfromfile{demonstrations/shortlines-mand1.tex}{\texttt{shortlines-mand1.tex}}{lst:shortlines-mand1}
	\cmhlistingsfromfile{demonstrations/shortlines-opt1.tex}{\texttt{shortlines-opt1.tex}}{lst:shortlines-opt1}

	Assuming that we turn \emph{off} the \texttt{all} switch (by setting it to \texttt{0}), then
	we can control the behaviour of \texttt{removeParagraphLineBreaks} either on a per-code-block-type basis,
	or on a per-name basis.

	For example, let's use the code in \cref{lst:shortlines-envs}, and consider the settings in \cref{lst:remove-para2-yaml,lst:remove-para3-yaml};
	note that in \cref{lst:remove-para2-yaml} we specify that \emph{every} environment should receive
	treatment from the routine, while in \cref{lst:remove-para3-yaml} we specify that \emph{only} the
	\texttt{one} environment should receive the treatment.

	\begin{minipage}{.45\linewidth}
		\cmhlistingsfromfile{demonstrations/shortlines-envs.tex}{\texttt{shortlines-envs.tex}}{lst:shortlines-envs}
	\end{minipage}
	\hfill
	\begin{minipage}{.49\linewidth}
		\cmhlistingsfromfile{demonstrations/remove-para2.yaml}[MLB-TCB]{\texttt{remove-para2.yaml}}{lst:remove-para2-yaml}
		\cmhlistingsfromfile{demonstrations/remove-para3.yaml}[MLB-TCB]{\texttt{remove-para3.yaml}}{lst:remove-para3-yaml}
	\end{minipage}

	Upon running the commands
	\begin{widepage}
		\begin{commandshell}
latexindent.pl -m shortlines-envs.tex -o shortlines-envs2.tex -l remove-para2.yaml
latexindent.pl -m shortlines-envs.tex -o shortlines-envs3.tex -l remove-para3.yaml
\end{commandshell}
	\end{widepage}
	then we obtain the respective output given in \cref{lst:shortlines-envs2,lst:shortlines-envs3}.

	\cmhlistingsfromfile{demonstrations/shortlines-envs2.tex}{\texttt{shortlines-envs2.tex}}{lst:shortlines-envs2}
	\cmhlistingsfromfile{demonstrations/shortlines-envs3.tex}{\texttt{shortlines-envs3.tex}}{lst:shortlines-envs3}

	The remaining code-block types can be customized in analogous ways, although note that \texttt{commands},
	\texttt{keyEqualsValuesBracesBrackets}, \texttt{namedGroupingBracesBrackets}, \texttt{UnNamedGroupingBracesBrackets}
	are controlled by the \texttt{optionalArguments} and the \texttt{mandatoryArguments}.

	The only special case is the \texttt{masterDocument} field; this is designed for `chapter'-type files that
	may contain paragraphs that are not within any other code-blocks. For example, consider the file in
	\cref{lst:shortlines-md}, with the YAML settings in \cref{lst:remove-para4-yaml}.

	\begin{minipage}{.45\linewidth}
		\cmhlistingsfromfile{demonstrations/shortlines-md.tex}{\texttt{shortlines-md.tex}}{lst:shortlines-md}
	\end{minipage}
	\hfill
	\begin{minipage}{.49\linewidth}
		\cmhlistingsfromfile{demonstrations/remove-para4.yaml}[MLB-TCB]{\texttt{remove-para4.yaml}}{lst:remove-para4-yaml}
	\end{minipage}

	Upon running the following command
	\begin{widepage}
		\begin{commandshell}
latexindent.pl -m shortlines-md.tex -o shortlines-md4.tex -l remove-para4.yaml
\end{commandshell}
	\end{widepage}
	then we obtain the output in \cref{lst:shortlines-md4}.
	\cmhlistingsfromfile{demonstrations/shortlines-md4.tex}{\texttt{shortlines-md4.tex}}{lst:shortlines-md4}

\yamltitle{paragraphsStopAt}*{fields}
	The paragraph line break routine considers blank lines and the \lstinline|\par| command to be the end of a paragraph;
	you can fine tune the behaviour of the routine further by using the \texttt{paragraphsStopAt} fields, shown in \cref{lst:paragraphsStopAt}.

	\cmhlistingsfromfile[style=paragraphsStopAt]{../defaultSettings.yaml}[MLB-TCB,width=.85\linewidth,before=\centering]{\texttt{paragraphsStopAt}}{lst:paragraphsStopAt}

	The fields specified in \texttt{paragraphsStopAt} tell \texttt{latexindent.pl} to stop the current paragraph
	when it reaches a line that \emph{begins} with any of the code-block types specified as \texttt{1} in \cref{lst:paragraphsStopAt}.
	By default, you'll see that the paragraph line break routine will stop when it reaches an environment at the
	beginning of a line. It is \emph{not} possible to specify these fields on a per-name basis.

	Let's use the \texttt{.tex} file in \cref{lst:sl-stop}; we will, in turn, consider the settings in
	\cref{lst:stop-command-yaml,lst:stop-comment-yaml}.

	\begin{minipage}{.45\linewidth}
		\cmhlistingsfromfile{demonstrations/sl-stop.tex}{\texttt{sl-stop.tex}}{lst:sl-stop}
	\end{minipage}
	\hfill
	\begin{minipage}{.49\linewidth}
		\cmhlistingsfromfile{demonstrations/stop-command.yaml}[MLB-TCB]{\texttt{stop-command.yaml}}{lst:stop-command-yaml}

		\cmhlistingsfromfile{demonstrations/stop-comment.yaml}[MLB-TCB]{\texttt{stop-comment.yaml}}{lst:stop-comment-yaml}
	\end{minipage}

	Upon using the settings from \vref{lst:remove-para4-yaml} and running the commands
	\begin{widepage}
		\begin{commandshell}
latexindent.pl -m sl-stop.tex -o sl-stop4.tex -l remove-para4.yaml
latexindent.pl -m sl-stop.tex -o sl-stop4-command.tex -l=remove-para4.yaml,stop-command.yaml
latexindent.pl -m sl-stop.tex -o sl-stop4-comment.tex -l=remove-para4.yaml,stop-comment.yaml
    \end{commandshell}
	\end{widepage}
	we obtain the respective outputs in \crefrange{lst:sl-stop4}{lst:sl-stop4-comment}; notice in particular that:
	\begin{itemize}
		\item in \cref{lst:sl-stop4} the paragraph line break routine has included commands and comments;
		\item in \cref{lst:sl-stop4-command} the paragraph line break routine has \emph{stopped} at the
		      \texttt{emph} command, because in \cref{lst:stop-command-yaml} we have specified \texttt{commands}
		      to be \texttt{1}, and \texttt{emph} is at the beginning of a line;
		\item in \cref{lst:sl-stop4-comment} the paragraph line break routine has \emph{stopped}
		      at the comments, because in \cref{lst:stop-comment-yaml} we have specified \texttt{comments}
		      to be \texttt{1}, and the comment is at the beginning of a line.
	\end{itemize}
	In all outputs in \crefrange{lst:sl-stop4}{lst:sl-stop4-comment} we notice that the paragraph line break
	routine has stopped at \lstinline!\begin{myenv}! because, by default, \texttt{environments}
	is set to \texttt{1} in \vref{lst:paragraphsStopAt}.

	\cmhlistingsfromfile{demonstrations/sl-stop4.tex}{\texttt{sl-stop4.tex}}{lst:sl-stop4}
	\cmhlistingsfromfile{demonstrations/sl-stop4-command.tex}{\texttt{sl-stop4-command.tex}}{lst:sl-stop4-command}
	\cmhlistingsfromfile{demonstrations/sl-stop4-comment.tex}{\texttt{sl-stop4-comment.tex}}{lst:sl-stop4-comment}

\subsection{Poly-switches}
	Every other field in the \texttt{modifyLineBreaks} field uses poly-switches, and can take
	one of four integer values\footnote{You might like to associate one of the four circles in the logo with one of the four given values}:
	\begin{itemize}[font=\bfseries]
		\item[$-1$] \emph{remove mode}: line breaks before or after the \emph{<part of thing>} can be removed (assuming that \texttt{preserveBlankLines} is set to \texttt{0});
		\item[0] \emph{off mode}: line breaks will not be modified for the \emph{<part of thing>} under consideration;
		\item[1] \emph{add mode}: a line break will be added before or after the \emph{<part of thing>} under consideration, assuming that
		      there is not already a line break before or after the \emph{<part of thing>};
		\item[2] \emph{comment then add mode}: a comment symbol will be added, followed by a line break before or after the \emph{<part of thing>} under consideration, assuming that
		      there is not already a comment and line break before or after the \emph{<part of thing>}.
	\end{itemize}
	All poly-switches are \emph{off} by default; \texttt{latexindent.pl} searches first of all for per-name settings, and then followed by global per-thing settings.

\subsection{modifyLineBreaks for environments}\label{sec:modifylinebreaks-environments}
	We start by viewing a snippet of \texttt{defaultSettings.yaml} in \cref{lst:environments-mlb}; note that it contains \emph{global} settings (immediately
	after the \texttt{environments} field) and that \emph{per-name} settings are also allowed -- in the case of \cref{lst:environments-mlb}, settings
	for \texttt{equation*} have been specified. Note that all poly-switches are \emph{off} by default.

	\cmhlistingsfromfile[style=modifylinebreaksEnv]{../defaultSettings.yaml}[width=.8\linewidth,before=\centering,MLB-TCB]{\texttt{environments}}{lst:environments-mlb}

\subsubsection{Adding line breaks (poly-switches set to $1$ or $2$)}
	Let's begin with the simple example given in \cref{lst:env-mlb1-tex}; note that we have annotated key parts of the file using $\BeginStartsOnOwnLine$,
	$\BodyStartsOnOwnLine$, $\EndStartsOnOwnLine$ and $\EndFinishesWithLineBreak$, these will be related to fields specified in \cref{lst:environments-mlb}.

	\begin{cmhlistings}[escapeinside={(*@}{@*)}]{\texttt{env-mlb1.tex}}{lst:env-mlb1-tex}
before words(*@$\BeginStartsOnOwnLine$@*) \begin{myenv}(*@$\BodyStartsOnOwnLine$@*)body of myenv(*@$\EndStartsOnOwnLine$@*)\end{myenv}(*@$\EndFinishesWithLineBreak$@*) after words
\end{cmhlistings}

	Let's explore \texttt{BeginStartsOnOwnLine} and \texttt{BodyStartsOnOwnLine} in \cref{lst:env-mlb1,lst:env-mlb2}, and in particular,
	let's allow each of them in turn to take a value of $1$.

	\begin{minipage}{.45\textwidth}
		\cmhlistingsfromfile[style=yaml-LST]{demonstrations/env-mlb1.yaml}[MLB-TCB]{\texttt{env-mlb1.yaml}}{lst:env-mlb1}
	\end{minipage}
	\hfill
	\begin{minipage}{.45\textwidth}
		\cmhlistingsfromfile[style=yaml-LST]{demonstrations/env-mlb2.yaml}[MLB-TCB]{\texttt{env-mlb2.yaml}}{lst:env-mlb2}
	\end{minipage}

	After running the following commands,
	\begin{commandshell}
latexindent.pl -m env-mlb.tex -l env-mlb1.yaml
latexindent.pl -m env-mlb.tex -l env-mlb2.yaml
\end{commandshell}
	the output is as in \cref{lst:env-mlb-mod1,lst:env-mlb-mod2} respectively.

	\begin{widepage}
		\begin{minipage}{.57\linewidth}
			\cmhlistingsfromfile{demonstrations/env-mlb-mod1.tex}{\texttt{env-mlb.tex} using \cref{lst:env-mlb1}}{lst:env-mlb-mod1}
		\end{minipage}
		\hfill
		\begin{minipage}{.42\linewidth}
			\cmhlistingsfromfile{demonstrations/env-mlb-mod2.tex}{\texttt{env-mlb.tex} using \cref{lst:env-mlb2}}{lst:env-mlb-mod2}
		\end{minipage}
	\end{widepage}

	There are a couple of points to note:
	\begin{itemize}
		\item in \cref{lst:env-mlb-mod1} a line break has been added at the point denoted by $\BeginStartsOnOwnLine$ in \cref{lst:env-mlb1-tex}; no
		      other line breaks have been changed;
		\item in \cref{lst:env-mlb-mod2} a line break has been added at the point denoted by $\BodyStartsOnOwnLine$ in \cref{lst:env-mlb1-tex};
		      furthermore, note that the \emph{body} of \texttt{myenv} has received the appropriate (default) indentation.
	\end{itemize}

	Let's now change each of the \texttt{1} values in \cref{lst:env-mlb1,lst:env-mlb2} so that they are $2$ and
	save them into \texttt{env-mlb3.yaml} and \texttt{env-mlb4.yaml} respectively (see \cref{lst:env-mlb3,lst:env-mlb4}).

	\begin{minipage}{.45\textwidth}
		\cmhlistingsfromfile[style=yaml-LST]{demonstrations/env-mlb3.yaml}[MLB-TCB]{\texttt{env-mlb3.yaml}}{lst:env-mlb3}
	\end{minipage}
	\hfill
	\begin{minipage}{.45\textwidth}
		\cmhlistingsfromfile[style=yaml-LST]{demonstrations/env-mlb4.yaml}[MLB-TCB]{\texttt{env-mlb4.yaml}}{lst:env-mlb4}
	\end{minipage}

	Upon running  commands analogous to the above, we obtain \cref{lst:env-mlb-mod3,lst:env-mlb-mod4}.

	\begin{widepage}
		\begin{minipage}{.57\linewidth}
			\cmhlistingsfromfile{demonstrations/env-mlb-mod3.tex}{\texttt{env-mlb.tex} using \cref{lst:env-mlb3}}{lst:env-mlb-mod3}
		\end{minipage}
		\hfill
		\begin{minipage}{.42\linewidth}
			\cmhlistingsfromfile{demonstrations/env-mlb-mod4.tex}{\texttt{env-mlb.tex} using \cref{lst:env-mlb4}}{lst:env-mlb-mod4}
		\end{minipage}
	\end{widepage}

	Note that line breaks have been added as in \cref{lst:env-mlb-mod1,lst:env-mlb-mod2}, but this time a comment symbol
	has been added before adding the line break; in both cases, trailing horizontal
	space has been stripped before doing so.

	Let's explore \texttt{EndStartsOnOwnLine} and \texttt{EndFinishesWithLineBreak} in \cref{lst:env-mlb5,lst:env-mlb6},
	and in particular, let's allow each of them in turn to take a value of $1$.

	\begin{minipage}{.49\textwidth}
		\cmhlistingsfromfile[style=yaml-LST]{demonstrations/env-mlb5.yaml}[MLB-TCB]{\texttt{env-mlb5.yaml}}{lst:env-mlb5}
	\end{minipage}
	\hfill
	\begin{minipage}{.49\textwidth}
		\cmhlistingsfromfile[style=yaml-LST]{demonstrations/env-mlb6.yaml}[MLB-TCB]{\texttt{env-mlb6.yaml}}{lst:env-mlb6}
	\end{minipage}

	After running the following commands,
	\begin{commandshell}
latexindent.pl -m env-mlb.tex -l env-mlb5.yaml
latexindent.pl -m env-mlb.tex -l env-mlb6.yaml
\end{commandshell}
	the output is as in \cref{lst:env-mlb-mod5,lst:env-mlb-mod6}.

	\begin{widepage}
		\begin{minipage}{.42\linewidth}
			\cmhlistingsfromfile{demonstrations/env-mlb-mod5.tex}{\texttt{env-mlb.tex} using \cref{lst:env-mlb5}}{lst:env-mlb-mod5}
		\end{minipage}
		\hfill
		\begin{minipage}{.57\linewidth}
			\cmhlistingsfromfile{demonstrations/env-mlb-mod6.tex}{\texttt{env-mlb.tex} using \cref{lst:env-mlb6}}{lst:env-mlb-mod6}
		\end{minipage}
	\end{widepage}

	There are a couple of points to note:
	\begin{itemize}
		\item in \cref{lst:env-mlb-mod5} a line break has been added at the point denoted by $\EndStartsOnOwnLine$ in \vref{lst:env-mlb1-tex}; no
		      other line breaks have been changed and the \lstinline!\end{myenv}! statement has \emph{not} received indentation (as intended);
		\item in \cref{lst:env-mlb-mod6} a line break has been added at the point denoted by $\EndFinishesWithLineBreak$ in \vref{lst:env-mlb1-tex}.
	\end{itemize}

	Let's now change each of the \texttt{1} values in \cref{lst:env-mlb5,lst:env-mlb6} so that they are $2$ and
	save them into \texttt{env-mlb7.yaml} and \texttt{env-mlb8.yaml} respectively (see \cref{lst:env-mlb7,lst:env-mlb8}).

	\begin{minipage}{.49\textwidth}
		\cmhlistingsfromfile[style=yaml-LST]{demonstrations/env-mlb7.yaml}[MLB-TCB]{\texttt{env-mlb7.yaml}}{lst:env-mlb7}
	\end{minipage}
	\hfill
	\begin{minipage}{.49\textwidth}
		\cmhlistingsfromfile[style=yaml-LST]{demonstrations/env-mlb8.yaml}[MLB-TCB]{\texttt{env-mlb8.yaml}}{lst:env-mlb8}
	\end{minipage}

	Upon running  commands analogous to the above, we obtain \cref{lst:env-mlb-mod7,lst:env-mlb-mod8}.

	\begin{widepage}
		\begin{minipage}{.42\linewidth}
			\cmhlistingsfromfile{demonstrations/env-mlb-mod7.tex}{\texttt{env-mlb.tex} using \cref{lst:env-mlb7}}{lst:env-mlb-mod7}
		\end{minipage}
		\hfill
		\begin{minipage}{.57\linewidth}
			\cmhlistingsfromfile{demonstrations/env-mlb-mod8.tex}{\texttt{env-mlb.tex} using \cref{lst:env-mlb8}}{lst:env-mlb-mod8}
		\end{minipage}
	\end{widepage}

	Note that line breaks have been added as in \cref{lst:env-mlb-mod5,lst:env-mlb-mod6}, but this time a comment symbol
	has been added before adding the line break; in both cases, trailing horizontal
	space has been stripped before doing so.

	If you ask \texttt{latexindent.pl} to add a line break (possibly with a comment) using a poly-switch value of $1$ (or $2$),
	it will only do so if necessary. For example, if you process the file in \vref{lst:mlb2} using any of the YAML
	files presented so far in this section, it will be left unchanged.

	\begin{minipage}{.45\linewidth}
		\cmhlistingsfromfile{demonstrations/env-mlb2.tex}{\texttt{env-mlb2.tex}}{lst:mlb2}
	\end{minipage}
	\hfill
	\begin{minipage}{.45\linewidth}
		\cmhlistingsfromfile{demonstrations/env-mlb3.tex}{\texttt{env-mlb3.tex}}{lst:mlb3}
	\end{minipage}

	In contrast, the output from processing the file in \cref{lst:mlb3} will vary depending
	on the poly-switches used; in \cref{lst:env-mlb3-mod2} you'll see that the comment symbol after
	the \lstinline!\begin{myenv}! has been moved to the next line, as \texttt{BodyStartsOnOwnLine}
	is set to \texttt{1}. In \cref{lst:env-mlb3-mod4} you'll see that the comment has been accounted
	for correctly because \texttt{BodyStartsOnOwnLine} has been set to \texttt{2},
	and the comment symbol has \emph{not} been moved to its own line. You're encouraged to experiment
	with \cref{lst:mlb3} and by setting the other poly-switches considered so far to \texttt{2} in turn.

	\begin{minipage}{.45\linewidth}
		\cmhlistingsfromfile{demonstrations/env-mlb3-mod2.tex}{\texttt{env-mlb3.tex} using \vref{lst:env-mlb2}}{lst:env-mlb3-mod2}
	\end{minipage}
	\hfill
	\begin{minipage}{.45\linewidth}
		\cmhlistingsfromfile{demonstrations/env-mlb3-mod4.tex}{\texttt{env-mlb3.tex} using \vref{lst:env-mlb4}}{lst:env-mlb3-mod4}
	\end{minipage}

	The details of the discussion in this section have concerned \emph{global} poly-switches in the \texttt{environments} field;
	each switch can also be specified on a \emph{per-name} basis, which would take priority over the global values; with
	reference to \vref{lst:environments-mlb}, an example is shown for the \texttt{equation*} environment.

\subsubsection{Removing line breaks (poly-switches set to $-1$)}
	Setting poly-switches to $-1$ tells \texttt{latexindent.pl} to remove line breaks of the \emph{<part of the thing>}, if necessary. We will consider the
	example code given in \cref{lst:mlb4}, noting in particular the positions of
	the line break highlighters, $\BeginStartsOnOwnLine$, $\BodyStartsOnOwnLine$, $\EndStartsOnOwnLine$
	and $\EndFinishesWithLineBreak$, together with the associated YAML files in \crefrange{lst:env-mlb9}{lst:env-mlb12}.

	\begin{minipage}{.45\linewidth}
		\begin{cmhlistings}[escapeinside={(*@}{@*)}]{\texttt{env-mlb4.tex}}{lst:mlb4}
before words(*@$\BeginStartsOnOwnLine$@*)
\begin{myenv}(*@$\BodyStartsOnOwnLine$@*)
body of myenv(*@$\EndStartsOnOwnLine$@*)
\end{myenv}(*@$\EndFinishesWithLineBreak$@*)
after words
\end{cmhlistings}
	\end{minipage}%
	\hfill
	\begin{minipage}{.51\textwidth}
		\cmhlistingsfromfile[style=yaml-LST]{demonstrations/env-mlb9.yaml}[MLB-TCB]{\texttt{env-mlb9.yaml}}{lst:env-mlb9}

		\cmhlistingsfromfile[style=yaml-LST]{demonstrations/env-mlb10.yaml}[MLB-TCB]{\texttt{env-mlb10.yaml}}{lst:env-mlb10}

		\cmhlistingsfromfile[style=yaml-LST]{demonstrations/env-mlb11.yaml}[MLB-TCB]{\texttt{env-mlb11.yaml}}{lst:env-mlb11}

		\cmhlistingsfromfile[style=yaml-LST]{demonstrations/env-mlb12.yaml}[MLB-TCB]{\texttt{env-mlb12.yaml}}{lst:env-mlb12}
	\end{minipage}

	After running the commands
	\begin{commandshell}
latexindent.pl -m env-mlb4.tex -l env-mlb9.yaml
latexindent.pl -m env-mlb4.tex -l env-mlb10.yaml
latexindent.pl -m env-mlb4.tex -l env-mlb11.yaml
latexindent.pl -m env-mlb4.tex -l env-mlb12.yaml
\end{commandshell}

	we obtain the respective output in \crefrange{lst:env-mlb4-mod9}{lst:env-mlb4-mod12}.

	\begin{minipage}{.45\linewidth}
		\cmhlistingsfromfile{demonstrations/env-mlb4-mod9.tex}{\texttt{env-mlb4.tex} using \cref{lst:env-mlb9}}{lst:env-mlb4-mod9}
	\end{minipage}
	\hfill
	\begin{minipage}{.45\linewidth}
		\cmhlistingsfromfile{demonstrations/env-mlb4-mod10.tex}{\texttt{env-mlb4.tex} using \cref{lst:env-mlb10}}{lst:env-mlb4-mod10}
	\end{minipage}

	\begin{minipage}{.45\linewidth}
		\cmhlistingsfromfile{demonstrations/env-mlb4-mod11.tex}{\texttt{env-mlb4.tex} using \cref{lst:env-mlb11}}{lst:env-mlb4-mod11}
	\end{minipage}
	\hfill
	\begin{minipage}{.45\linewidth}
		\cmhlistingsfromfile{demonstrations/env-mlb4-mod12.tex}{\texttt{env-mlb4.tex} using \cref{lst:env-mlb12}}{lst:env-mlb4-mod12}
	\end{minipage}

	Notice that in
	\begin{itemize}
		\item \cref{lst:env-mlb4-mod9} the line break denoted by $\BeginStartsOnOwnLine$ in \cref{lst:mlb4} has been removed;
		\item \cref{lst:env-mlb4-mod10} the line break denoted by $\BodyStartsOnOwnLine$ in \cref{lst:mlb4} has been removed;
		\item \cref{lst:env-mlb4-mod11} the line break denoted by $\EndStartsOnOwnLine$ in \cref{lst:mlb4} has been removed;
		\item \cref{lst:env-mlb4-mod12} the line break denoted by $\EndFinishesWithLineBreak$ in \cref{lst:mlb4} has been removed.
	\end{itemize}
	We examined each of these cases separately for clarity of explanation, but you can combine all of the YAML
	settings in \crefrange{lst:env-mlb9}{lst:env-mlb12} into one file; alternatively, you could tell \texttt{latexindent.pl}
	to load them all by using the following command, for example
	\begin{widepage}
		\begin{commandshell}
latexindent.pl -m env-mlb4.tex -l env-mlb9.yaml,env-mlb10.yaml,env-mlb11.yaml,env-mlb12.yaml
\end{commandshell}
	\end{widepage}
	which gives the output in \vref{lst:env-mlb1-tex}.

	\paragraph{About trailing horizontal space}
		Recall that on \cpageref{yaml:removeTrailingWhitespace} we discussed the YAML field \texttt{removeTrailingWhitespace},
		and that it has two (binary) switches to determine if horizontal space should be removed \texttt{beforeProcessing} and \texttt{afterProcessing}.
		The \texttt{beforeProcessing} is particularly relevant when considering the \texttt{-m} switch; let's consider the
		file shown in \cref{lst:mlb5}, which highlights trailing spaces.

		\begin{minipage}{.45\linewidth}
			\begin{cmhlistings}[showspaces=true,escapeinside={(*@}{@*)}]{\texttt{env-mlb5.tex}}{lst:mlb5}
before words   (*@$\BeginStartsOnOwnLine$@*) 
\begin{myenv}           (*@$\BodyStartsOnOwnLine$@*)
body of myenv      (*@$\EndStartsOnOwnLine$@*) 
\end{myenv}     (*@$\EndFinishesWithLineBreak$@*)
after words
\end{cmhlistings}
		\end{minipage}
		\hfill
		\begin{minipage}{.45\linewidth}
			\cmhlistingsfromfile[style=yaml-LST]{demonstrations/removeTWS-before.yaml}[yaml-TCB]{\texttt{removeTWS-before.yaml}}{lst:removeTWS-before}
		\end{minipage}

		The output from the following commands
		\begin{widepage}
			\begin{commandshell}
latexindent.pl -m env-mlb5.tex -l env-mlb9.yaml,env-mlb10.yaml,env-mlb11.yaml,env-mlb12.yaml
latexindent.pl -m env-mlb5.tex -l env-mlb9.yaml,env-mlb10.yaml,env-mlb11.yaml,env-mlb12.yaml,removeTWS-before.yaml
\end{commandshell}
		\end{widepage}
		is shown, respectively, in \cref{lst:env-mlb5-modAll,lst:env-mlb5-modAll-remove-WS}; note that
		the trailing horizontal white space has been preserved (by default) in \cref{lst:env-mlb5-modAll}, while
		in \cref{lst:env-mlb5-modAll-remove-WS}, it has been removed using the switch specified in \cref{lst:removeTWS-before}.

		\begin{widepage}
			\cmhlistingsfromfile{demonstrations/env-mlb5-modAll.tex}{\texttt{env-mlb5.tex} using \crefrange{lst:env-mlb4-mod9}{lst:env-mlb4-mod12}}{lst:env-mlb5-modAll}

			\cmhlistingsfromfile{demonstrations/env-mlb5-modAll-remove-WS.tex}{\texttt{env-mlb5.tex} using \crefrange{lst:env-mlb4-mod9}{lst:env-mlb4-mod12} \emph{and} \cref{lst:removeTWS-before}}{lst:env-mlb5-modAll-remove-WS}
		\end{widepage}

	\paragraph{Blank lines}
		Now let's consider the file in \cref{lst:mlb6}, which contains blank lines.

		\begin{minipage}{.45\linewidth}
			\begin{cmhlistings}[escapeinside={(*@}{@*)}]{\texttt{env-mlb6.tex}}{lst:mlb6}
before words(*@$\BeginStartsOnOwnLine$@*)


\begin{myenv}(*@$\BodyStartsOnOwnLine$@*)


body of myenv(*@$\EndStartsOnOwnLine$@*)


\end{myenv}(*@$\EndFinishesWithLineBreak$@*)

after words
\end{cmhlistings}
		\end{minipage}%
		\hfill
		\begin{minipage}{.45\linewidth}
			\cmhlistingsfromfile[style=yaml-LST]{demonstrations/UnpreserveBlankLines.yaml}[MLB-TCB]{\texttt{UnpreserveBlankLines.yaml}}{lst:UnpreserveBlankLines}
		\end{minipage}

		Upon running the following commands
		\begin{widepage}
			\begin{commandshell}
latexindent.pl -m env-mlb6.tex -l env-mlb9.yaml,env-mlb10.yaml,env-mlb11.yaml,env-mlb12.yaml
latexindent.pl -m env-mlb6.tex -l env-mlb9.yaml,env-mlb10.yaml,env-mlb11.yaml,env-mlb12.yaml,UnpreserveBlankLines.yaml
\end{commandshell}
		\end{widepage}
		we receive the respective outputs in \cref{lst:env-mlb6-modAll,lst:env-mlb6-modAll-un-Preserve-Blank-Lines}. In
		\cref{lst:env-mlb6-modAll} we see that the multiple blank lines have each been condensed into one blank line,
		but that blank lines have \emph{not} been removed by the poly-switches -- this is because, by default, \texttt{preserveBlankLines}
		is set to \texttt{1}. By contrast, in \cref{lst:env-mlb6-modAll-un-Preserve-Blank-Lines}, we have allowed
		the poly-switches to remove blank lines because, in \cref{lst:UnpreserveBlankLines}, we have set \texttt{preserveBlankLines} to \texttt{0}.

		\begin{widepage}
			\begin{minipage}{.30\linewidth}
				\cmhlistingsfromfile{demonstrations/env-mlb6-modAll.tex}{\texttt{env-mlb6.tex} using \crefrange{lst:env-mlb4-mod9}{lst:env-mlb4-mod12}}{lst:env-mlb6-modAll}
			\end{minipage}
			\hfill
			\begin{minipage}{.65\linewidth}
				\cmhlistingsfromfile{demonstrations/env-mlb6-modAll-un-Preserve-Blank-Lines.tex}{\texttt{env-mlb6.tex} using \crefrange{lst:env-mlb4-mod9}{lst:env-mlb4-mod12} \emph{and} \cref{lst:UnpreserveBlankLines}}{lst:env-mlb6-modAll-un-Preserve-Blank-Lines}
			\end{minipage}
		\end{widepage}

\subsection{Poly-switches for other code blocks}
	Rather than repeat the examples shown for the environment code blocks (in \vref{sec:modifylinebreaks-environments}), we choose to detail the poly-switches for
	all other code blocks in \cref{tab:poly-switch-mapping}; note that each and every one of these poly-switches is \emph{off by default}, i.e, set to \texttt{0}. Note also that,
	by design, line breaks involving \texttt{verbatim}, \texttt{filecontents} and `comment-marked' code blocks (\vref{lst:alignmentmarkup}) can \emph{not} be
	modified using \texttt{latexindent.pl}.

	\begin{longtable}{m{.2\textwidth}@{\hspace{.75cm}}m{.35\textwidth}@{}m{.4\textwidth}}
		\caption{Poly-switch mappings for all code-block types}\label{tab:poly-switch-mapping}\\
		\toprule
		Code block & Sample & Poly-switch mapping \\
		\midrule
		environment &
		\begin{lstlisting}[escapeinside={(*@}{@*)},nolol=true]
before words(*@$\BeginStartsOnOwnLine$@*)
\begin{myenv}(*@$\BodyStartsOnOwnLine$@*)
body of myenv(*@$\EndStartsOnOwnLine$@*)
\end{myenv}(*@$\EndFinishesWithLineBreak$@*)
after words
  \end{lstlisting}
		&
		\begin{tabular}[t]{c@{~}l@{}}
			$\BeginStartsOnOwnLine$     & BeginStartsOnOwnLine     \\
			$\BodyStartsOnOwnLine$      & BodyStartsOnOwnLine      \\
			$\EndStartsOnOwnLine$       & EndStartsOnOwnLine       \\
			$\EndFinishesWithLineBreak$ & EndFinishesWithLineBreak \\
		\end{tabular}
		\\
		\cmidrule{2-3}
		ifelsefi &
		\begin{lstlisting}[escapeinside={(*@}{@*)},nolol=true]
before words(*@$\BeginStartsOnOwnLine$@*)
\if...(*@$\BodyStartsOnOwnLine$@*)
body of if statement(*@$\ElseStartsOnOwnLine$@*)
\else(*@$\ElseFinishesWithLineBreak$@*)
body of else statement(*@$\EndStartsOnOwnLine$@*)
\fi(*@$\EndFinishesWithLineBreak$@*)
after words
  \end{lstlisting}
		&
		\begin{tabular}[t]{c@{~}l@{}}
			$\BeginStartsOnOwnLine$      & IfStartsOnOwnLine         \\
			$\BodyStartsOnOwnLine$       & BodyStartsOnOwnLine       \\
			$\ElseStartsOnOwnLine$       & ElseStartsOnOwnLine       \\
			$\ElseFinishesWithLineBreak$ & ElseFinishesWithLineBreak \\
			$\EndStartsOnOwnLine$        & FiStartsOnOwnLine         \\
			$\EndFinishesWithLineBreak$  & FiFinishesWithLineBreak   \\
		\end{tabular}
		\\
		\cmidrule{2-3}
		optionalArguments &
		\begin{lstlisting}[escapeinside={(*@}{@*)},nolol=true]
...(*@$\BeginStartsOnOwnLine$@*)
[(*@$\BodyStartsOnOwnLine$@*)
body of opt arg(*@$\EndStartsOnOwnLine$@*)
](*@$\EndFinishesWithLineBreak$@*)
...
  \end{lstlisting}
		&
		\begin{tabular}[t]{c@{~}l@{}}
			$\BeginStartsOnOwnLine$     & LSqBStartsOnOwnLine\footnote{LSqB stands for Left Square Bracket} \\
			$\BodyStartsOnOwnLine$      & OptArgBodyStartsOnOwnLine                                         \\
			$\EndStartsOnOwnLine$       & RSqBStartsOnOwnLine                                               \\
			$\EndFinishesWithLineBreak$ & RSqBFinishesWithLineBreak                                         \\
		\end{tabular}
		\\
		\cmidrule{2-3}
		mandatoryArguments &
		\begin{lstlisting}[escapeinside={(*@}{@*)},nolol=true]
...(*@$\BeginStartsOnOwnLine$@*)
{(*@$\BodyStartsOnOwnLine$@*)
body of mand arg(*@$\EndStartsOnOwnLine$@*)
}(*@$\EndFinishesWithLineBreak$@*)
...
  \end{lstlisting}
		&
		\begin{tabular}[t]{c@{~}l@{}}
			$\BeginStartsOnOwnLine$     & LCuBStartsOnOwnLine\footnote{LCuB stands for Left Curly Brace} \\
			$\BodyStartsOnOwnLine$      & MandArgBodyStartsOnOwnLine                                     \\
			$\EndStartsOnOwnLine$       & RCuBStartsOnOwnLine                                            \\
			$\EndFinishesWithLineBreak$ & RCuBFinishesWithLineBreak                                      \\
		\end{tabular}
		\\
		\cmidrule{2-3}
		commands &
		\begin{lstlisting}[escapeinside={(*@}{@*)},morekeywords={mycommand},nolol=true,]
before words(*@$\BeginStartsOnOwnLine$@*)
\mycommand(*@$\BodyStartsOnOwnLine$@*)
(*@$\langle$\itshape{arguments}$\rangle$@*)
  \end{lstlisting}
		&
		\begin{tabular}[t]{c@{~}l@{}}
			$\BeginStartsOnOwnLine$ & CommandStartsOnOwnLine           \\
			$\BodyStartsOnOwnLine$  & CommandNameFinishesWithLineBreak \\
		\end{tabular}
		\\
		\cmidrule{2-3}
		namedGroupingBraces Brackets &
		\begin{lstlisting}[escapeinside={(*@}{@*)},morekeywords={myname},nolol=true,]
before words(*@$\BeginStartsOnOwnLine$@*)
myname(*@$\BodyStartsOnOwnLine$@*)
(*@$\langle$\itshape{braces/brackets}$\rangle$@*)
  \end{lstlisting}
		&
		\begin{tabular}[t]{c@{~}l@{}}
			$\BeginStartsOnOwnLine$ & NameStartsOnOwnLine       \\
			$\BodyStartsOnOwnLine$  & NameFinishesWithLineBreak \\
		\end{tabular}
		\\
		\cmidrule{2-3}
		keyEqualsValuesBraces\newline Brackets &
		\begin{lstlisting}[escapeinside={(*@}{@*)},morekeywords={key},nolol=true,]
before words(*@$\BeginStartsOnOwnLine$@*)
key(*@$\EqualsStartsOnOwnLine$@*)=(*@$\BodyStartsOnOwnLine$@*)
(*@$\langle$\itshape{braces/brackets}$\rangle$@*)
  \end{lstlisting}
		&
		\begin{tabular}[t]{c@{~}l@{}}
			$\BeginStartsOnOwnLine$  & KeyStartsOnOwnLine          \\
			$\EqualsStartsOnOwnLine$ & EqualsStartsOnOwnLine       \\
			$\BodyStartsOnOwnLine$   & EqualsFinishesWithLineBreak \\
		\end{tabular}
		\\
		\cmidrule{2-3}
		items &
		\begin{lstlisting}[escapeinside={(*@}{@*)},nolol=true]
before words(*@$\BeginStartsOnOwnLine$@*)
\item(*@$\BodyStartsOnOwnLine$@*)
...
  \end{lstlisting}
		&
		\begin{tabular}[t]{c@{~}l@{}}
			$\BeginStartsOnOwnLine$ & ItemStartsOnOwnLine       \\
			$\BodyStartsOnOwnLine$  & ItemFinishesWithLineBreak \\
		\end{tabular}
		\\
		\cmidrule{2-3}
		specialBeginEnd &
		\begin{lstlisting}[escapeinside={(*@}{@*)},nolol=true]
before words(*@$\BeginStartsOnOwnLine$@*)
\[(*@$\BodyStartsOnOwnLine$@*)
body of special(*@$\EndStartsOnOwnLine$@*)
\](*@$\EndFinishesWithLineBreak$@*)
after words
  \end{lstlisting}
		&
		\begin{tabular}[t]{c@{~}l@{}}
			$\BeginStartsOnOwnLine$     & SpecialBeginStartsOnOwnLine     \\
			$\BodyStartsOnOwnLine$      & SpecialBodyStartsOnOwnLine      \\
			$\EndStartsOnOwnLine$       & SpecialEndStartsOnOwnLine       \\
			$\EndFinishesWithLineBreak$ & SpecialEndFinishesWithLineBreak \\
		\end{tabular}
		\\
		\bottomrule
	\end{longtable}

%% arara: pdflatex: { files: [latexindent]}
\section{indentconfig.yaml, local settings and the -y switch }\label{sec:indentconfig}
 The behaviour of \texttt{latexindent.pl} is controlled from the settings specified in any
 of the YAML files that you tell it to load. By default, \texttt{latexindent.pl} will only
 load \texttt{defaultSettings.yaml}, but there are a few ways that you can tell it to load
 your own settings files.

\subsection{indentconfig.yaml and .indentconfig.yaml}
	\texttt{latexindent.pl} will always check your home directory for
	\texttt{indentconfig.yaml}
	and \texttt{.indentconfig.yaml} (unless it is called with the \texttt{-d} switch), which
	is a plain text file you can create that contains the \emph{absolute} paths for any
	settings files that you wish \texttt{latexindent.pl} to load. There is no difference
	between \texttt{indentconfig.yaml} and \texttt{.indentconfig.yaml}, other than the fact
	that \texttt{.indentconfig.yaml} is a `hidden' file; thank you to
	\cite{jacobo-diaz-hidden-config} for providing this feature. In what follows, we will use
	\texttt{indentconfig.yaml}, but it is understood that this could equally represent
	\texttt{.indentconfig.yaml}. If you have both files in existence then
	\texttt{indentconfig.yaml} takes priority.

	For Mac and Linux users, their home directory is \texttt{~/username} while Windows (Vista
	onwards) is \lstinline!C:\Users\username!\footnote{If you're not sure where to put
		\texttt{indentconfig.yaml}, don't worry \texttt{latexindent.pl} will tell you in the log
		file exactly where to put it assuming it doesn't exist already.} \Cref{lst:indentconfig}
	shows a sample \texttt{indentconfig.yaml} file.

	\begin{yaml}{\texttt{indentconfig.yaml} (sample)}{lst:indentconfig}
# Paths to user settings for latexindent.pl
#
# Note that the settings will be read in the order you
# specify here- each successive settings file will overwrite
# the variables that you specify

paths:
- /home/cmhughes/Documents/yamlfiles/mysettings.yaml
- /home/cmhughes/folder/othersettings.yaml
- /some/other/folder/anynameyouwant.yaml
- C:\Users\chughes\Documents\mysettings.yaml
- C:\Users\chughes\Desktop\test spaces\more spaces.yaml
\end{yaml}

	Note that the \texttt{.yaml} files you specify in \texttt{indentconfig.yaml} will be
	loaded in the order in which you write them. Each file doesn't have to have every switch
	from \texttt{defaultSettings.yaml}; in fact, I recommend that you only keep the switches
	that you want to \emph{change} in these settings files.

	To get started with your own settings file, you might like to save a copy of
	\texttt{defaultSettings.yaml} in another directory and call it, for example,
	\texttt{mysettings.yaml}. Once you have added the path to \texttt{indentconfig.yaml} you
	can change the switches and add more code-block names to it as you see fit -- have a look
	at \cref{lst:mysettings} for an example that uses four tabs for the default indent, adds
	the \texttt{tabbing} environment/command to the list of environments that contains
	alignment delimiters; you might also like to refer to the many YAML files detailed
	throughout the rest of this documentation.
	\index{indentation!defaultIndent using YAML file}

	\begin{yaml}{\texttt{mysettings.yaml} (example)}{lst:mysettings}
# Default value of indentation
defaultIndent: "\t\t\t\t"

# environments that have tab delimiters, add more
# as needed
lookForAlignDelims:
    tabbing: 1
\end{yaml}

	You can make sure that your settings are loaded by checking \texttt{indent.log} for
	details -- if you have specified a path that \texttt{latexindent.pl} doesn't recognise
	then you'll get a warning, otherwise you'll get confirmation that \texttt{latexindent.pl}
	has read your settings file \footnote{Windows users may find that they have to end
		\texttt{.yaml} files with a blank line}.
	\index{warning!editing YAML files}

	\begin{warning}
		When editing \texttt{.yaml} files it is \emph{extremely} important to remember how
		sensitive they are to spaces. I highly recommend copying and pasting from
		\texttt{defaultSettings.yaml} when you create your first
		\texttt{whatevernameyoulike.yaml} file.

		If \texttt{latexindent.pl} can not read your \texttt{.yaml} file it will tell you so in
		\texttt{indent.log}.
	\end{warning}

	If you find that%
	\announce{2021-06-19}{encoding option for indentconfig.yaml} \texttt{latexindent.pl} does not read your YAML file, then it
	might be as a result of the default commandline encoding not being UTF-8; normally this
	will only occcur for Windows users. In this case, you might like to explore the
	\texttt{encoding} option for \texttt{indentconfig.yaml} as demonstrated in
	\cref{lst:indentconfig-encoding}.

	\cmhlistingsfromfile{demonstrations/encoding.yaml}[yaml-TCB]{The \texttt{encoding} option for \texttt{indentconfig.yaml}}{lst:indentconfig-encoding}

	Thank you to \cite{qiancy98} for this contribution; please see \vref{app:encoding} and
	details within \cite{encoding} for further information.

\subsection{localSettings.yaml and friends}\label{sec:localsettings}
	The \texttt{-l} switch tells \texttt{latexindent.pl} to look for
	\texttt{localSettings.yaml} and/or friends in the \emph{same directory} as
	\texttt{myfile.tex}. For%
	\announce{2021-03-14}*{-l
		switch: localSettings and friends} example, if you use the following command
	\index{switches!-l demonstration}
	\begin{commandshell}
latexindent.pl -l myfile.tex
\end{commandshell}
	then \texttt{latexindent.pl} will search for and then, assuming they exist, load each of
	the following files in the following order:
	\begin{enumerate}
		\item localSettings.yaml
		\item latexindent.yaml
		\item .localSettings.yaml
		\item .latexindent.yaml
	\end{enumerate}
	These files will be assumed to be in the same directory as \texttt{myfile.tex}, or
	otherwise in the current working directory. You do not need to have all of the above
	files, usually just one will be sufficient. In what follows, whenever we refer to
	\texttt{localSettings.yaml} it is assumed that it can mean any of the four named options
	listed above.

	If you'd prefer to name your \texttt{localSettings.yaml} file something different, (say,
	\texttt{mysettings.yaml} as in \cref{lst:mysettings}) then you can call
	\texttt{latexindent.pl} using, for example,
	\begin{commandshell}
latexindent.pl -l=mysettings.yaml myfile.tex
\end{commandshell}

	Any settings file(s) specified using the \texttt{-l} switch will be read \emph{after}
	\texttt{defaultSettings.yaml} and, assuming they exist, any user setting files specified
	in \texttt{indentconfig.yaml}.

	Your settings file can contain any switches that you'd like to change; a sample is shown
	in \cref{lst:localSettings}, and you'll find plenty of further examples throughout this
	manual.
	\index{verbatim!verbatimEnvironments demonstration (-l switch)}

	\begin{yaml}{\texttt{localSettings.yaml} (example)}{lst:localSettings}
#  verbatim environments - environments specified
#  here will not be changed at all!
verbatimEnvironments:
    cmhenvironment: 0
    myenv: 1
\end{yaml}

	You can make sure that your settings file has been loaded by checking \texttt{indent.log}
	for details; if it can not be read then you receive a warning, otherwise you'll get
	confirmation that \texttt{latexindent.pl} has read your settings file.

\subsection{The -y|yaml switch}\label{sec:yamlswitch}
	You%
	\announce{2017-08-21}{demonstration of the -y switch}
	may use the \texttt{-y} switch to load your settings;  for example, if you wished to
	specify the settings from \cref{lst:localSettings} using the \texttt{-y} switch, then you
	could use the following command:
	\index{verbatim!verbatimEnvironments demonstration (-y switch)}
	\begin{commandshell}
latexindent.pl -y="verbatimEnvironments:cmhenvironment:0;myenv:1" myfile.tex
\end{commandshell}
	Note the use of \texttt{;} to specify another field within \texttt{verbatimEnvironments}.
	This is shorthand, and equivalent, to using the following command:
	\index{switches!-y demonstration}
	\begin{commandshell}
latexindent.pl -y="verbatimEnvironments:cmhenvironment:0,verbatimEnvironments:myenv:1" myfile.tex
\end{commandshell}
	You may, of course, specify settings using the \texttt{-y} switch as well as, for
	example, settings loaded using the \texttt{-l} switch; for example,
	\index{switches!-l demonstration}
	\index{switches!-y demonstration}
	\begin{commandshell}
latexindent.pl -l=mysettings.yaml -y="verbatimEnvironments:cmhenvironment:0;myenv:1" myfile.tex
\end{commandshell}
	Any settings specified using the \texttt{-y} switch will be loaded \emph{after} any
	specified using \texttt{indentconfig.yaml} and the \texttt{-l} switch.

	If you wish to specify any regex-based settings using the \texttt{-y} switch,
	\index{regular expressions!using -y switch} it is important not to use quotes surrounding
	the regex; for example, with reference to the `one sentence per line' feature
	(\vref{sec:onesentenceperline}) and the listings within \vref{lst:sentencesEndWith}, the
	following settings give the option to have sentences end with a semicolon
	\index{switches!-y demonstration}
	\begin{commandshell}
latexindent.pl -m --yaml='modifyLineBreaks:oneSentencePerLine:sentencesEndWith:other:\;'
\end{commandshell}

\subsection{Settings load order}\label{sec:loadorder}
	\texttt{latexindent.pl} loads the settings files in the following order:
	\index{switches!-l in relation to other settings}
	\begin{enumerate}
		\item \texttt{defaultSettings.yaml} is always loaded, and can not be renamed;
		\item \texttt{anyUserSettings.yaml} and any other arbitrarily-named files specified in
		      \texttt{indentconfig.yaml};
		\item \texttt{localSettings.yaml} but only if found in the same directory as
		      \texttt{myfile.tex}
		      and called with \texttt{-l} switch; this file can be renamed, provided that the call to
		      \texttt{latexindent.pl} is adjusted accordingly (see \cref{sec:localsettings}). You may
		      specify both relative and absolute%
		      \announce{2017-08-21}*{-l absolute paths} paths to other YAML files using the \texttt{-l}
		      switch, separating multiple files using commas;
		\item any settings%
		      \announce{2017-08-21}{-y switch load order}
		      specified in the \texttt{-y} switch.
	\end{enumerate}
	A visual representation of this is given in \cref{fig:loadorder}.

	\begin{figure}[!htb]
		\centering
		\documentclass{standalone}
\usepackage{tikz}
\usetikzlibrary{positioning}
\begin{document}
\begin{tikzpicture}[
		needed/.style={very thick, draw=blue,fill=blue!20, text centered, minimum height=2.5em,rounded corners=1ex},
		optional/.style={draw=black, very thick,scale=0.8, text centered, minimum height=2.5em,rounded corners=1ex},
		optionalfill/.style={fill=black!10},
		connections/.style={draw=black!30,dotted,line width=3pt,text=red},
	]
	% Draw diagram elements
	\node (latexindent) [needed,circle] {\texttt{latexindent.pl}};
	\node (default) [needed,above right=.5cm of latexindent] {\texttt{defaultSettings.yaml}};
	\node (indentconfig) [optional,right=of latexindent] {\texttt{indentconfig.yaml}};
	\node (any) [optional,optionalfill,above right=of indentconfig] {\texttt{any.yaml}};
	\node (name) [optional,optionalfill,right=of indentconfig] {\texttt{name.yaml}};
	\node (you) [optional,optionalfill,below right=of indentconfig] {\texttt{you.yaml}};
	\node (want) [optional,optionalfill,below=of indentconfig] {\texttt{want.yaml}};
	\node (local) [optional,below=of latexindent] {\texttt{localSettings.yaml}};
	\node (yamlswitch) [optional,left=of latexindent] {\texttt{-y switch}};
	% Draw arrows between elements
	\draw[connections,solid] (latexindent) to[in=-90]node[pos=0.5,anchor=north]{1} (default.south) ;
	\draw[connections,optional] (latexindent) -- node[pos=0.5,anchor=north]{2} (indentconfig) ;
	\draw[connections,optional] (indentconfig) to[in=-90] (any.south) ;
	\draw[connections,optional] (indentconfig) -- (name) ;
	\draw[connections,optional] (indentconfig) to[out=-45,in=90] (you) ;
	\draw[connections,optional] (indentconfig) -- (want) ;
	\draw[connections,optional] (latexindent) -- node[pos=0.5,anchor=west]{3} (local) ;
	\draw[connections,optional] (latexindent) -- node[pos=0.5,anchor=north]{4} (yamlswitch) ;
\end{tikzpicture}
\end{document}

		\caption{Schematic of the load order described in \cref{sec:loadorder}; solid lines represent
			mandatory files, dotted lines represent optional files. \texttt{indentconfig.yaml} can
			contain as many files as you like. The files will be loaded in order; if you specify
			settings for the same field in more than one file, the most recent takes priority. }
		\label{fig:loadorder}
	\end{figure}

\end{document}

%    # set noAdditionalIndent globally for codeblocks
%    noAdditionalIndentGlobal:
%        environments: 0
%        commands: 1
%        optionalArguments: 0
%        mandatoryArguments: 0
%        ifElseFi: 0
%        items: 0

This specifies noAdditionalIndent globally for code blocks;

%# set indentRules globally for codeblocks; these need 
%# to be horizontal spaces, if they are to be used
%indentRulesGlobal:
%    environments: 0
%    commands: 0
%    optionalArguments: 0
%    mandatoryArguments: 0
%    ifElseFi: 0
%    items: 0

The hierachy is
\begin{enumerate}
	\item noAdditionalIndent (on a per-name basis)
	\item indentRules (on a per-name basis)
	\item noAdditionalIndentGlobal
	\item indentRulesGlobal
	\item defaultIndent
\end{enumerate}

\item[\verbitem{logFilePreferences}]
\texttt{latexindent.pl} writes information to \texttt{indent.log}, some
of which can be customised by changing \texttt{logFilePreferences}; see \cref{lst:logFilePreferences}.
\begin{cmhlistings}[style=yaml]{\texttt{logFilePreferences}}{lst:logFilePreferences}
	logFilePreferences:
	showEveryYamlRead: 1
	showAmalgamatedSettings: 0
	endLogFileWith: '--------------'
\end{cmhlistings}
If you load your own user settings (see \vref{sec:indentconfig}) then \texttt{latexindent.pl} will
detail them in \texttt{indent.log}; you can choose not to have the details logged by switching
\texttt{showEveryYamlRead} to \texttt{0}. Once all of your settings have
been loaded, you can see the amalgamated settings by switching \texttt{showAmalgamatedSettings}
to \texttt{1}, if you wish. The log file will end with the characters
given in \texttt{endLogFileWith}.

\section{Known limitations}\label{sec:knownlimitations}

The following command struggles because of the []
%\strip_marks:n #1
%	{
%	\regex_replace_all:nnN { \c{marks}\cB\{ [^\}]* \cE\} } {  } \l_solution_text_tl
%	}

The following struggles:

%	\@ifnextchar[{\@assignmentwithcutoff}{\@assignmentnocutoff}

because of the unmatched [

There are a number of known limitations of the script, and almost certainly quite a
few that are \emph{unknown}!

The main limitation is to do with the alignment routine of environments that contain
delimiters--in other words, environments that are entered in \texttt{lookForAlignDelims}.
Indeed, this is the only part of the script that can \emph{potentially} remove
lines from \texttt{myfile.tex}. Note that \texttt{indent.log} will always
finish with a comparison of line counts before and after.

The routine works well for `standard' blocks of code that have the same number of \texttt{&}
per line, but it will not do anything for lines that do not--such examples
include \texttt{tabular} environments that use \texttt{\multicolumn} or
perhaps spread cell contents across multiple lines.  For each alignment block (\texttt{tabular},
\texttt{align}, etc) \texttt{latexindent.pl} first of all makes a record
of the maximum number of \texttt{&}; if each row does not have that
number of \texttt{&} then it will not try to format that row. Details
will be given in \texttt{indent.log} assuming that \texttt{trace} mode
is active.

If you have a \texttt{verbatim}-like environment inside a \texttt{tabular}-like
environment, the \texttt{verbatim} environment \emph{will} be formatted, which
is probably not what you want. I hope to address this in future versions, but for the
moment wrap it in a \texttt{noIndentBlock} (see \cpageref{lst:noIndentBlockdemo}).

You can run \texttt{latexindent} on \texttt{.sty}, \texttt{.cls} and any filetypes
that you specify in \lstinline[breaklines=true]!fileExtensionPreference! (see \vref{lst:fileExtensionPreference});
if you find a case in which the script struggles, please feel free
to report it at \cite{latexindent-home}, and
in the meantime, consider using a \texttt{noIndentBlock} (see \cpageref{lst:noIndentBlockdemo}).

I hope that this script is useful to some; if you find an example where the
script does not behave as you think it should, the best way to contact me is to
report an issue on \cite{latexindent-home}; otherwise, feel free to find me on
the \url{http://tex.stackexchange.com}.

\nocite{*}
\section{References}
\printbibliography[heading=subbibnumbered,title={External links},notkeyword=contributor]
\printbibliography[env=specialbib,heading=subbibnumbered,title={Contributors\label{sec:contributors}},keyword=contributor]

\appendix
\section{Required \texttt{Perl} modules}\label{sec:requiredmodules}
If you intend to use \texttt{latexindent.pl} and \emph{not} one of the supplied standalone executable files, then you will need a few standard Perl modules--if you can run the
minimum code in \cref{lst:helloworld} (\texttt{perl helloworld.pl}) then you will be able to run \texttt{latexindent.pl}, otherwise you may
need to install the missing modules.

\begin{cmhlistings}[language=Perl]{\texttt{helloworld.pl}}{lst:helloworld}
	#!/usr/bin/perl

	use strict;
	use warnings;
	use FindBin;
	use YAML::Tiny;
	use File::Copy;
	use File::Basename;
	use Getopt::Long;
	use File::HomeDir;

	print "hello world";
	exit;
\end{cmhlistings}
My default installation on Ubuntu 12.04 did \emph{not} come
with all of these modules as standard, but Strawberry Perl for Windows \cite{strawberryperl}
did.

Installing the modules given in \cref{lst:helloworld} will vary depending on your
operating system and \texttt{Perl} distribution. For example, Ubuntu users
might visit the software center, or else run
\begin{lstlisting}[numbers=none]
sudo perl -MCPAN -e 'install "File::HomeDir"'
\end{lstlisting}

Linux users may be interested in exploring Perlbrew \cite{perlbrew}; possible installation and setup
options follow for Ubuntu (other distributions will need slightly different commands).
\begin{commandshell}
sudo apt-get install perlbrew
perlbrew install perl-5.20.1
perlbrew switch perl-5.20.1
sudo apt-get install curl
curl -L http://cpanmin.us | perl - App::cpanminus
cpanm YAML::Tiny
cpanm File::HomeDir
\end{commandshell}

Strawberry Perl users on Windows might use
\texttt{CPAN client}. All of the modules are readily available on CPAN \cite{cpan}.

As of Version 2.1,  \texttt{indent.log} will contain details of the location
of the Perl modules on your system.  \texttt{latexindent.exe} is a standalone
executable for Windows (and therefore does not require a Perl distribution) and caches copies of the Perl modules onto your system; if you
wish to see where they are cached, use the  \texttt{trace} option, e.g  \texttt{latexindent.exe -t myfile.tex}.

\section{The \texttt{arara} rule}
The \texttt{arara} rule (\texttt{indent.yaml}) contains lines such as those
given in \cref{lst:arararule}. With this setup, the user \emph{always} has
to specify whether or not they want (in this example) to use the \texttt{trace}
identifier.
\begin{cmhlistings}[style=yaml,numbers=none]{The \texttt{arara} rule}{lst:arararule}
	...
	arguments:
	- identifier: trace
	flag: <arara> @{ isTrue( parameters.trace, "-t" ) }
	...
\end{cmhlistings}

If you would like to have the \texttt{trace} option on by default every time you
call \texttt{latexindent.pl} from \texttt{arara} (without having to write \texttt{% arara: indent: {trace: yes}}), then simply
amend \cref{lst:arararule} so that it looks like \cref{lst:arararulemod}.
\begin{cmhlistings}[style=yaml,numbers=none]{The \texttt{arara} rule (modified)}{lst:arararulemod}
	...
	arguments:
	- identifier: trace
	flag: <arara> @{ isTrue( parameters.trace, "-t" ) }
	default: "-t"
	...
\end{cmhlistings}

With this modification in place, you now simply to write \texttt{% arara: indent} and
\texttt{trace} mode will be activated by default. If you wish to turn off \texttt{trace}
mode then you can write \texttt{% arara: indent: {trace: off}}.

Of course, you can apply these types of modifications to \emph{any} of the identifiers,
but proceed with caution if you intend to do this for \texttt{overwrite}.

\section{Updating the \texttt{path} variable}\label{sec:updating-path}
\texttt{latexindent.pl} ships with a few scripts that can update the \texttt{path} variables
\footnote{Thanks to \cite{jasjuang} for this feature!}. If you're
on a Linux or Mac machine, then you'll want \texttt{CMakeLists.txt} from \cite{latexindent-home}.
\subsection{Add to path for Linux}
To add \texttt{latexindent.pl} to the path for Linux, follow these steps:
\begin{enumerate}
	\item download  \texttt{latexindent.pl}, \texttt{defaultSettings.yaml},  to your
	      chosen directory from \cite{latexindent-home} ;
	\item within your directory, create a directory called \texttt{path-helper-files} and
	      download \texttt{CMakeLists.txt} and \texttt{cmake_uninstall.cmake.in}
	      from \cite{latexindent-home}/path-helper-files to this directory;
	\item run \texttt{ls /usr/local/bin} to see what is \emph{currently} in there;
	\item run the commands given in \cref{linux-add-to-path};
	\item run \texttt{ls /usr/local/bin} again to check that \texttt{latexindent.pl} and \texttt{defaultSettings.yaml}
	      have been added.
\end{enumerate}
\begin{cmhlistings}[style=yaml,numbers=none]{Add to path from a Linux terminal}{linux-add-to-path}
	sudo apt-get install cmake
	sudo apt-get update && sudo apt-get install build-essential
	mkdir build && cd build
	cmake ../path-helper-files
	sudo make install
\end{cmhlistings}
To \emph{remove} the files, run \texttt{sudo make uninstall}.
\subsection{Add to path for Windows}
To add \texttt{latexindent.exe} to the path for Windows, follow these steps:
\begin{enumerate}
	\item download  \texttt{latexindent.exe}, \texttt{defaultSettings.yaml},  \texttt{add-to-path.bat}
	      from \cite{latexindent-home} to your chosen directory;
	\item open a command prompt and run \texttt{echo %path%} to see what is \emph{currently} in your \texttt variable;
	\item right click on \texttt{add-to-path.bat} and \emph{Run as administrator};
	\item log out, and log back in;
	\item open a command prompt and run \texttt{echo %path%} to check that the appropriate directory has been added to your
	      \texttt.
\end{enumerate}
To \emph{remove} the directory from your \texttt, run \texttt{remove-from-path.bat} as administrator.

\section{Differences from Version 2.1 to 3.0 (and beyond)}
There are a few (small) changes to the interface when comparing Version 2.1 to Version 3.0.
Explicitly, these are:
\begin{description}
	\item[OLD] \texttt{latexindent.pl -o myfile.tex outputfile.tex}
	\item[NEW] \texttt{latexindent.pl -o=outputfile.tex myfile.tex}

	\texttt{latexindent.pl -o outputfile.tex myfile.tex}
\end{description}

\section{The \texttt{-m} switch}\label{sec:m-switch}
\subsection{The phases of \texttt{latexindent.pl}}
latexindent.pl essentially has 3 phases:
\begin{enumerate}
	\item packing, in which code blocks are replaced with unique ids; if the -m switch is active, then during this phase,
	      \begin{itemize}
		      \item line breaks at the beginning of the \emph{body} can be added (BodyStartsOnOwnLine: 1) or removed (BodyStartsOnOwnLine: 0);
		      \item line breaks at the end of the body can be added (see EndStartsOnOwnLine: 1) or removed (see EndStartsOnOwnLine: 0);
		      \item NOTE: let's say that any key (such as, for example, BeginStartsOnOwnLine) is set to 1, and the
		            appropriate <thing> does already begin on its own line and is immediately preceeded by one or more blank line; in this situation,
		            the blank lines will \emph{not} be removed
	      \end{itemize}
	\item indentation, in which white space is added to the begin, body, and end statements;
	\item unpacking, in which unique ids are replaced by their \emph{indented} code blocks; if the -m switch is active, then during this phase,
	      \begin{itemize}
		      \item line breaks before \emph{begin} statements can be added or removed (if BeginStartsOnOwnLine==0);
		      \item line breaks after \emph{end} statements can be removed but \emph{NOT} added (see EndFinishesWithLineBreak).
	      \end{itemize}
\end{enumerate}
With these rules in mind, let's study a few test cases:

\fixthis{go into details about each of the following}
latexindent.pl environments-line-break-conflict.tex -s -t -m -o environments-line-break-conflict-mod1.tex -l=env-conflicts-mod1.yaml
latexindent.pl environments-line-break-conflict-nested.tex -s -t -m -o environments-line-break-conflict-nested-mod-2.tex -l=env-conflicts-mod2.yaml
latexindent.pl environments-line-break-conflict-nested.tex -s -t -m -o environments-line-break-conflict-nested-mod-3.tex -l=env-conflicts-mod3.yaml
environments-first-opt-args.tex, see all of the different examples in test-cases.sh
environments-second-opt-args.tex provides some interesting cases too

The \texttt{\fi} command knows to insert a space, so as to give, for example, \texttt{\fi} text, and avoid things such as \texttt{\fitext}

from yaml

\section{Moving from version 2.* to version 3.*}\label{sec:moving-version}
%# EVERYTHING BELOW HERE IS OBSOLETE IN V3.0  
%# EVERYTHING BELOW HERE IS OBSOLETE IN V3.0  
%# EVERYTHING BELOW HERE IS OBSOLETE IN V3.0  
%# *** NOTE ***
%# If you have specified alwaysLookforSplitBraces: 1
%# and alwaysLookforSplitBrackets: 1 then you don't need
%# to worry about completing
%#
%#       checkunmatched
%#       checkunmatchedELSE
%#       checkunmatchedbracket
%#
%# in other words, you don't really need to edit anything 
%# below this line- it used to be necessary for older 
%# versions of the script, but not anymore :)
%#***      ***
%#
%# always look for split { }, which means that the user doesn't
%# have to complete checkunmatched, checkunmatchedELSE
%
%# if you want to indent if, else, fi constructs such as, for example,
%#
%#       \ifnum#1=2
%#               something
%#       \else
%#               something else
%#       \fi
%#
%# then populate them in constructIfElseFi
%#
%# The following list is by no means exhaustive, but is taken from etex documentation;
%# add your own if commands to your own yaml settings.
%constructIfElseFi:
%    if: 1
%    ifcat: 1
%    ifnum: 1
%    ifdim: 1
%    ifodd: 1
%    ifvmode: 1
%    ifhmode: 1
%    ifmmode: 1
%    ifinner: 1
%    ifvoid: 1
%    ifhbox: 1
%    ifvbox: 1
%    ifx: 1
%    ifeof: 1
%    iftrue: 1
%    ifcase: 1
%    ifdefined: 1
%    ifcsname: 1
%    iffontchar: 1
%
%#noAdditionalIndent:
%#    \[: 0
%#    \]: 0
%
%alwaysLookforSplitBraces: 1
%
%# always look for split [ ], which means that the user doesn't
%# have to complete checkunmatchedbracket
%alwaysLookforSplitBrackets: 1
%
%
%# commands that might split {} across lines
%# such as \parbox, \marginpar, etc
%checkunmatched:
%    parbox: 1
%    vbox: 1
%
%# very similar to %checkunmatched except these 
%# commands might have an else construct
%checkunmatchedELSE:
%    pgfkeysifdefined: 1
%    DTLforeach: 1
%    ifthenelse: 1
%
%# commands that might split []  across lines
%# such as \pgfplotstablecreatecol, etc
%checkunmatchedbracket:
%    pgfplotstablecreatecol: 1
%    pgfplotstablesave: 1
%    pgfplotstabletypeset: 1
%    mycommand: 1
%    psSolid: 1
%
% OLD
%indentAfterHeadings:
%    part:
%       indent: 0
%       level: 1
% NEW:
%indentAfterHeadings:
%    part:
%       indentAfterThisHeading: 0
%       level: 1
