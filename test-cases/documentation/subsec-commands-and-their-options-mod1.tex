% arara: pdflatex: {shell: yes, files: [latexindent]}
\subsection{Commands and the strings between their arguments}
	\label{subsec:commands-string-between}
	The \texttt{command} code blocks will always look for optional (square bracketed) and mandatory (curly braced) arguments which can contain comments, line breaks and `beamer' commands \lstinline!<.*?>!  between them.
	There are switches that can allow them to contain other strings, which we discuss next.

\yamltitle{commandCodeBlocks}*{fields}

	The \texttt{commandCodeBlocks} field \announce*{2017-08-21}*{commandCodeBlocks} contains a few switches detailed in \cref{lst:commandCodeBlocks}.
	%

	\cmhlistingsfromfile[style=commandCodeBlocks]{../defaultSettings.yaml}[width=.8\linewidth,before=\centering,yaml-TCB]{\texttt{commandCodeBlocks}}{lst:commandCodeBlocks}

\yamltitle{roundParenthesesAllowed}{0|1}

	The need for this field was mostly motivated by commands found in code used to generate images in \texttt{PSTricks} and \texttt{tikz}; for example, let's consider the code given in \cref{lst:pstricks1}.

	\begin{minipage}{.45\textwidth}
		\cmhlistingsfromfile[columns=fixed]{demonstrations/pstricks1.tex}{\texttt{pstricks1.tex}}{lst:pstricks1}
	\end{minipage}
	\hfill
	\begin{minipage}{.45\textwidth}
		\cmhlistingsfromfile[columns=fixed]{demonstrations/pstricks1-default.tex}{\texttt{pstricks1} default output}{lst:pstricks1-default}
	\end{minipage}

	Notice that the \lstinline!\defFunction! command has an optional argument, followed by a mandatory argument, followed by a round-parenthesis argument, $(u,v)$.

	By default, because \texttt{roundParenthesesAllowed} is set to $1$ in \cref{lst:commandCodeBlocks}, then \texttt{latexindent.pl} will allow round parenthesis between optional and mandatory arguments.
	In the case of the code in \cref{lst:pstricks1}, \texttt{latexindent.pl} finds \emph{all} the arguments of \lstinline!defFunction!, both before and after \lstinline!(u,v)!.

	The default output from running \texttt{latexindent.pl} on \cref{lst:pstricks1} actually leaves it unchanged (see \cref{lst:pstricks1-default}); note in particular, this is because of \texttt{noAdditionalIndentGlobal} as discussed on \cpageref{page:command:noAddGlobal}.

	Upon using the YAML settings in \cref{lst:noRoundParentheses}, and running the command \begin{commandshell}
latexindent.pl pstricks1.tex -l noRoundParentheses.yaml
        \end{commandshell} we obtain the output given in  \cref{lst:pstricks1-nrp}.

	\begin{minipage}{.45\textwidth}
		\cmhlistingsfromfile{demonstrations/pstricks1-nrp.tex}{\texttt{pstricks1.tex} using \cref{lst:noRoundParentheses}}{lst:pstricks1-nrp}
	\end{minipage}
	\hfill
	\begin{minipage}{.45\textwidth}
		\cmhlistingsfromfile[style=yaml-LST]{demonstrations/noRoundParentheses.yaml}[yaml-TCB]{\texttt{noRoundParentheses.yaml}}{lst:noRoundParentheses}
	\end{minipage}

	Notice the difference between \cref{lst:pstricks1-default} and \cref{lst:pstricks1-nrp}; in particular, in \cref{lst:pstricks1-nrp}, because round parentheses are \emph{not} allowed, \texttt{latexindent.pl} finds that the \lstinline!\defFunction! command finishes at the first opening round parenthesis.
	As such, the remaining braced, mandatory, arguments are found to be \texttt{UnNamedGroupingBracesBrackets} (see \vref{tab:code-blocks}) which, by default, assume indentation for their body, and hence the tabbed indentation in \cref{lst:pstricks1-nrp}.

	Let's explore this using the YAML given in \cref{lst:defFunction} and run the command \begin{commandshell}
latexindent.pl pstricks1.tex -l defFunction.yaml
        \end{commandshell} then the output is as in \cref{lst:pstricks1-indent-rules}.

	\begin{minipage}{.45\textwidth}
		\cmhlistingsfromfile{demonstrations/pstricks1-indent-rules.tex}{\texttt{pstricks1.tex} using \cref{lst:defFunction}}{lst:pstricks1-indent-rules}
	\end{minipage}
	\hfill
	\begin{minipage}{.45\textwidth}
		\cmhlistingsfromfile[style=yaml-LST]{demonstrations/defFunction.yaml}[yaml-TCB]{\texttt{defFunction.yaml}}{lst:defFunction}
	\end{minipage}

	Notice in \cref{lst:pstricks1-indent-rules} that the \emph{body} of the \lstinline!defFunction! command i.e, the subsequent lines containing arguments after the command name, have received the single space of indentation specified by \cref{lst:defFunction}.

\yamltitle{stringsAllowedBetweenArguments}*{fields}
	\texttt{tikz} users may well specify code such as that given in \cref{lst:tikz-node1}; processing this code using \texttt{latexindent.pl} gives the default output in \cref{lst:tikz-node1-default}.

	\begin{minipage}{.45\textwidth}
		\cmhlistingsfromfile[columns=fixed]{demonstrations/tikz-node1.tex}{\texttt{tikz-node1.tex}}{lst:tikz-node1}
	\end{minipage}
	\hfill
	\begin{minipage}{.45\textwidth}
		\cmhlistingsfromfile[columns=fixed]{demonstrations/tikz-node1-default.tex}{\texttt{tikz-node1} default output}{lst:tikz-node1-default}
	\end{minipage}

	With reference to \vref{lst:commandCodeBlocks}, we see that the strings \begin{quote} to, node, ++ \end{quote} are all allowed to appear between arguments, as they are each set to $1$; importantly, you are encouraged to add further names to this field as necessary.
	This means that when \texttt{latexindent.pl} processes \cref{lst:tikz-node1}, it consumes: \begin{itemize} \item the optional argument \lstinline![thin]!
		\item the round-bracketed argument \lstinline!(c)! because \texttt{roundParenthesesAllowed} is $1$ by default
		\item the string \lstinline!to! (specified in \texttt{stringsAllowedBetweenArguments})
		\item the optional argument \lstinline![in=110,out=-90]!
		\item the string \lstinline!++! (specified in \texttt{stringsAllowedBetweenArguments})
		\item the round-bracketed argument \lstinline!(0,-0.5cm)! because \texttt{roundParenthesesAllowed} is $1$ by default
		\item the string \lstinline!node! (specified in \texttt{stringsAllowedBetweenArguments})
		\item the optional argument \lstinline![below,align=left,scale=0.5]! \end{itemize} 

	We can explore this further, for example using \cref{lst:draw} and running the command \begin{commandshell}
latexindent.pl tikz-node1.tex -l draw.yaml  
\end{commandshell} we receive the output given in \cref{lst:tikz-node1-draw}.

	\begin{minipage}{.45\textwidth}
		\cmhlistingsfromfile{demonstrations/tikz-node1-draw.tex}{\texttt{tikz-node1.tex} using \cref{lst:draw}}{lst:tikz-node1-draw}
	\end{minipage}
	\hfill
	\begin{minipage}{.45\textwidth}
		\cmhlistingsfromfile[style=yaml-LST]{demonstrations/draw.yaml}[yaml-TCB]{\texttt{draw.yaml}}{lst:draw}
	\end{minipage}

	Notice that each line after the \lstinline!\draw! command (its `body') in \cref{lst:tikz-node1-draw} has been given the appropriate two-spaces worth of indentation specified in \cref{lst:draw}.

	Let's compare this with the output from using the YAML settings in \cref{lst:no-to}, and running the command \begin{commandshell}
latexindent.pl tikz-node1.tex -l no-to.yaml  
\end{commandshell} given in \cref{lst:tikz-node1-no-to}.

	\begin{minipage}{.45\textwidth}
		\cmhlistingsfromfile{demonstrations/tikz-node1-no-to.tex}{\texttt{tikz-node1.tex} using \cref{lst:no-to}}{lst:tikz-node1-no-to}
	\end{minipage}
	\hfill
	\begin{minipage}{.45\textwidth}
		\cmhlistingsfromfile[style=yaml-LST]{demonstrations/no-to.yaml}[yaml-TCB]{\texttt{no-to.yaml}}{lst:no-to}
	\end{minipage}

	In this case, \texttt{latexindent.pl} sees that: \begin{itemize} \item the \lstinline!\draw! command finishes after the \lstinline!(c)! as (\texttt{stringsAllowedBetweenArguments} has \texttt{to} set to $0$)
		\item it finds a \texttt{namedGroupingBracesBrackets} called \texttt{to} (see \vref{tab:code-blocks}) \emph{with} argument \lstinline![in=110,out=-90]!
		\item it finds another \texttt{namedGroupingBracesBrackets} but this time called \texttt{node} with argument \lstinline![below,align=left,scale=0.5]! \end{itemize} 

\yamltitle{commandNameSpecial}*{fields} There are some special command names \announce*{2017-08-21}{commandNameSpecial} that do not fit within the names recognized by \texttt{latexindent.pl}, the first one of which is \lstinline!\@ifnextchar[!.
	%
	From the perspective of \texttt{latexindent.pl}, the whole of the text \lstinline!\@ifnextchar[! is is a command, because it is immediately followed by sets of mandatory arguments.
	However, without the \texttt{commandNameSpecial} field, \texttt{latexindent.pl} would not be able to label it as such, because the \lstinline![! is, necessarily, not matched by a closing \lstinline!]!.

	For example, consider the sample file in \cref{lst:ifnextchar}, which has default output in \cref{lst:ifnextchar-default}.

	\begin{minipage}{.45\textwidth}
		\cmhlistingsfromfile{demonstrations/ifnextchar.tex}{\texttt{ifnextchar.tex}}{lst:ifnextchar}
	\end{minipage}
	\hfill
	\begin{minipage}{.45\textwidth}
		\cmhlistingsfromfile{demonstrations/ifnextchar-default.tex}{\texttt{ifnextchar.tex} default output}{lst:ifnextchar-default}
	\end{minipage}

	Notice that in \cref{lst:ifnextchar-default} the \texttt{parbox} command has been able to indent its body, because \texttt{latexindent.pl} has successfully found the command \lstinline!\@ifnextchar! first; the pattern-matching of \texttt{latexindent.pl} starts from \emph{the inner most <thing> and works outwards}, discussed in more detail on \cpageref{page:phases}.
	For demonstration, we can compare this output with that given in \cref{lst:ifnextchar-off} in which the settings from \cref{lst:no-ifnextchar} have dictated that \lstinline!\@ifnextchar[! command should not be searched for specially; as such, the \texttt{parbox} command has been \emph{unable} to indent its body successfully, because the \lstinline!\@ifnextchar[! command has not been found.

	\begin{minipage}{.45\textwidth}
		\cmhlistingsfromfile{demonstrations/ifnextchar-off.tex}{\texttt{ifnextchar.tex} using \cref{lst:no-ifnextchar}}{lst:ifnextchar-off}
	\end{minipage}
	\hfill
	\begin{minipage}{.45\textwidth}
		\cmhlistingsfromfile[style=yaml-LST]{demonstrations/no-ifnextchar.yaml}[yaml-TCB]{\texttt{no-ifnextchar.yaml}}{lst:no-ifnextchar}
	\end{minipage}
