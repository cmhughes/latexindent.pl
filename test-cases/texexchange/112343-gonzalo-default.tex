\documentclass[a4,portrait,semrot]{seminar}
\usepackage[T1]{fontenc}
\usepackage[utf8]{inputenc}
\usepackage[ngerman]{babel}
\usepackage[skip=4pt]{caption}
\usepackage{booktabs}
\usepackage{dcolumn}
\usepackage{units}
\usepackage{array}

\pagestyle{empty}
\renewcommand{\printlandscape}{\special{landscape}}
\slideframe{none}
\centerslidesfalse
\slidesmag{3}
\setlength{\slideheight}{183mm}
\setlength{\slidewidth}{264mm}

\makeatletter
\newcommand{\armultirow}[3]{%
	\multicolumn{#1}{#2}{%
		\begin{picture}(0,0)%
			\put(0,0){%
				\begin{tabular}[t]{@{}#2@{}} %
					#3            %
				\end{tabular}%
			}%
		\end{picture}%
	}%
}%

\newcolumntype{f}{>{$}l<{$}}
\newcolumntype{n}{l}
\newcolumntype{N}{>{\scriptsize}l}
\newcolumntype{v}[1]{>{\raggedright\hspace{0pt}}p{#1}}
\newcolumntype{V}[1]{>{\scriptsize\raggedright\hspace{0pt}}p{#1}}
%
% array.sty, dcolumn.sty
\newcolumntype{B}[1]{>{\boldmath\DC@{.}{,}{#1}}l<{\DC@end}}
\newcolumntype{d}[1]{>{\DC@{.}{,}{#1}}l<{\DC@end}}
\newcolumntype{i}[1]{>{\DC@{.}{,}{#1}\mathnormal\bgroup}l<{\egroup\DC@end}}
\newcolumntype{s}[1]{>{\DC@{.}{,}{#1}\mathsf\bgroup}l<{\egroup\DC@end}}
%
% array.sty, rotating.sty
\newcolumntype{R}[1]{%
	>{\begin{turn}{90}\begin{minipage}{#1}\scriptsize\raggedright\hspace{0pt}}l%
					<{\end{minipage}\end{turn}}%
}
%
% array.sty, tabularx.sty
\newcolumntype{x}{>{\scriptsize\raggedright\hspace{0pt}}X}
\makeatother
\begin{document}

\begin{slide*}
	\begin{table}
		\centering
		\caption{Minuskelziffern}
		\label{tab:minuskelziffern}
		\begin{tabular}{@{}v{7em}i{4.0}i{3.0}i{5.0}n@{}}
			\toprule
			&
			\multicolumn{4}{N@{}}{Diese also Sachen} \          \cmidrule(l){2-5}
			&
			\multicolumn{1}{V{5.5em}}{Blick linken sonst endlich} &
			\multicolumn{1}{V{5.5em}}{auf nicht weit Soll des} &
			\multicolumn{1}{V{5em}}{gleich man kann ist} &
			\multicolumn{1}{V{5em}@{}}{weil Sache zu einem} \            &
			&
			\multicolumn{1}{N}{\unit{\%}} \          \cmidrule(r){1-1}\cmidrule(lr){2-2}\cmidrule(lr){3-3}\cmidrule(lr){4-4}%
			\cmidrule(l){5-5}
			\armultirow{1}{@{}v{7em}}{Um hier sonst damit Platz ist gegeben} &
			1991 & 20 & 45637 & \oldstylenums{657} unter  \   & 1992 & 47 & 47916 & \oldstylenums{645} linken \   & 1993 & 65  & 22848 & \oldstylenums{347} nein   \          \addlinespace
			\armultirow{1}{@{}v{7em}}{Durch gehört wollen und} &
			1994 & 87 & 46475 & \oldstylenums{957} einem  \   & 1995 & 95 & 94356 & \oldstylenums{8363} Sache \   & 1996 & 100 & 84646 & \oldstylenums{93635} nein \          \cmidrule(r){1-1}\cmidrule(lr){2-2}\cmidrule(lr){3-3}\cmidrule(lr){4-4} %
			\cmidrule(l){5-5}
			&
			\multicolumn{4}{N@{}}{Gerade langt hinauf sonst nicht gleich
				man} \          \cmidrule(r){1-1}\cmidrule(l){2-5}
			\armultirow{1}{@{}v{7em}}{Um hier damit Platz hat} &
			1796 & 4  & 46032 & \oldstylenums{56} scheidet \  & 1896 & 25 & 38937 & \oldstylenums{746} linken  \  & 1996 & 100 & 83458 & \oldstylenums{48746} eine  \          \bottomrule
		\end{tabular}
	\end{table}
\end{slide*}

\begin{slide*}
	\begin{table}
		\centering
		\footnotesize
		\caption{Kathodenfallableiter}
		\label{tab:kathoden}
		\begin{tabular}{@{}nd{1.1}*{3}{d{1.2}}d{1.1}d{3.2}@{}}
			\toprule
			\multicolumn{1}{@{}N}{Typenbezeichnung} &
			\multicolumn{5}{N}{Spannungsschutz für Netze} &
			\multicolumn{1}{N@{}}{Preis} \            &
			\multicolumn{5}{N}{Leiterspannung an der Einbaustelle} \          \cmidrule(lr){2-6}
			&
			\multicolumn{2}{V{6.5em}}{Nicht geerdeter Sternpunkt} &
			\multicolumn{2}{V{6.5em}}{Starr geerdeter Sternpunkt} &
			\multicolumn{1}{V{4em}}{Nenn"-spannung} \          \cmidrule(lr){2-3}\cmidrule(lr){4-5}
			&
			\multicolumn{1}{V{4.5em}}{Normale Leiterspannung} &
			\multicolumn{1}{V{4.5em}}{Zulässiger Bereich} &
			\multicolumn{1}{V{4.5em}}{Normale Leiterspannung} &
			\multicolumn{1}{V{4.5em}}{Zulässiger Bereich} \            &
			\multicolumn{1}{N}{\unit{kV}} &
			\multicolumn{1}{N}{\unit{kV}} &
			\multicolumn{1}{N}{\unit{kV}} &
			\multicolumn{1}{N}{\unit{kV}} &
			\multicolumn{1}{N}{\unit{kV}} &
			\multicolumn{1}{N}{DM} \          \cmidrule(r){1-1}\cmidrule(lr){2-2}\cmidrule(lr){3-3}\cmidrule(lr){4-4}%
			\cmidrule(lr){5-5}\cmidrule(lr){6-6}\cmidrule(l){7-7}
			H 484--1 & 1 & 1.15 & 1.25 & 1.45 & 1 & 220.$---$ \            H 484--1,5 & 1.5 & 1.8 & 1.9 & 2.2 & 1.5 & 233.$---$ \            H 484--2 & 2 & 2.3 & 2.5 & 2.9 & 2 & 252.$---$ \            H 484--2,5 & 3 & 2.9 & 3.1 & 3.6 & 2.5 & 261.$---$ \            H 484--3 & 3.5 & 3.5 & 3.8 & 4.3 & 3 & 264.$---$ \          \bottomrule
		\end{tabular}
	\end{table}
\end{slide*}

\end{document}
